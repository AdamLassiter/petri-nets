\documentclass{article}

% ~~~~~~~~~~~~~~~~~~~~~~~ Preamble ~~~~~~~~~~~~~~~~~~~~~~~

\usepackage{bpextra}
\usepackage[margin=1.5in]{geometry}
\usepackage{amsmath}
\usepackage{amsthm}
\usepackage{subcaption}
\usepackage{float}
\usepackage{enumitem}
\usepackage{titlesec}
\usepackage{latexsym}


%\titleformat{\section}
%  {\normalfont\scshape}{\thesection}{1em}{}

% no paragraph indent
\setlength{\parindent}{0em}
\setlength{\parskip}{1em}
%\setlength{\columnsep}{2em}

% delta-equals (unused)
\def\deltaeq{\mathrel{\ensurestackMath{\stackon[1pt]{=}{\scriptstyle\Delta}}}}
% define-equals
\def\defeq{::=}
% set-to-equals
\def\seteq{:=}


% indented definitions, lemmas etc
\makeatletter
\newtheoremstyle{indented}
    {5pt}% space before
    {5pt}% space after
    {\addtolength{\@totalleftmargin}{0em}
     \addtolength{\linewidth}{-0em}
     \parshape 1 0em \linewidth}% body font
    {-0em}% indent
    {\bfseries}% header font
    {.}% punctuation
    {\newline}% after theorem header
    {}% header specification (empty for default)
\makeatother

% theorems with global counter
\theoremstyle{indented}
\newtheorem{sec-ctr}{???}[section]
\newtheorem{definition}[sec-ctr]{Definition}
\newtheorem*{definition*}{Definition}
\newtheorem{proposition}[sec-ctr]{Proposition}
\newtheorem*{proposition*}{Proposition}
\newtheorem{lemma}[sec-ctr]{Lemma}
\newtheorem*{lemma*}{Lemma}
\newtheorem{example}[sec-ctr]{Example}
\newtheorem*{example*}{Example}
\newtheorem*{examples}{Examples}
\newtheorem{corollary}[sec-ctr]{Corollary}
\newtheorem*{corollary*}{Corollary}
\newtheorem{remark}[sec-ctr]{Remark}
\newtheorem*{remark*}{Remark}
\newtheorem*{remarks}{Remarks}


% ~~~~~~~~~~~~~~~~~~~~~~~~~~~~~~~~~~~~~~~~~~~~~~~~~~~~~~~~


\title{Natural Proof Search for Classical Logic}
\author{Adam Lassiter\\Department of Computer Science\\University of Bath \and Willem Heijltjes\\Department of Computer Science\\University of Bath}
\date{\today}

\begin{document}

    \maketitle
    \begin{abstract}
        \textbf{We investigate a natural algorithm for proof search within classical logic and prove bounds on the complexity class of such a search.}
    \end{abstract}


% ~~~~~~~~~~~~~~~~~~~~~~~~~~~~~~~~~~~~~~~~~~~~~~~~~~~~~~~~


    \section{Classical Logic}

        \begin{definition}[Formulae]
            A \textit{formula} within classical logic is constructed as follows:
            \begin{align*}
                \quad A, B, C                &\defeq \bot \,|\, \top \,|\, a \,|\, \neg\, a \,|\, A \vee B \,|\, A \wedge B \\
                \quad \Gamma, \Delta, \Sigma &\defeq A_1 \ldots A_n
            \end{align*}
            where $\vee, \wedge$ are additive linear logic disjunction and conjunction respectively and $\Gamma, \Delta, \Sigma$ are contexts..
        \end{definition}

        \begin{example*}
        \end{example*}


        \begin{definition}[Sequent Proofs]
            Within \textit{classical logic}, a \textit{sequent proof} is constructed from the following rules:

            \begin{minipage}[H]{\linewidth}
                \centering
                \begin{minipage}[H]{.3\linewidth}
                    \begin{prooftree}
                        \AxiomC{~}
                        \RightLabel{$\top$}
                        \UnaryInfC{$\vdash \top$}
                    \end{prooftree}
                    \begin{prooftree}
                        \AxiomC{~}
                        \RightLabel{$ax$}
                        \UnaryInfC{$\vdash a, \neg\, a$}
                    \end{prooftree}
                \end{minipage}
                \begin{minipage}[H]{.3\linewidth}
                    \begin{prooftree}
                        \AxiomC{$\vdash \Gamma, A$}
                        \RightLabel{$\vee$R}
                        \UnaryInfC{$\vdash \Gamma, A \vee B$}
                    \end{prooftree}
                    \begin{prooftree}
                        \AxiomC{$\vdash \Gamma, A$}
                        \AxiomC{$\vdash \Gamma, B$}
                        \RightLabel{$\wedge$R}
                        \BinaryInfC{$\vdash \Gamma, A \wedge B$}
                    \end{prooftree}
                \end{minipage}
                \begin{minipage}[H]{.3\linewidth}
                    \begin{prooftree}
                        \AxiomC{$\vdash \Gamma$}
                        \RightLabel{$w$}
                        \UnaryInfC{$\vdash \Gamma, A$}
                    \end{prooftree}
                    \begin{prooftree}
                        \AxiomC{$\vdash \Gamma, A, A$}
                        \RightLabel{$c$}
                        \UnaryInfC{$\vdash \Gamma, A$}
                    \end{prooftree}
                \end{minipage}
            \end{minipage}~\par
            where $A, B, C$ are formulae and $\Gamma, \Delta, \Sigma$ are sequents.
            A sequent proof provides, without context, a proof of its conclusion and each line of the proof represents a tautology.
        \end{definition}

        \begin{example*}
        \end{example*}


        \begin{remark}
            Within the context of weakening and contraction, \textit{additive} and \textit{multiplicative} rules are inter-derivable.
        \end{remark}


        \begin{definition}[Derivations]
            Given \textit{tops} $\Gamma_1 \ldots \Gamma_n$ for the sequent proof $\vdash \Delta$, a \textit{derivation} is a tree providing a proof of $\Gamma_1 \ldots \Gamma_n \implies \Delta$.

            A derivation is written as:
            \begin{prooftree}
                \AxiomC{$\vdash \Gamma_1$}
                \AxiomC{$\ldots$}
                \AxiomC{$\vdash \Gamma_n$}
                \RightLabel{\textit{[label]}} \doubleLine\TrinaryInfC{$\vdash \Delta$}
            \end{prooftree}
            where the \textit{label} describes which rules may be used within the derivation.
        \end{definition}

        \begin{corollary}[Derivation Equivalence]
            A sequent proof is a derivation where all top derivations of the tree are $=\joinrel= \top, ax$.
            Equivalence of derivations may be weakly defined up to equivalence of leaves and conclusion.
        \end{corollary}

        \begin{example*}
        \end{example*}


        \begin{definition}[Additive Stratification]
            A proof tree is said to be \textit{additively stratified} if $\vdash P$ is structured as follows:
            \begin{prooftree}
                \AxiomC{}
                \RightLabel{$\top, ax$}\doubleLine\UnaryInfC{$\vdash A_1$}
                \RightLabel{$w$}\doubleLine\UnaryInfC{$\vdash \Gamma_1$}
                \AxiomC{\ldots}
                \AxiomC{}
                \RightLabel{$\top, ax$}\doubleLine\UnaryInfC{$\vdash A_n$}
                \RightLabel{$w$}\doubleLine\UnaryInfC{$\vdash \Gamma_n$}
                \RightLabel{$\wedge, \vee$}\doubleLine\TrinaryInfC{$\vdash P \ldots P$}
                \RightLabel{$c$}\doubleLine\UnaryInfC{$\vdash P$}
            \end{prooftree}
            That is, the inferences made in an additively stratified proof are strictly ordered by:
            \setlist{nolistsep}
            \begin{enumerate}[noitemsep]
                \item Top/Axiomatic
                \item Weakening
                \item Conjunction/Disjunction
                \item Contraction
            \end{enumerate}
        \end{definition}
        
        \begin{example*}
        \end{example*}


        \begin{proposition}[Stratification Equivalence]
            Given $\vdash A$, there exists an additively stratified proof of $A$.
        \end{proposition}
        \begin{proof}
            For each instance of a weakening below another inference, there exists an equivalent subproof that is additively stratified:

            \begin{minipage}[H]{\linewidth}
                \centering
                \begin{minipage}[H]{0.4\linewidth}
                    \begin{prooftree}
                        \AxiomC{$\vdash \Gamma, A, B$}
                        \RightLabel{$\vee$}\UnaryInfC{$\vdash \Gamma, A \vee B$}
                        \RightLabel{$w$}\UnaryInfC{$\vdash \Gamma, A \vee B$, C}
                    \end{prooftree}
                \end{minipage}
                $\leadsto$
                \begin{minipage}[H]{0.4\linewidth}
                    \begin{prooftree}
                        \AxiomC{$\vdash \Gamma, A, B$}
                        \RightLabel{$w$}\UnaryInfC{$\vdash \Gamma, A, B, C$}
                        \RightLabel{$\vee$}\UnaryInfC{$\vdash \Gamma, A \vee B$, C}
                    \end{prooftree}
                \end{minipage}
            \end{minipage}
            
            \begin{minipage}[H]{\linewidth}
                \centering
                \begin{minipage}[H]{0.4\linewidth}
                    \begin{prooftree}
                        \AxiomC{$\vdash \Gamma, A$}
                        \AxiomC{$\vdash \Gamma, B$}
                        \RightLabel{$\wedge$}\BinaryInfC{$\vdash \Gamma, A \wedge B$}
                        \RightLabel{$w$}\UnaryInfC{$\vdash \Gamma, A \wedge B, C$}
                    \end{prooftree}
                \end{minipage}
                $\leadsto\quad$
                \begin{minipage}[H]{0.4\linewidth}
                    \begin{prooftree}
                        \AxiomC{$\vdash \Gamma, A$}
                        \RightLabel{$w$}\UnaryInfC{$\vdash \Gamma, A, C$}
                        \AxiomC{$\vdash \Gamma, B$}
                        \RightLabel{$w$}\UnaryInfC{$\vdash \Gamma, B, C$}
                        \RightLabel{$\wedge$}\BinaryInfC{$\vdash \Gamma, A \wedge B, C$}
                    \end{prooftree}
                \end{minipage}
            \end{minipage}

            \begin{minipage}[H]{\linewidth}
                \centering
                \begin{minipage}[H]{0.4\linewidth}
                    \begin{prooftree}
                        \AxiomC{$\vdash \Gamma, A, A$}
                        \RightLabel{$c$}\UnaryInfC{$\vdash \Gamma, A$}
                        \RightLabel{$w$}\UnaryInfC{$\vdash \Gamma, A, B$}
                    \end{prooftree}
                \end{minipage}
                $\leadsto\quad$
                \begin{minipage}[H]{0.4\linewidth}
                    \begin{prooftree}
                        \AxiomC{$\vdash \Gamma, A, A$}
                        \RightLabel{$w$}\UnaryInfC{$\vdash \Gamma, A, A, B$}
                        \RightLabel{$c$}\UnaryInfC{$\vdash \Gamma, A, B$}
                    \end{prooftree}
                \end{minipage}
            \end{minipage}

            Similarly, for each instance of a contraction above another inference, there exists an equivalent subproof that is additively stratified:

            \begin{minipage}[H]{\linewidth}
                \centering
                \begin{minipage}[H]{0.4\linewidth}
                    \begin{prooftree}
                        \AxiomC{$\vdash \Gamma, A, A, B$}
                        \RightLabel{$c$}\UnaryInfC{$\vdash \Gamma, A, B$}
                        \RightLabel{$\vee$}\UnaryInfC{$\vdash \Gamma, A \vee B$}
                    \end{prooftree}
                \end{minipage}
                $\leadsto$
                \begin{minipage}[H]{0.4\linewidth}
                    \begin{prooftree}
                        \AxiomC{$\vdash \Gamma, A, A, B$}
                        \RightLabel{$w$}\UnaryInfC{$\vdash \Gamma, A, A, B, B$}
                        \RightLabel{$\vee$}\UnaryInfC{$\vdash \Gamma, A \vee B, A, B$}
                        \RightLabel{$\vee$}\UnaryInfC{$\vdash \Gamma, A \vee B, A \vee B$}
                        \RightLabel{$c$}\UnaryInfC{$\vdash \Gamma, A \vee B$}
                    \end{prooftree}
                \end{minipage}
            \end{minipage}

            \begin{minipage}[H]{\linewidth}
                \centering
                \begin{minipage}[H]{0.3\linewidth}
                    \begin{prooftree}
                        \AxiomC{$\vdash \Gamma, A, A$}
                        \RightLabel{$c$}\UnaryInfC{$\vdash \Gamma, A$}
                        \AxiomC{$\vdash \Gamma, B$}
                        \RightLabel{$\wedge$}\BinaryInfC{$\vdash \Gamma, A \wedge B$}
                    \end{prooftree}
                \end{minipage}
                $\leadsto\quad$
                \begin{minipage}[H]{0.6\linewidth}
                    \begin{prooftree}
                        \AxiomC{$\vdash \Gamma, A, A$}
                        \AxiomC{$\vdash \Gamma, B$}
                        \RightLabel{$w$}\UnaryInfC{$\vdash \Gamma, A, B$}
                        \RightLabel{$\wedge$}\BinaryInfC{$\vdash \Gamma, A, A \wedge B$}
                        \AxiomC{$\vdash \Gamma, B$}
                        \RightLabel{$w$}\UnaryInfC{$\vdash \Gamma, B, A \wedge B$}
                        \RightLabel{$\wedge$}\BinaryInfC{$\vdash \Gamma, A \wedge B, A \wedge B$}
                        \RightLabel{$c$}\UnaryInfC{$\vdash \Gamma, A \wedge B$}
                    \end{prooftree}
                \end{minipage}
            \end{minipage}
            
            By induction from the leaves downwards on a finite height tree, apply the associated rule to each pair of inferences of the form ($c$ above \textit{inf}).
            Any given $\vdash P$ may be rewritten:
            \begin{prooftree}
                \AxiomC{}
                \RightLabel{$\top, ax$}\doubleLine\UnaryInfC{$\vdash A_1$}
                \RightLabel{$\wedge, \vee, w$}\doubleLine\UnaryInfC{$\vdash \Gamma_1$}
                \AxiomC{\ldots}
                \AxiomC{}
                \RightLabel{$\top, ax$}\doubleLine\UnaryInfC{$\vdash A_n$}
                \RightLabel{$\wedge, \vee, w$}\doubleLine\UnaryInfC{$\vdash \Gamma_n$}
                \RightLabel{$c$}\doubleLine\TrinaryInfC{$\vdash P$}
            \end{prooftree}
            
            Again, by induction from the root upwards on this partially stratified tree, apply the associated rule to each pair of inferences of the form ($w$ below \textit{inf}).
            $\vdash P$ may then be further rewritten:
            \begin{prooftree}
                \AxiomC{}
                \RightLabel{$\top, ax$}\doubleLine\UnaryInfC{$\vdash A_1$}
                \RightLabel{$w$}\doubleLine\UnaryInfC{$\vdash \Gamma_1$}
                \AxiomC{\ldots}
                \AxiomC{}
                \RightLabel{$\top, ax$}\doubleLine\UnaryInfC{$\vdash A_n$}
                \RightLabel{$w$}\doubleLine\UnaryInfC{$\vdash \Gamma_n$}
                \RightLabel{$\wedge, \vee$}\doubleLine\TrinaryInfC{$\vdash P \ldots P$}
                \RightLabel{$c$}\doubleLine\UnaryInfC{$\vdash P$}
            \end{prooftree}

        \end{proof}

        \begin{example*}
        \end{example*}
 

    % ~~~~~~~~~~~~~~~~~~~~~~~~~~~~~~~~~~~~~~~~~~~~~~~~~~~~~~~~


    \section{Coalescence}
        
        \begin{definition}[Petri Nets]
            For the purposes required here, a \textit{petri net} $\mathcal{N}$ is $(\mathcal{P, F})$ where $f \in \mathcal{F} : \mathcal{P}^m \times \mathcal{P}$.
            In particular, $\mathcal{P}$ is a set of places and $\mathcal{F}$ a set of flows.
            A \textit{configuration} is a set $\mathcal{C} \subset \mathcal{P}$ of tokens in places.
        \end{definition}

        \begin{example*}
        \end{example*}


        \begin{definition}[Firing Petri Nets]
            Given a petri net $\mathcal{N}$ and configuration $\mathcal{C}$, a \textit{firing} of the net $\mathcal{N}$ is a new configuration generated by application of a transition $f \in \mathcal{F}$ on $m$ tokens $c_1 \ldots c_n \in \mathcal{C}$.
            In particular, $(\mathcal{N = (P, F), C}) \mapsto (\mathcal{N, C} \cup f_{right} \setminus f_{left})$ for some $f = (f_{left}, f_{right}) \in \mathcal{F}$
            
            A petri net is said to be \textit{exhaustively fired} if it is fired until there does not exist any such $f \in \mathcal{F}$ to fire.
        \end{definition}

        \begin{example*}
        \end{example*}

        \begin{remark}
            Implementing fireable petri nets is straightforward using an $n$-dimensional boolean array of places visited and a collection of $n$-tuples representing tokens.
        \end{remark}

        
        \begin{definition}[Coalescence]
            Given a formula $P$, the coalescence algorithm is as follows:
            \begin{enumerate}[noitemsep]
                \item Set  $n \seteq 1$
                \item Construct a $n$-dimensional petri net $\mathcal{N}$ from $P$ where each subformula is a place, each conjunction and disjunction a flow
                \item Construct a configuration $C$ with a token at each place $p = (\ldots, a, \ldots, \neg a, \ldots)$ the intersection of a pair of tautological atoms
                \item Exhaustively fire the petri net $\mathcal{N}$ using the \textit{spawning} method.
                \item If there exists a token in the configuration $\mathcal{C^*}$ at the root of the formula $P$, halt and return $n$
                \item Otherwise, increment $n \seteq n + 1$ and go to step 2
            \end{enumerate}
        \end{definition}

        \begin{example*}
        \end{example*}


        \begin{proposition}[Coalescence Proof Search]
            The coalescence algorithm on $P$ is exactly a proof search on $P$.
        \end{proposition}

        \begin{proof}
            In particular, consider the additively stratified $\vdash P$ with $n - 1$ contractions (should such a proof exist).
            The first configuration of tokens is precisely a proof all possible combinations of the \textit{axiom} rule with $n - 3$ other terms through the \textit{weakening}.
            Each flow transition followed when fired is an application of either the $\vee$ or $\wedge$ rule.
            Finally, a token at the root of the formula is $\vdash P \ldots P$, with the \textit{contraction} rule applied implicitly.

            If there exists an additively stratified proof of $P$, the coalescence algorithm will find it.
            Since for every proof there exists an additively stratified proof, coalescence is precisely proof search.
        \end{proof}
 

    % ~~~~~~~~~~~~~~~~~~~~~~~~~~~~~~~~~~~~~~~~~~~~~~~~~~~~~~~~


    \section{Dimensionality}
        
        \begin{definition}[Dimensionality]
            Given a formula $P$, the coalescence proof search produces a proof in an $n$ dimensional petri net.
            The dimensionality of $P$ is then defined $\dim(P) \defeq n$.
            Equivalently, an additively stratified sequent proof requires $n - 1$ contractions at the bottom of the proof.
            Given $\vdash P$, its dimensionality is defined $\dim(P) \defeq n$.
        \end{definition}

        \begin{proposition}[Deducing Dimensionality]
            Given formulae $P, Q$ such that $\exists \vdash P$ with $\dim(P) = n$ and $\exists \vdash Q$ with $\dim(Q) = m$, then:
            \begin{align*}
                \dim(P \vee Q) &= \min(n, m) \\
                \dim(P \wedge Q) &= \max(n, m)
            \end{align*}
        \end{proposition}

        \begin{proof}
            Omitted
        \end{proof}


        \begin{remark*}[Satisfiability vs Provability]
            For any formula $P$, there exist four distinct classes: \textit{true, false, satisfiably true, satisfiably false}, where satisfiable differs by finding a particular assignment of values to each variable.
            Coalescence searches for a proof of $P$ in \textit{true}, whereas the SAT problem addresses a proof of $P$ in \textit{true}.
            In particular, \textit{true} is the complement class to \textit{satisfiably false} and \textit{false} complement to \textit{satisfiably true}. % TODO: Is this true?
        \end{remark*}
        
        
        \begin{definition}[Dimensionality when not Provable]
            Given a formula $P$ such that there does not exist $\vdash P$, for all formulae $Q$ such that there does exist $\vdash Q$, the dimensionality of $P$ is defined as the least $n$ such that:
            \begin{equation*}
                \dim(P) \defeq n \implies 
                \begin{cases}
                    \exists \vdash (P \vee Q) \implies \dim(P \vee Q) = \min(n, \dim(Q)) \\
                    \exists \vdash (P \wedge Q) \implies \dim(P \wedge Q) = \max(n, \dim(Q))
                \end{cases}
            \end{equation*}
        \end{definition}


        \begin{remark*}
            This definition of a `partial dimensionality' amounts to finding the highest dimension in coalescence proof search that some useful deduction was made, outside of trivial cases.
        \end{remark*}


        \begin{definition}[Paths in a Tree]
            Path from root to any leaf (effectively iterating leaves, but tracing parents)
        \end{definition}


        \begin{proposition}[$\vee$-Bound on Dimensionality]
            \begin{align*}
                \dim(P) &\leq 1 + \#\{\vee \in P\} \\
                \dim(P) &\leq 1 + \max\{\#\{\vee \in path\} \, \forall \, path \in tree(P)\}
            \end{align*}
        \end{proposition}

        \begin{proof}
            Omitted
        \end{proof}


        \begin{proposition}[$ax$-Bound on Dimensionality]
            \begin{align*}
                \dim(P) &\leq 1 + \#\{ax\textit{-Rule} \in \,\, \vdash P\} \\
                \dim(P) &\leq 1 + \#\{vars \in P\}
            \end{align*}
        \end{proposition}

        \begin{proof}
            The first case is trivial.
            Given a formula $P$ and a constructed sequent proof $\vdash P$ through coalescence, additive stratification ensures that $\dim(P) = n \implies leaves(P) \geq n$.
            As each of the $n$ branches of the tree moves `upwards', they must either terminate at an $ax$-rule or branch, leaving $n + 1$ branches total.
            Subsequently, for each $ax$-rule forming a leaf of the tree, $\dim(P)$ increases by no more than one.
            Finally, consider a base case such as $P \seteq \vdash a \vee \neg a$ where $\dim(P) = 2 \leq 1 + \#\{ax-Rule \in \vdash P\}$ (in this case, it is equal).
            
            Second case omitted
        \end{proof}

    
\end{document}
