\documentclass[draft]{article}

% ~~~~~~~~~~~~~~~~~~~~~~~ Preamble ~~~~~~~~~~~~~~~~~~~~~~~

\usepackage{amsmath}
\usepackage{amssymb}
\usepackage{amsthm}
\usepackage{bpextra}
\usepackage{enumitem}
\usepackage{float}
\usepackage[margin=1.5in]{geometry}
\usepackage[hidelinks]{hyperref}
\usepackage{latexsym}
\usepackage{natbib}
\usepackage{subcaption}
\usepackage{tikz}
\usepackage{tkz-berge}
\usepackage{titlesec}
\usepackage{varwidth}

% anti-tikz things
\usetikzlibrary{external}
\tikzexternalize

% tikz things
\usetikzlibrary{petri, topaths, positioning, decorations.pathmorphing}

% natbib source
\bibliographystyle{custom}
\renewcommand{\bibsection}{}

\pgfarrowsdeclare{arr}{arr}{
  \setlength{\arrowsize}{0.6pt}
  \addtolength{\arrowsize}{.5\pgflinewidth}
  \pgfarrowsrightextend{1\arrowsize}
  \pgfarrowsleftextend{1\arrowsize}
}{
  \setlength{\arrowsize}{0.6pt}
  \addtolength{\arrowsize}{.5\pgflinewidth}
  \pgfpathmoveto{\pgfpoint{-5\arrowsize}{4\arrowsize}}
  \pgfpathlineto{\pgfpointorigin}
  \pgfpathlineto{\pgfpoint{-5\arrowsize}{-4\arrowsize}}
  \pgfusepathqstroke
}

\newcommand{\drawsquig}{\draw[-arr,
line join=round,
decorate, decoration={
    zigzag,
    segment length=4,
    amplitude=.9,post=lineto,
    post length=2pt
}]}


\titlespacing{\section}{0pt}{12pt plus 4pt minus 2pt}{0pt plus 2pt minus 2pt}
%\titleformat{\section}
%  {\normalfont\scshape}{\thesection}{1em}{}

% no paragraph indent
\setlength{\parindent}{0em}
\setlength{\parskip}{1em}
%\setlength{\columnsep}{2em}

% delta-equals (unused)
\def\deltaeq{\mathrel{\ensurestackMath{\stackon[1pt]{=}{\scriptstyle\Delta}}}}
% define-equals
\def\defeq{::=}
% set-to-equals
\def\seteq{:=}

% scalable proof tree
\newenvironment{scprooftree}[1]%
  {\gdef\scalefactor{#1}\begin{center}\proofSkipAmount \leavevmode}%
  {\scalebox{\scalefactor}{\DisplayProof}\proofSkipAmount \end{center} }


% indented definitions, lemmas etc
\makeatletter
\newtheoremstyle{indented}
    {15pt}% space before
    {5pt}% space after
    {\addtolength{\@totalleftmargin}{0em}
     \addtolength{\linewidth}{-0em}
     \parshape 1 0em \linewidth}% body font
    {-0em}% indent
    {\bfseries}% header font
    {.}% punctuation
    {\newline}% after theorem header
    {}% header specification (empty for default)
\makeatother


% theorems with global counter
\theoremstyle{indented}
\newtheorem{sec-ctr}{???}[section]
\newtheorem{definition}[sec-ctr]{Definition}
\newtheorem*{definition*}{Definition}
\newtheorem{proposition}[sec-ctr]{Proposition}
\newtheorem*{proposition*}{Proposition}
\newtheorem{lemma}[sec-ctr]{Lemma}
\newtheorem*{lemma*}{Lemma}
\newtheorem*{example}{Example}
\newtheorem*{examples}{Examples}
\newtheorem{corollary}[sec-ctr]{Corollary}
\newtheorem*{corollary*}{Corollary}
\newtheorem{remark}[sec-ctr]{Remark}
\newtheorem*{remark*}{Remark}
\newtheorem*{remarks}{Remarks}


% ~~~~~~~~~~~~~~~~~~~~~~~~~~~~~~~~~~~~~~~~~~~~~~~~~~~~~~~~


\title{Natural Proof Search for Classical Logic}
\author{Adam Lassiter\\Department of Computer Science\\University of Bath \and Willem Heijltjes\\Department of Computer Science\\University of Bath}
\date{\today}

\begin{document}

    \maketitle
    \begin{abstract}
        \textbf{
            We investigate a natural algorithm for proof search within classical logic and prove bounds on the complexity class of such a search.
            We further examine natural optimisations to this algorithm and how they affect complexity bounds.
        }
    \end{abstract}

    \chapter{Introduction}
    
    This paper investigates a natural proof search introduced by \citet{petri-nets}.
    Given a formula composed of conjunctions and disjunctions of variables, it falls into one of three categories: provably true, satisfiably true/false, provably false.
    Satisfiably true or false terms are equivalent to those studied in the boolean satisfiability problem (SAT).
    Since the set of provably false formulae forms the complement to the set of satisfiably true formulae, with provably true similarly forming the complement to satisfiably false, the question of proof search and its complexity ties directly into problems of P versus NP\@.
    The motivating question was `What is the complexity of the chosen proof search'.
    
    The paper begins with a run-through of classical logic and sequent proofs before examining the proof search algorithm in question.
    The research entails studying the properties of this algorithm and implementation details that affect computational complexity, in particular a natural optimisation that gives great performance benefits for certain classes of formulae.
    Finally, the essence of the problem is dissected --- how can the complexity of proof search be bounded by the structure of the formula?
    
    A C implementation of the described algorithm complements the research and may be found at \url{https://gitlab.com/adamlassiter/petri-nets}.
    This is broken down into:
    \begin{itemize}
        \item An additive linear logic (ALL) formula parser with some additional inferences for some operators ($(a \implies b) \equiv (\neg a \vee b)$ etc.)
        \item An implementation of the coalescence algorithm using a derivative of petri nets, including a described optimisation on the complexity bound for certain formulae
        \item A system to backtrack over petri nets fired through coalescence to their equivalent sequent proofs --- a bug remains present in this code where proofs are occasionally backtracked incorrectly
    \end{itemize}
    While this remains somewhat of a work in progress, the code as is provides consistent results and moderately good performance --- when benchmarked against Naoyuki Tamura's Prolog sequent prover\footnote{\url{http://bach.istc.kobe-u.ac.jp/seqprover/}} saw a decrease in runtime of up to 25x for some formulae.


    \chapter{Classical Logic}

    \begin{definition*}[Formulae]
        A \textit{formula} within classical logic is constructed as follows:
        \begin{align*}
            A, B, C                \quad &\defeq \quad \top \,|\, \bot \,|\, a \,|\, \neg\, a \,|\, A \vee B \,|\, A \wedge B \\
            \Gamma, \Delta, \Sigma \quad &\defeq \quad A \,|\, A,B \,|\, A,B,C \ldots
        \end{align*}
        where $\vee, \wedge$ are additive linear logic disjunction and conjunction respectively and $\Gamma, \Delta, \Sigma$ are contexts.
    \end{definition*}

    \begin{example*}
        Consider the formulae constructed as follows:
        \begin{equation*}
            A \seteq a \vee b \quad B \seteq \neg b \vee c \quad C \seteq A \wedge B \equiv (a \vee b) \wedge (\neg b \vee c)
        \end{equation*}
    \end{example*}


    \begin{definition*}[Sequent Proofs]
        Within \textit{classical logic}, a \textit{sequent proof} is constructed from the following rules:

        \begin{minipage}[H]{\linewidth}
            \centering
            \begin{minipage}[H]{.3\linewidth}
                \begin{prooftree}
                    \AxiomC{~}
                    \RightLabel{$\top$}
                    \UnaryInfC{$\vdash \top$}
                \end{prooftree}
                \begin{prooftree}
                    \AxiomC{~}
                    \RightLabel{$ax$}
                    \UnaryInfC{$\vdash a, \neg\, a$}
                \end{prooftree}
            \end{minipage}
            \begin{minipage}[H]{.3\linewidth}
                \begin{prooftree}
                    \AxiomC{$\vdash \Gamma, A$}
                    \RightLabel{$\vee$}
                    \UnaryInfC{$\vdash \Gamma, A \vee B$}
                \end{prooftree}
                \begin{prooftree}
                    \AxiomC{$\vdash \Gamma, A$}
                    \AxiomC{$\vdash \Gamma, B$}
                    \RightLabel{$\wedge$}
                    \BinaryInfC{$\vdash \Gamma, A \wedge B$}
                \end{prooftree}
            \end{minipage}
            \begin{minipage}[H]{.3\linewidth}
                \begin{prooftree}
                    \AxiomC{$\vdash \Gamma$}
                    \RightLabel{$w$}
                    \UnaryInfC{$\vdash \Gamma, A$}
                \end{prooftree}
                \begin{prooftree}
                    \AxiomC{$\vdash \Gamma, A, A$}
                    \RightLabel{$c$}
                    \UnaryInfC{$\vdash \Gamma, A$}
                \end{prooftree}
            \end{minipage}
        \end{minipage}

        where $A, B, C$ are formulae and $\Gamma, \Delta, \Sigma$ are sequents.
        A sequent proof provides, without context, a proof of its conclusion and each line of the proof represents a tautology.
    \end{definition*}

    \begin{example*}
        Consider the formula $P \seteq (a \vee \neg a) \wedge \top$.
        The sequent proof of $P$, written $\vdash P$, is constructed as follows:
        \begin{prooftree}
            \AxiomC{$~$}
            \RightLabel{$ax$}\UnaryInfC{$\vdash a, \neg a$}
            \RightLabel{$\vee R$}\UnaryInfC{$\vdash a \vee \neg a, \neg a$}
            \RightLabel{$\vee R$}\UnaryInfC{$\vdash a \vee \neg a, a \vee \neg a$}
            \RightLabel{$c$}\UnaryInfC{$\vdash a \vee \neg a$}
            \AxiomC{$~$}
            \RightLabel{$\top$}\UnaryInfC{$\vdash \top$}
            \RightLabel{$\wedge R$}\BinaryInfC{$\vdash (a \vee \neg a) \wedge \top$}
        \end{prooftree}
    \end{example*}


    \begin{remark*}
        Within the context of weakening and contraction, \textit{additive} and \textit{multiplicative} rules of linear logic are inter-derivable.
        In this case, the \textit{additive} rules are used, with the effect of $\vee R, \wedge R$ maintaining the number of formulae in a seqent across derivation steps.
    \end{remark*}


    \begin{definition*}[Derivations]
        Given \textit{tops} $\Gamma_1 \ldots \Gamma_n$ for the sequent proof $\vdash \Delta$, a \textit{derivation} is a tree providing a proof of $\Gamma_1 \ldots \Gamma_n \implies \Delta$.

        A derivation is written as:
        \begin{prooftree}
            \AxiomC{$\vdash \Gamma_1$}
            \AxiomC{$\ldots$}
            \AxiomC{$\vdash \Gamma_n$}
            \RightLabel{\textit{[label]}} \doubleLine\TrinaryInfC{$\vdash \Delta$}
        \end{prooftree}
        where the \textit{label} describes which rules may be used within the derivation.
    \end{definition*}

    \begin{corollary*}[Derivation Equivalence]
        A sequent proof is a derivation where all top derivations of the tree are $=\joinrel= \top, ax$.
        Equivalence of derivations may be weakly defined up to equivalence of leaves and conclusion.

        This is described in detail by \citet{proofs-and-types} as \textit{morally equivalent}.
    \end{corollary*}

    \begin{example*}
        Considering the leaves $\vdash A, A$ and $\vdash B$ with the conclusion $\vdash A \wedge B$, the following proofs are morally equivalent:
        \begin{prooftree}
            \AxiomC{$\vdash A, A$}
            \RightLabel{$c$}\UnaryInfC{$\vdash A$}
            \AxiomC{$\vdash B$}
            \RightLabel{$\wedge$}\BinaryInfC{$\vdash A \wedge B$}
        \end{prooftree}
        \begin{prooftree}
            \AxiomC{$\vdash A, A$}
            \AxiomC{$\vdash B$}
            \RightLabel{$w$}\UnaryInfC{$\vdash A, B$}
            \RightLabel{$\wedge$}\BinaryInfC{$\vdash A, A \wedge B$}
            \AxiomC{$\vdash B$}
            \RightLabel{$w$}\UnaryInfC{$\vdash B, A \wedge B$}
            \RightLabel{$\wedge$}\BinaryInfC{$\vdash A \wedge B, A \wedge B$}
            \RightLabel{$c$}\UnaryInfC{$\vdash A \wedge B$}
        \end{prooftree}
        Note that the collection of leaves is a \texttt{Set}, so equivalence is up to existence of terms only and not number.
        Furthermore, equivalence of leaves is also defined up to equivalence of sequents, in particular equivalence up to commutativity of formulae $\vdash A, B \equiv \,\, \vdash B, A$.
        Equivalence may also be considered up to idempotency of formulae $\vdash A, A \equiv \,\, \vdash A$, but this interferes with the correctness of some definitions.
    \end{example*}


    \begin{definition*}[Additive Stratification]
        A proof tree is said to be \textit{additively stratified} if $\vdash P$ is structured as follows:
        \begin{prooftree}
            \AxiomC{}
            \RightLabel{$\top, ax$}\doubleLine\UnaryInfC{$\vdash A_1$}
            \RightLabel{$w$}\doubleLine\UnaryInfC{$\vdash \Gamma_1$}
            \AxiomC{\ldots}
            \AxiomC{}
            \RightLabel{$\top, ax$}\doubleLine\UnaryInfC{$\vdash A_n$}
            \RightLabel{$w$}\doubleLine\UnaryInfC{$\vdash \Gamma_n$}
            \RightLabel{$\wedge, \vee$}\doubleLine\TrinaryInfC{$\vdash P \ldots P$}
            \RightLabel{$c$}\doubleLine\UnaryInfC{$\vdash P$}
        \end{prooftree}
        That is, the inferences made in an additively stratified proof are strictly ordered by:
        \begin{enumerate}[nosep]
            \item Top/Axiomatic
            \item Weakening
            \item Conjunction/Disjunction
            \item Contraction
        \end{enumerate}
    \end{definition*}
    
    \begin{example}\label{example:add-strat-proof}
        The following proof tree is additively stratified:
        \begin{prooftree}
            \AxiomC{}
            \RightLabel{$ax$}\UnaryInfC{$\vdash a, \neg a$}
            \RightLabel{$\vee$}\UnaryInfC{$\vdash a \vee \neg a, \neg a$}
            \RightLabel{$\vee$}\UnaryInfC{$\vdash a \vee \neg a, a \vee \neg a$}
            \AxiomC{}
            \RightLabel{$\top$}\UnaryInfC{$\vdash \top$}
            \RightLabel{$w$}\UnaryInfC{$\vdash \top, a \vee \neg a$}
            \RightLabel{$\wedge$}\BinaryInfC{$\vdash (a \vee \neg a) \wedge \top, a \vee \neg a$}
            \AxiomC{}
            \RightLabel{$\top$}\UnaryInfC{$\vdash \top$}
            \RightLabel{$w$}\UnaryInfC{$\vdash \top, (a \vee \neg a) \wedge \top$}
            \RightLabel{$\wedge$}\BinaryInfC{$\vdash (a \vee \neg a) \wedge \top, (a \vee \neg a) \wedge \top$}
            \RightLabel{$c$}\UnaryInfC{$\vdash (a \vee \neg a) \wedge \top$}
        \end{prooftree}
    \end{example}


    \begin{proposition*}[Stratification Equivalence]
        Given $\vdash A$, there exists a morally equivalent additively stratified proof of $A$.
    \end{proposition*}

    \begin{proof}
        For each instance of a weakening below another inference, there exists an equivalent subproof that is additively stratified:

        \begin{minipage}[H]{\linewidth}
            \centering
            \begin{minipage}[H]{0.4\linewidth}
                \begin{prooftree}
                    \AxiomC{$\vdash \Gamma, A$}
                    \RightLabel{$\vee$}\UnaryInfC{$\vdash \Gamma, A \vee B$}
                    \RightLabel{$w$}\UnaryInfC{$\vdash \Gamma, A \vee B$, C}
                \end{prooftree}
            \end{minipage}
            $\leadsto$
            \begin{minipage}[H]{0.4\linewidth}
                \begin{prooftree}
                    \AxiomC{$\vdash \Gamma, A$}
                    \RightLabel{$w$}\UnaryInfC{$\vdash \Gamma, A, C$}
                    \RightLabel{$\vee$}\UnaryInfC{$\vdash \Gamma, A \vee B, C$}
                \end{prooftree}
            \end{minipage}
        \end{minipage}
        
        \begin{minipage}[H]{\linewidth}
            \centering
            \begin{minipage}[H]{0.4\linewidth}
                \begin{prooftree}
                    \AxiomC{$\vdash \Gamma, A$}
                    \AxiomC{$\vdash \Gamma, B$}
                    \RightLabel{$\wedge$}\BinaryInfC{$\vdash \Gamma, A \wedge B$}
                    \RightLabel{$w$}\UnaryInfC{$\vdash \Gamma, A \wedge B, C$}
                \end{prooftree}
            \end{minipage}
            $\leadsto\quad$
            \begin{minipage}[H]{0.4\linewidth}
                \begin{prooftree}
                    \AxiomC{$\vdash \Gamma, A$}
                    \RightLabel{$w$}\UnaryInfC{$\vdash \Gamma, A, C$}
                    \AxiomC{$\vdash \Gamma, B$}
                    \RightLabel{$w$}\UnaryInfC{$\vdash \Gamma, B, C$}
                    \RightLabel{$\wedge$}\BinaryInfC{$\vdash \Gamma, A \wedge B, C$}
                \end{prooftree}
            \end{minipage}
        \end{minipage}

        \begin{minipage}[H]{\linewidth}
            \centering
            \begin{minipage}[H]{0.4\linewidth}
                \begin{prooftree}
                    \AxiomC{$\vdash \Gamma, A, A$}
                    \RightLabel{$c$}\UnaryInfC{$\vdash \Gamma, A$}
                    \RightLabel{$w$}\UnaryInfC{$\vdash \Gamma, A, B$}
                \end{prooftree}
            \end{minipage}
            $\leadsto\quad$
            \begin{minipage}[H]{0.4\linewidth}
                \begin{prooftree}
                    \AxiomC{$\vdash \Gamma, A, A$}
                    \RightLabel{$w$}\UnaryInfC{$\vdash \Gamma, A, A, B$}
                    \RightLabel{$c$}\UnaryInfC{$\vdash \Gamma, A, B$}
                \end{prooftree}
            \end{minipage}
        \end{minipage}

        Similarly, for each instance of a contraction above another inference, there exists an equivalent subproof that is additively stratified:

        \begin{minipage}[H]{\linewidth}
            \centering
            \begin{minipage}[H]{0.4\linewidth}
                \begin{prooftree}
                    \AxiomC{$\vdash \Gamma, A, A$}
                    \RightLabel{$c$}\UnaryInfC{$\vdash \Gamma, A$}
                    \RightLabel{$\vee$}\UnaryInfC{$\vdash \Gamma, A \vee B$}
                \end{prooftree}
            \end{minipage}
            $\leadsto$
            \begin{minipage}[H]{0.4\linewidth}
                \begin{prooftree}
                    \AxiomC{$\vdash \Gamma, A, A$}
                    \RightLabel{$\vee$}\UnaryInfC{$\vdash \Gamma, A \vee B, A$}
                    \RightLabel{$\vee$}\UnaryInfC{$\vdash \Gamma, A \vee B, A \vee B$}
                    \RightLabel{$c$}\UnaryInfC{$\vdash \Gamma, A \vee B$}
                \end{prooftree}
            \end{minipage}
        \end{minipage}

        \begin{minipage}[H]{\linewidth}
            \centering
            \begin{minipage}[H]{0.3\linewidth}
                \begin{prooftree}
                    \AxiomC{$\vdash \Gamma, A, A$}
                    \RightLabel{$c$}\UnaryInfC{$\vdash \Gamma, A$}
                    \AxiomC{$\vdash \Gamma, B$}
                    \RightLabel{$\wedge$}\BinaryInfC{$\vdash \Gamma, A \wedge B$}
                \end{prooftree}
            \end{minipage}
            $\leadsto\quad$
            \begin{minipage}[H]{0.6\linewidth}
                \begin{prooftree}
                    \AxiomC{$\vdash \Gamma, A, A$}
                    \AxiomC{$\vdash \Gamma, B$}
                    \RightLabel{$w$}\UnaryInfC{$\vdash \Gamma, A, B$}
                    \RightLabel{$\wedge$}\BinaryInfC{$\vdash \Gamma, A, A \wedge B$}
                    \AxiomC{$\vdash \Gamma, B$}
                    \RightLabel{$w$}\UnaryInfC{$\vdash \Gamma, B, A \wedge B$}
                    \RightLabel{$\wedge$}\BinaryInfC{$\vdash \Gamma, A \wedge B, A \wedge B$}
                    \RightLabel{$c$}\UnaryInfC{$\vdash \Gamma, A \wedge B$}
                \end{prooftree}
            \end{minipage}
        \end{minipage}
        
        By induction from the leaves downwards on a finite height tree, apply the associated rule to each pair of inferences of the form ($c$ above \textit{inf}).
        Any given $\vdash P$ may be rewritten:
        \begin{prooftree}
            \AxiomC{}
            \RightLabel{$\top, ax$}\doubleLine\UnaryInfC{$\vdash A_1$}
            \RightLabel{$\wedge, \vee, w$}\doubleLine\UnaryInfC{$\vdash \Gamma_1$}
            \AxiomC{\ldots}
            \AxiomC{}
            \RightLabel{$\top, ax$}\doubleLine\UnaryInfC{$\vdash A_n$}
            \RightLabel{$\wedge, \vee, w$}\doubleLine\UnaryInfC{$\vdash \Gamma_n$}
            \RightLabel{$c$}\doubleLine\TrinaryInfC{$\vdash P$}
        \end{prooftree}
        
        Again, by induction from the root upwards on this partially stratified tree, apply the associated rule to each pair of inferences of the form ($w$ below \textit{inf}).
        $\vdash P$ may then be further rewritten:
        \begin{prooftree}
            \AxiomC{}
            \RightLabel{$\top, ax$}\doubleLine\UnaryInfC{$\vdash A_1$}
            \RightLabel{$w$}\doubleLine\UnaryInfC{$\vdash \Gamma_1$}
            \AxiomC{\ldots}
            \AxiomC{}
            \RightLabel{$\top, ax$}\doubleLine\UnaryInfC{$\vdash A_n$}
            \RightLabel{$w$}\doubleLine\UnaryInfC{$\vdash \Gamma_n$}
            \RightLabel{$\wedge, \vee$}\doubleLine\TrinaryInfC{$\vdash P \ldots P$}
            \RightLabel{$c$}\doubleLine\UnaryInfC{$\vdash P$}
        \end{prooftree}

    \end{proof}

    \begin{example*}
        Similarly to Example~\ref{example:add-strat-proof}, consider the two following morally equivalent proofs, with only the latter additively stratified.
        \begin{prooftree}
            \AxiomC{}
            \RightLabel{$ax$}\UnaryInfC{$\vdash a, \neg a$}
            \RightLabel{$\vee$}\UnaryInfC{$\vdash a \vee \neg a, \neg a$}
            \RightLabel{$\vee$}\UnaryInfC{$\vdash a \vee \neg a, a \vee \neg a$}
            \RightLabel{$w$}\UnaryInfC{$\vdash a \vee \neg a$}
            \AxiomC{}
            \RightLabel{$\top$}\UnaryInfC{$\vdash \top$}
            \RightLabel{$\wedge$}\BinaryInfC{$\vdash (a \vee \neg a) \wedge \top$}
        \end{prooftree}
        \begin{prooftree}
            \AxiomC{}
            \RightLabel{$ax$}\UnaryInfC{$\vdash a, \neg a$}
            \RightLabel{$\vee$}\UnaryInfC{$\vdash a \vee \neg a, \neg a$}
            \RightLabel{$\vee$}\UnaryInfC{$\vdash a \vee \neg a, a \vee \neg a$}
            \AxiomC{}
            \RightLabel{$\top$}\UnaryInfC{$\vdash \top$}
            \RightLabel{$w$}\UnaryInfC{$\vdash \top, a \vee \neg a$}
            \RightLabel{$\wedge$}\BinaryInfC{$\vdash (a \vee \neg a) \wedge \top, a \vee \neg a$}
            \AxiomC{}
            \RightLabel{$\top$}\UnaryInfC{$\vdash \top$}
            \RightLabel{$w$}\UnaryInfC{$\vdash \top, (a \vee \neg a) \wedge \top$}
            \RightLabel{$\wedge$}\BinaryInfC{$\vdash (a \vee \neg a) \wedge \top, (a \vee \neg a) \wedge \top$}
            \RightLabel{$c$}\UnaryInfC{$\vdash (a \vee \neg a) \wedge \top$}
        \end{prooftree}
        Note that moral equivalence here is without regard to number --- there are two leaves $\vdash \top$ under additive stratification versus one otherwise.
        Furthermore, notice that the proof without additive stratification is shorter --- this will be of importance when improving asymptotic performance.
    \end{example*}

 
    \section{Coalescence}
    
    \begin{definition*}[Petri Nets]
        For the purposes required here, a \textit{petri net} $\mathcal{N}$ is $(\mathcal{P, F})$ where $f \in \mathcal{F} : \mathcal{P}^m \times \mathcal{P}$.
        In particular, $\mathcal{P}$ is a set of places and $\mathcal{F}$ a set of flows or transitions.
        A \textit{configuration} is a set $\mathcal{C} \subset \mathcal{P}$ of tokens in places.

        Notation from here follows the convention used by \citet{naming-proofs-in-cl} of iterating over places in a formula and indexing by $i \in \mathcal{N}$.
        Given a formula in classical logic, conjunction or disjunction is encoded as:
        \begin{align*}
            P_1 \vee_2 Q_3    &\quad\mapsto\quad \{ (P_1) \times \vee_2, (Q_3) \times \vee_2 \} \\
            P_1 \wedge_2 Q_3 &\quad\mapsto\quad \{ (P_1, Q_3) \times \wedge_2 \}
        \end{align*}
        where $A, B$ are (not necessarily unique) subformulae in unique places iterated over by $\{1, 2, \ldots\}$.
        Subsequently, in an expression such as $A_1 \wedge_2 A_3$, a token on $A_1$ is unique to a token on $A_3$

        This is then a direct continuation of the concepts explored by \citet{petri-nets}.
    \end{definition*}

    \begin{example*}
        Consider the 2-d petri net representing the cross-product $a \vee \neg a \otimes \neg a \wedge a$ as follows:
        \begin{figure}[H]
    \centering
    \begin{tikzpicture}[transform shape, label distance=5mm, every node/.style={circle, fill=black!100, inner sep=0.05cm}]
        % nodes
        % top
        \node[on grid, anchor=center, label=left:\rotatebox{-90}{$\neg a$}, label=above:{$a$}](aa){};
        \node[on grid, right=1cm of aa, label=above:{$\vee$}](av){};
        \node[on grid, right=1cm of av, label=above:{$\neg a$}](ana){};
        % middle
        \node[on grid, below=1cm of aa, label=left:\rotatebox{-90}{$\wedge$}](va){};
        \node[on grid, right=1cm of va](vv){};
        \node[on grid, right=1cm of vv](vna){};
        % bottom
        \node[on grid, below=1cm of va, label=left:\rotatebox{-90}{$a$}](naa){};
        \node[on grid, right=1cm of naa](nav){};
        \node[on grid, right=1cm of nav](nana){};
        % disjunction helpers
        \coordinate[on grid, right=0.28cm of va](da);
        \coordinate[on grid, right=0.28cm of vv](dv);
        \coordinate[on grid, right=0.28cm of vna](dna);

        % edges
        % horizontal
        \draw[-arr] (aa) to [bend left] (av);
        \draw[-arr] (ana) to [bend right] (av);
        \draw[-arr] (va) to [bend left] (vv);
        \draw[-arr] (vna) to [bend right] (vv);
        \draw[-arr] (naa) to [bend left] (nav);
        \draw[-arr] (nana) to [bend right] (nav);
        % vertical
        \draw[-] (aa) to [bend left] (naa);
        \draw[-arr] (da) to (va);
        \draw[-] (av) to [bend left] (nav);
        \draw[-arr] (dv) to (vv);
        \draw[-] (ana) to [bend left] (nana);
        \draw[-arr] (dna) to (vna);
    \end{tikzpicture}
\end{figure}

    \end{example*}


    \begin{definition*}[Firing Petri Nets]
        Given a petri net $\mathcal{N}$ and configuration $\mathcal{C}$, a \textit{firing} of the net $\mathcal{N}$ is a new configuration generated by application of a transition $f \in \mathcal{F}$ on $m$ tokens $c_1 \ldots c_n \in \mathcal{C}$.
        In particular:
        \begin{equation*}
            (\mathcal{N = (P, F), C}) \mapsto (\mathcal{N}, (\mathcal{C} \cup f_{right}) \setminus f_{left})
        \end{equation*}
        for some $f = (f_{left}, f_{right}) \in \mathcal{F}$.
        
        For the uses required here, a variant of firing is used instead called \textit{spawning}.
        This generates new configurations in the same manner as firing, with one key difference:
        \begin{equation*}
            (\mathcal{N = (P, F), C}) \mapsto (\mathcal{N}, \mathcal{C} \cup f_{right})
        \end{equation*}
        for some $f = (f_{left}, f_{right}) \in \mathcal{F}$.
        That is, when a transition $f$ is performed on tokens $x_1 \ldots _n$, these tokens remain present in the configuration in addition to the new token $f(x_1 \ldots x_n)$.
        
        A petri net is said to be \textit{exhaustively fired} if it is fired until there does not exist any such $f \in \mathcal{F}$ to fire.
    \end{definition*}

    \begin{example*}
        Consider the 2d petri net representing $a \vee \neg a \otimes \neg a \wedge a$ with tokens at $\{(a, \neg a), (\neg a, a)\}$.
        \begin{figure}[H]
    \centering
    \begin{subfigure}{0.3\linewidth}
        \begin{tikzpicture}[transform shape, label distance=5mm, every path/.style={color=black!25}, every node/.style={circle, fill=black!25, inner sep=0.05cm}, every label/.append style={color=black!100}]
            % nodes
            % top
            \node[anchor=center, label=left:\rotatebox{-90}{$\neg a$}, label=above:{$a$}, style={color=black!100}](aa){};
            \node[right=1cm of aa, label=above:{$\vee$}](av){};
            \node[right=1cm of av, label=above:{$\neg a$}](ana){};
            % middle
            \node[below=1cm of aa, label=left:\rotatebox{-90}{$\wedge$}](va){};
            \node[right=1cm of va](vv){};
            \node[right=1cm of vv](vna){};
            % bottom
            \node[below=1cm of va, label=left:\rotatebox{-90}{$a$}](naa){};
            \node[right=1cm of naa](nav){};
            \node[right=1cm of nav, style={color=black!100}](nana){};
            % disjunction helpers
            \coordinate[right=0.28cm of va](da);
            \coordinate[right=0.28cm of vv](dv);
            \coordinate[right=0.28cm of vna](dna);

            % edges
            % horizontal
            \draw[-arr] (aa) to [bend left] (av);
            \draw[-arr] (ana) to [bend right] (av);
            \draw[-arr] (va) to [bend left] (vv);
            \draw[-arr] (vna) to [bend right] (vv);
            \draw[-arr] (naa) to [bend left] (nav);
            \draw[-arr] (nana) to [bend right] (nav);
            % vertical
            \draw[-] (aa) to [bend left] (naa);
            \draw[-arr] (da) to (va);
            \draw[-] (av) to [bend left] (nav);
            \draw[-arr] (dv) to (vv);
            \draw[-] (ana) to [bend left] (nana);
            \draw[-arr] (dna) to (vna);
        \end{tikzpicture}
    \end{subfigure}
    $\leadsto$
    \begin{subfigure}{0.3\linewidth}
        \begin{tikzpicture}[transform shape, label distance=5mm, every path/.style={color=black!25}, every node/.style={circle, fill=black!25, inner sep=0.05cm}, every label/.append style={color=black!100}]
            % nodes
            % top
            \node[anchor=center, label=left:\rotatebox{-90}{$\neg a$}, label=above:{$a$}, style={color=black!100}](aa){};
            \node[right=1cm of aa, label=above:{$\vee$}, style={color=black!100}](av){};
            \node[right=1cm of av, label=above:{$\neg a$}](ana){};
            % middle
            \node[below=1cm of aa, label=left:\rotatebox{-90}{$\wedge$}](va){};
            \node[right=1cm of va](vv){};
            \node[right=1cm of vv](vna){};
            % bottom
            \node[below=1cm of va, label=left:\rotatebox{-90}{$a$}](naa){};
            \node[right=1cm of naa, style={color=black!100}](nav){};
            \node[right=1cm of nav, style={color=black!100}](nana){};
            % disjunction helpers
            \coordinate[right=0.28cm of va](da);
            \coordinate[right=0.28cm of vv](dv);
            \coordinate[right=0.28cm of vna](dna);

            % edges
            % horizontal
            \draw[-arr, style={color=black!100}] (aa) to [bend left] (av);
            \draw[-arr] (ana) to [bend right] (av);
            \draw[-arr] (va) to [bend left] (vv);
            \draw[-arr] (vna) to [bend right] (vv);
            \draw[-arr] (naa) to [bend left] (nav);
            \draw[-arr, style={color=black!100}] (nana) to [bend right] (nav);
            % vertical
            \draw[-] (aa) to [bend left] (naa);
            \draw[-arr] (da) to (va);
            \draw[-] (av) to [bend left] (nav);
            \draw[-arr] (dv) to (vv);
            \draw[-] (ana) to [bend left] (nana);
            \draw[-arr] (dna) to (vna);
        \end{tikzpicture}
    \end{subfigure}
    $\leadsto$
    \begin{subfigure}{0.3\linewidth}
        \begin{tikzpicture}[transform shape, label distance=5mm, every path/.style={color=black!25}, every node/.style={circle, fill=black!25, inner sep=0.05cm}, every label/.append style={color=black!100}]
            % nodes
            % top
            \node[anchor=center, label=left:\rotatebox{-90}{$\neg a$}, label=above:{$a$}, style={color=black!100}](aa){};
            \node[right=1cm of aa, label=above:{$\vee$}, style={color=black!100}](av){};
            \node[right=1cm of av, label=above:{$\neg a$}](ana){};
            % middle
            \node[below=1cm of aa, label=left:\rotatebox{-90}{$\wedge$}](va){};
            \node[right=1cm of va, style={color=black!100}](vv){};
            \node[right=1cm of vv](vna){};
            % bottom
            \node[below=1cm of va, label=left:\rotatebox{-90}{$a$}](naa){};
            \node[right=1cm of naa, style={color=black!100}](nav){};
            \node[right=1cm of nav, style={color=black!100}](nana){};
            % disjunction helpers
            \coordinate[right=0.28cm of va](da);
            \coordinate[right=0.28cm of vv](dv);
            \coordinate[right=0.28cm of vna](dna);

            % edges
            % horizontal
            \draw[-arr, style={color=black!100}] (aa) to [bend left] (av);
            \draw[-arr] (ana) to [bend right] (av);
            \draw[-arr] (va) to [bend left] (vv);
            \draw[-arr] (vna) to [bend right] (vv);
            \draw[-arr] (naa) to [bend left] (nav);
            \draw[-arr, style={color=black!100}] (nana) to [bend right] (nav);
            % vertical
            \draw[-] (aa) to [bend left] (naa);
            \draw[-arr] (da) to (va);
            \draw[-, style={color=black!100}] (av) to [bend left] (nav);
            \draw[-arr, style={color=black!100}] (dv) to (vv);
            \draw[-] (ana) to [bend left] (nana);
            \draw[-arr] (dna) to (vna);
        \end{tikzpicture}
    \end{subfigure}
\end{figure}


        Note that, in the final step, a pair of tokens are required to perform the conjunction transition.
    \end{example*}


    \begin{definition*}[Coalescence]
        Given a formula $P$, the coalescence algorithm is as follows:
        \begin{enumerate}[nosep]
            \item Set  $n \seteq 1$
            \item Construct a $n$-dimensional petri net $\mathcal{N}$ as $\bigotimes_n P$ where each subformula represents a place and each conjunction and disjunction a flow
            \item Construct a configuration $C$ with a token at each place $p = (\ldots, a, \ldots, \neg a, \ldots)$ the intersection of a pair of tautological atoms or $(\ldots, \top, \ldots)$ an instance of \textit{top}
            \item Exhaustively fire the petri net $\mathcal{N}$ using the \textit{spawning} method
            \item If there exists a token in the configuration $\mathcal{C^*}$ at the root of the formula $P$, halt and return $n$
            \item Otherwise, increment $n \seteq n + 1$ and go to step 2
        \end{enumerate}
    \end{definition*}
    
    \begin{example*}
        Consider the petri net proof for the formula $(a \vee b) \vee \neg a$, with a solution for $n = 2$.
        The net is initialised with tokens in all places satisfying either the $ax$-rule or $\top$-rule and fired exhaustively:
        \begin{figure}[H]
    \def\x{0.8cm}
    \centering
    \begin{subfigure}{0.4\linewidth}
        \begin{tikzpicture}[transform shape, label distance=5mm, every path/.style={color=black!25}, every node/.style={circle, fill=black!25, inner sep=0.05cm}, every label/.append style={color=black!100}]
            % nodes
            % a
            \node[on grid, anchor=center, label=left:\rotatebox{-90}{$(a$}, label=above:{$(a$}](aa){};
            \node[on grid, right=\x of aa, label=above:{$\vee$}](av){};
            \node[on grid, right=\x of av, label=above:{$b)$}](ab){};
            \node[on grid, right=\x of ab, label=above:{$\vee$}](aw){};
            \node[on grid, right=\x of aw, style={color=black!100}, label=above:{$\neg a$}](ae){};
            % v
            \node[on grid, below=\x of aa, label=left:\rotatebox{-90}{$\vee$}](va){};
            \node[on grid, right=\x of va](vv){};
            \node[on grid, right=\x of vv](vb){};
            \node[on grid, right=\x of vb](vw){};
            \node[on grid, right=\x of vw](ve){};
            % b
            \node[on grid, below=\x of va, label=left:\rotatebox{-90}{$b)$}](ba){};
            \node[on grid, right=\x of ba](bv){};
            \node[on grid, right=\x of bv](bb){};
            \node[on grid, right=\x of bb](bw){};
            \node[on grid, right=\x of bw](be){};
            % v
            \node[on grid, below=\x of ba, label=left:\rotatebox{-90}{$\vee$}](wa){};
            \node[on grid, right=\x of wa](wv){};
            \node[on grid, right=\x of wv](wb){};
            \node[on grid, right=\x of wb](ww){};
            \node[on grid, right=\x of ww](we){};
            % ~a
            \node[on grid, below=\x of wa, style={color=black!100}, label=left:\rotatebox{-90}{$\neg a$}](ea){};
            \node[on grid, right=\x of ea](ev){};
            \node[on grid, right=\x of ev](eb){};
            \node[on grid, right=\x of eb](ew){};
            \node[on grid, right=\x of ew](ee){};

            % edges
            % horizontal
            % a
            \draw[-arr] (aa) to [bend left]  (av);
            \draw[-arr] (ab) to [bend right] (av);
            \draw[-arr] (av) to [bend right] (aw);
            \draw[-arr] (ae) to [bend right] (aw);
            % v
            \draw[-arr] (va) to [bend left]  (vv);
            \draw[-arr] (vb) to [bend right] (vv);
            \draw[-arr] (vv) to [bend right] (vw);
            \draw[-arr] (ve) to [bend right] (vw);
            % b
            \draw[-arr] (ba) to [bend left]  (bv);
            \draw[-arr] (bb) to [bend right] (bv);
            \draw[-arr] (bv) to [bend right] (bw);
            \draw[-arr] (be) to [bend right] (bw);
            % v
            \draw[-arr] (wa) to [bend left]  (wv);
            \draw[-arr] (wb) to [bend right] (wv);
            \draw[-arr] (wv) to [bend right] (ww);
            \draw[-arr] (we) to [bend right] (ww);
            % ~a
            \draw[-arr] (ea) to [bend left]  (ev);
            \draw[-arr] (eb) to [bend right] (ev);
            \draw[-arr] (ev) to [bend right] (ew);
            \draw[-arr] (ee) to [bend right] (ew);
            
            %vertical
            % a
            \draw[-arr] (aa) to [bend left]  (va);
            \draw[-arr] (ba) to [bend right] (va);
            \draw[-arr] (va) to [bend right] (wa);
            \draw[-arr] (ea) to [bend right] (wa);
            % v
            \draw[-arr] (av) to [bend left]  (vv);
            \draw[-arr] (bv) to [bend right] (vv);
            \draw[-arr] (vv) to [bend right] (wv);
            \draw[-arr] (ev) to [bend right] (wv);
            % b
            \draw[-arr] (ab) to [bend left]  (vb);
            \draw[-arr] (bb) to [bend right] (vb);
            \draw[-arr] (vb) to [bend right] (wb);
            \draw[-arr] (eb) to [bend right] (wb);
            % v
            \draw[-arr] (aw) to [bend left]  (vw);
            \draw[-arr] (bw) to [bend right] (vw);
            \draw[-arr] (vw) to [bend right] (ww);
            \draw[-arr] (ew) to [bend right] (ww);
            % ~a
            \draw[-arr] (ae) to [bend left]  (ve);
            \draw[-arr] (be) to [bend right] (ve);
            \draw[-arr] (ve) to [bend right] (we);
            \draw[-arr] (ee) to [bend right] (we);
        \end{tikzpicture}
    \end{subfigure}
    \begin{subfigure}{0.1\linewidth}
        $\leadsto$
    \end{subfigure}
    \begin{subfigure}{0.4\linewidth}
        \begin{tikzpicture}[transform shape, label distance=5mm, every path/.style={color=black!25}, every node/.style={circle, fill=black!25, inner sep=0.05cm}, every label/.append style={color=black!100}]
            % nodes
            % a
            \node[on grid, anchor=center, label=left:\rotatebox{-90}{$(a$}, label=above:{$(a$}](aa){};
            \node[on grid, right=\x of aa, label=above:{$\vee$}](av){};
            \node[on grid, right=\x of av, label=above:{$b)$}](ab){};
            \node[on grid, right=\x of ab, style={color=black!100}, label=above:{$\vee$}](aw){};
            \node[on grid, right=\x of aw, style={color=black!100}, label=above:{$\neg a$}](ae){};
            % v
            \node[on grid, below=\x of aa, label=left:\rotatebox{-90}{$\vee$}](va){};
            \node[on grid, right=\x of va](vv){};
            \node[on grid, right=\x of vv](vb){};
            \node[on grid, right=\x of vb, style={color=black!100}](vw){};
            \node[on grid, right=\x of vw](ve){};
            % b
            \node[on grid, below=\x of va, label=left:\rotatebox{-90}{$b)$}](ba){};
            \node[on grid, right=\x of ba](bv){};
            \node[on grid, right=\x of bv](bb){};
            \node[on grid, right=\x of bb](bw){};
            \node[on grid, right=\x of bw](be){};
            % v
            \node[on grid, below=\x of ba, style={color=black!100}, label=left:\rotatebox{-90}{$\vee$}](wa){};
            \node[on grid, right=\x of wa, style={color=black!100}](wv){};
            \node[on grid, right=\x of wv](wb){};
            \node[on grid, right=\x of wb, style={color=black!100}, style={color=black!100}](ww){};
            \node[on grid, right=\x of ww](we){};
            % ~a
            \node[on grid, below=\x of wa, style={color=black!100}, label=left:\rotatebox{-90}{$\neg a$}](ea){};
            \node[on grid, right=\x of ea](ev){};
            \node[on grid, right=\x of ev](eb){};
            \node[on grid, right=\x of eb](ew){};
            \node[on grid, right=\x of ew](ee){};

            % edges
            % horizontal
            % a
            \draw[-arr] (aa) to [bend left]  (av);
            \draw[-arr] (ab) to [bend right] (av);
            \draw[-arr] (av) to [bend right] (aw);
            \draw[-arr, style={color=black!100}] (ae) to [bend right] (aw);
            % v
            \draw[-arr] (va) to [bend left]  (vv);
            \draw[-arr] (vb) to [bend right] (vv);
            \draw[-arr] (vv) to [bend right] (vw);
            \draw[-arr] (ve) to [bend right] (vw);
            % b
            \draw[-arr] (ba) to [bend left]  (bv);
            \draw[-arr] (bb) to [bend right] (bv);
            \draw[-arr] (bv) to [bend right] (bw);
            \draw[-arr] (be) to [bend right] (bw);
            % v
            \draw[-arr, style={color=black!100}] (wa) to [bend left]  (wv);
            \draw[-arr] (wb) to [bend right] (wv);
            \draw[-arr, style={color=black!100}] (wv) to [bend right] (ww);
            \draw[-arr] (we) to [bend right] (ww);
            % ~a
            \draw[-arr] (ea) to [bend left]  (ev);
            \draw[-arr] (eb) to [bend right] (ev);
            \draw[-arr] (ev) to [bend right] (ew);
            \draw[-arr] (ee) to [bend right] (ew);
            
            %vertical
            % a
            \draw[-arr] (aa) to [bend left]  (va);
            \draw[-arr] (ba) to [bend right] (va);
            \draw[-arr] (va) to [bend right] (wa);
            \draw[-arr, style={color=black!100}] (ea) to [bend right] (wa);
            % v
            \draw[-arr] (av) to [bend left]  (vv);
            \draw[-arr] (bv) to [bend right] (vv);
            \draw[-arr] (vv) to [bend right] (wv);
            \draw[-arr] (ev) to [bend right] (wv);
            % b
            \draw[-arr] (ab) to [bend left]  (vb);
            \draw[-arr] (bb) to [bend right] (vb);
            \draw[-arr] (vb) to [bend right] (wb);
            \draw[-arr] (eb) to [bend right] (wb);
            % v
            \draw[-arr, style={color=black!100}] (aw) to [bend left]  (vw);
            \draw[-arr] (bw) to [bend right] (vw);
            \draw[-arr, style={color=black!100}] (vw) to [bend right] (ww);
            \draw[-arr] (ew) to [bend right] (ww);
            % ~a
            \draw[-arr] (ae) to [bend left]  (ve);
            \draw[-arr] (be) to [bend right] (ve);
            \draw[-arr] (ve) to [bend right] (we);
            \draw[-arr] (ee) to [bend right] (we);
        \end{tikzpicture}
    \end{subfigure}
\end{figure}


    \end{example*}

    \begin{remark*}
        Note the symmetry along the upper-left lower-right diagonal.
        In fact, this symmetry can be exploited in the general case, ignoring all tokens below (w.l.o.g.) the diagonal as they simply express the commutativity property $\vdash A, B \iff \vdash B, A$.
        If this is the case, the coalescence algorithm may halt early if the root of the formula is reached.
    \end{remark*}

    \begin{example*}
        Consider the petri net proof for the formula $(a \vee \neg a) \wedge (b \vee \neg b)$, with a solution for $n = 3$.
        In an exhaustively fired net for $n = 2$, the proof looks as follows:
        \begin{figure}[H]
    \centering
    \begin{subfigure}{0.4\linewidth}
        \begin{tikzpicture}
            \begin{scope}[transform shape, every path/.style={color=black!25}, every node/.style={circle, fill=black!25, inner sep=0.05cm}, every label/.append style={color=black!100}]
                % nodes
                % a
                \node[anchor=center    ](aa){};
                \node[right=0.5cm of aa](av){};
                \node[right=0.5cm of av, style={color=black!100}](ae){};
                \node[right=0.5cm of ae](an){};
                \node[right=0.5cm of an](ab){};
                \node[right=0.5cm of ab](aw){};
                \node[right=0.5cm of aw](ap){};
                % v
                \node[below=0.5cm of aa](va){};
                \node[right=0.5cm of va](vv){};
                \node[right=0.5cm of vv](ve){};
                \node[right=0.5cm of ve](vn){};
                \node[right=0.5cm of vn](vb){};
                \node[right=0.5cm of vb](vw){};
                \node[right=0.5cm of vw](vp){};
                % ~a
                \node[below=0.5cm of va, style={color=black!100}](ea){};
                \node[right=0.5cm of ea](ev){};
                \node[right=0.5cm of ev](ee){};
                \node[right=0.5cm of ee](en){};
                \node[right=0.5cm of en](eb){};
                \node[right=0.5cm of eb](ew){};
                \node[right=0.5cm of ew](ep){};
                % ^
                \node[below=0.5cm of ea](na){};
                \node[right=0.5cm of na](nv){};
                \node[right=0.5cm of nv](ne){};
                \node[right=0.5cm of ne](nn){};
                \node[right=0.5cm of nn](nb){};
                \node[right=0.5cm of nb](nw){};
                \node[right=0.5cm of nw](np){};
                % b
                \node[below=0.5cm of na](ba){};
                \node[right=0.5cm of ba](bv){};
                \node[right=0.5cm of bv](be){};
                \node[right=0.5cm of be](bn){};
                \node[right=0.5cm of bn](bb){};
                \node[right=0.5cm of bb](bw){};
                \node[right=0.5cm of bw, style={color=black!100}](bp){};
                % v
                \node[below=0.5cm of ba](wa){};
                \node[right=0.5cm of wa](wv){};
                \node[right=0.5cm of wv](we){};
                \node[right=0.5cm of we](wn){};
                \node[right=0.5cm of wn](wb){};
                \node[right=0.5cm of wb](ww){};
                \node[right=0.5cm of ww](wp){};
                % ~b
                \node[below=0.5cm of wa](pa){};
                \node[right=0.5cm of pa](pv){};
                \node[right=0.5cm of pv](pe){};
                \node[right=0.5cm of pe](pn){};
                \node[right=0.5cm of pn, style={color=black!100}](pb){};
                \node[right=0.5cm of pb](pw){};
                \node[right=0.5cm of pw](pp){};

                % disjunction helpers
                % horizontal
                \coordinate[below=0.2cm of an](ad);
                \coordinate[below=0.2cm of vn](vd);
                \coordinate[below=0.2cm of en](ed);
                \coordinate[below=0.2cm of nn](nd);
                \coordinate[below=0.2cm of bn](bd);
                \coordinate[below=0.2cm of wn](wd);
                \coordinate[below=0.2cm of pn](pd);
                % vertical
                \coordinate[left=0.2cm of na](da);
                \coordinate[left=0.2cm of nv](dv);
                \coordinate[left=0.2cm of ne](de);
                \coordinate[left=0.2cm of nn](dn);
                \coordinate[left=0.2cm of nb](db);
                \coordinate[left=0.2cm of nw](dw);
                \coordinate[left=0.2cm of np](dp);


                % edges
                % horizontal
                % a
                \draw[-arr] (aa) to [bend left]     (av);
                \draw[-arr] (ae) to [bend right]    (av);
                \draw[-arr] (ab) to [bend left]     (aw);
                \draw[-arr] (ap) to [bend right]    (aw);
                \draw[-]    (av) to [bend right=20] (aw);
                \draw[-arr] (ad) to                 (an);
                % v
                \draw[-arr] (va) to [bend left]     (vv);
                \draw[-arr] (ve) to [bend right]    (vv);
                \draw[-arr] (vb) to [bend left]     (vw);
                \draw[-arr] (vp) to [bend right]    (vw);
                \draw[-]    (vv) to [bend right=20] (vw);
                \draw[-arr] (vd) to                 (vn);
                % ~a
                \draw[-arr] (ea) to [bend left]     (ev);
                \draw[-arr] (ee) to [bend right]    (ev);
                \draw[-arr] (eb) to [bend left]     (ew);
                \draw[-arr] (ep) to [bend right]    (ew);
                \draw[-]    (ev) to [bend right=20] (ew);
                \draw[-arr] (ed) to                 (en);
                % ^
                \draw[-arr] (na) to [bend left]     (nv);
                \draw[-arr] (ne) to [bend right]    (nv);
                \draw[-arr] (nb) to [bend left]     (nw);
                \draw[-arr] (np) to [bend right]    (nw);
                \draw[-]    (nv) to [bend right=20] (nw);
                \draw[-arr] (nd) to                 (nn);
                % b
                \draw[-arr] (ba) to [bend left]     (bv);
                \draw[-arr] (be) to [bend right]    (bv);
                \draw[-arr] (bb) to [bend left]     (bw);
                \draw[-arr] (bp) to [bend right]    (bw);
                \draw[-]    (bv) to [bend right=20] (bw);
                \draw[-arr] (bd) to                 (bn);
                % v
                \draw[-arr] (wa) to [bend left]     (wv);
                \draw[-arr] (we) to [bend right]    (wv);
                \draw[-arr] (wb) to [bend left]     (ww);
                \draw[-arr] (wp) to [bend right]    (ww);
                \draw[-]    (wv) to [bend right=20] (ww);
                \draw[-arr] (wd) to                 (wn);
                % ~b
                \draw[-arr] (pa) to [bend left]     (pv);
                \draw[-arr] (pe) to [bend right]    (pv);
                \draw[-arr] (pb) to [bend left]     (pw);
                \draw[-arr] (pp) to [bend right]    (pw);
                \draw[-]    (pv) to [bend right=20] (pw);
                \draw[-arr] (pd) to                 (pn);
                % vertical
                % a
                \draw[-arr] (aa) to [bend left]     (va);
                \draw[-arr] (ea) to [bend right]    (va);
                \draw[-arr] (ba) to [bend left]     (wa);
                \draw[-arr] (pa) to [bend right]    (wa);
                \draw[-]    (va) to [bend right=20] (wa);
                \draw[-arr] (da) to                 (na);
                % v                                    
                \draw[-arr] (av) to [bend left]     (vv);
                \draw[-arr] (ev) to [bend right]    (vv);
                \draw[-arr] (bv) to [bend left]     (wv);
                \draw[-arr] (pv) to [bend right]    (wv);
                \draw[-]    (vv) to [bend right=20] (wv);
                \draw[-arr] (dv) to                 (nv);
                % ~a                                   
                \draw[-arr] (ae) to [bend left]     (ve);
                \draw[-arr] (ee) to [bend right]    (ve);
                \draw[-arr] (be) to [bend left]     (we);
                \draw[-arr] (pe) to [bend right]    (we);
                \draw[-]    (ve) to [bend right=20] (we);
                \draw[-arr] (de) to                 (ne);
                % ^                                    
                \draw[-arr] (an) to [bend left]     (vn);
                \draw[-arr] (en) to [bend right]    (vn);
                \draw[-arr] (bn) to [bend left]     (wn);
                \draw[-arr] (pn) to [bend right]    (wn);
                \draw[-]    (vn) to [bend right=20] (wn);
                \draw[-arr] (dn) to                 (nn);
                % b                                    
                \draw[-arr] (ab) to [bend left]     (vb);
                \draw[-arr] (eb) to [bend right]    (vb);
                \draw[-arr] (bb) to [bend left]     (wb);
                \draw[-arr] (pb) to [bend right]    (wb);
                \draw[-]    (vb) to [bend right=20] (wb);
                \draw[-arr] (db) to                 (nb);
                % v                                    
                \draw[-arr] (aw) to [bend left]     (vw);
                \draw[-arr] (ew) to [bend right]    (vw);
                \draw[-arr] (bw) to [bend left]     (ww);
                \draw[-arr] (pw) to [bend right]    (ww);
                \draw[-]    (vw) to [bend right=20] (ww);
                \draw[-arr] (dw) to                 (nw);
                % ~b                                   
                \draw[-arr] (ap) to [bend left]     (vp);
                \draw[-arr] (ep) to [bend right]    (vp);
                \draw[-arr] (bp) to [bend left]     (wp);
                \draw[-arr] (pp) to [bend right]    (wp);
                \draw[-]    (vp) to [bend right=20] (wp);
                \draw[-arr] (dp) to                 (np);
            \end{scope}

            % labels
            % horizontal
            \node[above=0.5cm of aa](la){\small $(a$};
            \node[above=0.5cm of av](lv){\small $\vee$};
            \node[above=0.5cm of ae](le){\small $\neg a)$};
            \node[above=0.5cm of an](ln){\small $\wedge$};
            \node[above=0.5cm of ab](lb){\small $(b$};
            \node[above=0.5cm of aw](lw){\small $\vee$};
            \node[above=0.5cm of ap](lp){\small $\neg b)$};
            % vertical
            \node[left=0.5cm of aa](al){\rotatebox{-90}{\small $(a$}};
            \node[left=0.5cm of va](vl){\rotatebox{-90}{\small $\vee$}};
            \node[left=0.5cm of ea](el){\rotatebox{-90}{\small $\neg a)$}};
            \node[left=0.5cm of na](nl){\rotatebox{-90}{\small $\wedge$}};
            \node[left=0.5cm of ba](bl){\rotatebox{-90}{\small $(b$}};
            \node[left=0.5cm of wa](wl){\rotatebox{-90}{\small $\vee$}};
            \node[left=0.5cm of pa](pl){\rotatebox{-90}{\small $\neg b)$}};

        \end{tikzpicture}
    \end{subfigure}
    \begin{subfigure}{0.1\linewidth}
        $\leadsto$
    \end{subfigure}
    \begin{subfigure}{0.4\linewidth}
        \begin{tikzpicture}
            \begin{scope}[transform shape, label distance=5mm, every path/.style={color=black!25}, every node/.style={circle, fill=black!25, inner sep=0.05cm}, every label/.append style={color=black!100}]
                % nodes
                % a
                \node[anchor=center    ](aa){};
                \node[right=0.5cm of aa, style={color=black!100}](av){};
                \node[right=0.5cm of av, style={color=black!100}](ae){};
                \node[right=0.5cm of ae](an){};
                \node[right=0.5cm of an](ab){};
                \node[right=0.5cm of ab](aw){};
                \node[right=0.5cm of aw](ap){};
                % v
                \node[below=0.5cm of aa, style={color=black!100}](va){};
                \node[right=0.5cm of va, style={color=black!100}](vv){};
                \node[right=0.5cm of vv, style={color=black!100}](ve){};
                \node[right=0.5cm of ve](vn){};
                \node[right=0.5cm of vn](vb){};
                \node[right=0.5cm of vb](vw){};
                \node[right=0.5cm of vw](vp){};
                % ~a
                \node[below=0.5cm of va, style={color=black!100}](ea){};
                \node[right=0.5cm of ea, style={color=black!100}](ev){};
                \node[right=0.5cm of ev](ee){};
                \node[right=0.5cm of ee](en){};
                \node[right=0.5cm of en](eb){};
                \node[right=0.5cm of eb](ew){};
                \node[right=0.5cm of ew](ep){};
                % ^
                \node[below=0.5cm of ea](na){};
                \node[right=0.5cm of na](nv){};
                \node[right=0.5cm of nv](ne){};
                \node[right=0.5cm of ne](nn){};
                \node[right=0.5cm of nn](nb){};
                \node[right=0.5cm of nb](nw){};
                \node[right=0.5cm of nw](np){};
                % b
                \node[below=0.5cm of na](ba){};
                \node[right=0.5cm of ba](bv){};
                \node[right=0.5cm of bv](be){};
                \node[right=0.5cm of be](bn){};
                \node[right=0.5cm of bn](bb){};
                \node[right=0.5cm of bb, style={color=black!100}](bw){};
                \node[right=0.5cm of bw, style={color=black!100}](bp){};
                % v
                \node[below=0.5cm of ba](wa){};
                \node[right=0.5cm of wa](wv){};
                \node[right=0.5cm of wv](we){};
                \node[right=0.5cm of we](wn){};
                \node[right=0.5cm of wn, style={color=black!100}](wb){};
                \node[right=0.5cm of wb, style={color=black!100}](ww){};
                \node[right=0.5cm of ww, style={color=black!100}](wp){};
                % ~b
                \node[below=0.5cm of wa](pa){};
                \node[right=0.5cm of pa](pv){};
                \node[right=0.5cm of pv](pe){};
                \node[right=0.5cm of pe](pn){};
                \node[right=0.5cm of pn, style={color=black!100}](pb){};
                \node[right=0.5cm of pb, style={color=black!100}](pw){};
                \node[right=0.5cm of pw](pp){};

                % disjunction helpers
                % horizontal
                \coordinate[below=0.2cm of an](ad);
                \coordinate[below=0.2cm of vn](vd);
                \coordinate[below=0.2cm of en](ed);
                \coordinate[below=0.2cm of nn](nd);
                \coordinate[below=0.2cm of bn](bd);
                \coordinate[below=0.2cm of wn](wd);
                \coordinate[below=0.2cm of pn](pd);
                % vertical
                \coordinate[left=0.2cm of na](da);
                \coordinate[left=0.2cm of nv](dv);
                \coordinate[left=0.2cm of ne](de);
                \coordinate[left=0.2cm of nn](dn);
                \coordinate[left=0.2cm of nb](db);
                \coordinate[left=0.2cm of nw](dw);
                \coordinate[left=0.2cm of np](dp);

                % edges
                % horizontal
                % a
                \draw[-arr] (aa) to [bend left]     (av);
                \draw[-arr, style={color=black!100}] (ae) to [bend right]    (av);
                \draw[-arr] (ab) to [bend left]     (aw);
                \draw[-arr] (ap) to [bend right]    (aw);
                \draw[-]    (av) to [bend right=20] (aw);
                \draw[-arr] (ad) to                 (an);
                % v
                \draw[-arr, style={color=black!100}] (va) to [bend left]     (vv);
                \draw[-arr, style={color=black!100}] (ve) to [bend right]    (vv);
                \draw[-arr] (vb) to [bend left]     (vw);
                \draw[-arr] (vp) to [bend right]    (vw);
                \draw[-]    (vv) to [bend right=20] (vw);
                \draw[-arr] (vd) to                 (vn);
                % ~a
                \draw[-arr, style={color=black!100}] (ea) to [bend left]     (ev);
                \draw[-arr] (ee) to [bend right]    (ev);
                \draw[-arr] (eb) to [bend left]     (ew);
                \draw[-arr] (ep) to [bend right]    (ew);
                \draw[-]    (ev) to [bend right=20] (ew);
                \draw[-arr] (ed) to                 (en);
                % ^
                \draw[-arr] (na) to [bend left]     (nv);
                \draw[-arr] (ne) to [bend right]    (nv);
                \draw[-arr] (nb) to [bend left]     (nw);
                \draw[-arr] (np) to [bend right]    (nw);
                \draw[-]    (nv) to [bend right=20] (nw);
                \draw[-arr] (nd) to                 (nn);
                % b
                \draw[-arr] (ba) to [bend left]     (bv);
                \draw[-arr] (be) to [bend right]    (bv);
                \draw[-arr] (bb) to [bend left]     (bw);
                \draw[-arr, style={color=black!100}] (bp) to [bend right]    (bw);
                \draw[-]    (bv) to [bend right=20] (bw);
                \draw[-arr] (bd) to                 (bn);
                % v
                \draw[-arr] (wa) to [bend left]     (wv);
                \draw[-arr] (we) to [bend right]    (wv);
                \draw[-arr, style={color=black!100}] (wb) to [bend left]     (ww);
                \draw[-arr, style={color=black!100}] (wp) to [bend right]    (ww);
                \draw[-]    (wv) to [bend right=20] (ww);
                \draw[-arr] (wd) to                 (wn);
                % ~b
                \draw[-arr] (pa) to [bend left]     (pv);
                \draw[-arr] (pe) to [bend right]    (pv);
                \draw[-arr, style={color=black!100}] (pb) to [bend left]     (pw);
                \draw[-arr] (pp) to [bend right]    (pw);
                \draw[-]    (pv) to [bend right=20] (pw);
                \draw[-arr] (pd) to                 (pn);
                % vertical
                % a
                \draw[-arr] (aa) to [bend left]     (va);
                \draw[-arr, style={color=black!100}] (ea) to [bend right]    (va);
                \draw[-arr] (ba) to [bend left]     (wa);
                \draw[-arr] (pa) to [bend right]    (wa);
                \draw[-]    (va) to [bend right=20] (wa);
                \draw[-arr] (da) to                 (na);
                % v                                    
                \draw[-arr, style={color=black!100}] (av) to [bend left]     (vv);
                \draw[-arr, style={color=black!100}] (ev) to [bend right]    (vv);
                \draw[-arr] (bv) to [bend left]     (wv);
                \draw[-arr] (pv) to [bend right]    (wv);
                \draw[-]    (vv) to [bend right=20] (wv);
                \draw[-arr] (dv) to                 (nv);
                % ~a                                   
                \draw[-arr, style={color=black!100}] (ae) to [bend left]     (ve);
                \draw[-arr] (ee) to [bend right]    (ve);
                \draw[-arr] (be) to [bend left]     (we);
                \draw[-arr] (pe) to [bend right]    (we);
                \draw[-]    (ve) to [bend right=20] (we);
                \draw[-arr] (de) to                 (ne);
                % ^                                    
                \draw[-arr] (an) to [bend left]     (vn);
                \draw[-arr] (en) to [bend right]    (vn);
                \draw[-arr] (bn) to [bend left]     (wn);
                \draw[-arr] (pn) to [bend right]    (wn);
                \draw[-]    (vn) to [bend right=20] (wn);
                \draw[-arr] (dn) to                 (nn);
                % b                                    
                \draw[-arr] (ab) to [bend left]     (vb);
                \draw[-arr] (eb) to [bend right]    (vb);
                \draw[-arr] (bb) to [bend left]     (wb);
                \draw[-arr, style={color=black!100}] (pb) to [bend right]    (wb);
                \draw[-]    (vb) to [bend right=20] (wb);
                \draw[-arr] (db) to                 (nb);
                % v                                    
                \draw[-arr] (aw) to [bend left]     (vw);
                \draw[-arr] (ew) to [bend right]    (vw);
                \draw[-arr, style={color=black!100}] (bw) to [bend left]     (ww);
                \draw[-arr, style={color=black!100}] (pw) to [bend right]    (ww);
                \draw[-]    (vw) to [bend right=20] (ww);
                \draw[-arr] (dw) to                 (nw);
                % ~b                                   
                \draw[-arr] (ap) to [bend left]     (vp);
                \draw[-arr] (ep) to [bend right]    (vp);
                \draw[-arr, style={color=black!100}] (bp) to [bend left]     (wp);
                \draw[-arr] (pp) to [bend right]    (wp);
                \draw[-]    (vp) to [bend right=20] (wp);
                \draw[-arr] (dp) to                 (np);
            \end{scope}

            % labels
            % horizontal
            \node[above=0.5cm of aa](la){\small $(a$};
            \node[above=0.5cm of av](lv){\small $\vee$};
            \node[above=0.5cm of ae](le){\small $\neg a)$};
            \node[above=0.5cm of an](ln){\small $\wedge$};
            \node[above=0.5cm of ab](lb){\small $(b$};
            \node[above=0.5cm of aw](lw){\small $\vee$};
            \node[above=0.5cm of ap](lp){\small $\neg b)$};
            % vertical
            \node[left=0.5cm of aa](al){\rotatebox{-90}{\small $(a$}};
            \node[left=0.5cm of va](vl){\rotatebox{-90}{\small $\vee$}};
            \node[left=0.5cm of ea](el){\rotatebox{-90}{\small $\neg a)$}};
            \node[left=0.5cm of na](nl){\rotatebox{-90}{\small $\wedge$}};
            \node[left=0.5cm of ba](bl){\rotatebox{-90}{\small $(b$}};
            \node[left=0.5cm of wa](wl){\rotatebox{-90}{\small $\vee$}};
            \node[left=0.5cm of pa](pl){\rotatebox{-90}{\small $\neg b)$}};

        \end{tikzpicture}
    \end{subfigure}
\end{figure}


        The coalescence algorithm fails to find a solution in 2 dimensions due to the lack of tokens in upper-right or lower-left quadrants.
        If there were tokens in such a place, the tree would be able to continue to grow towards the formula root at the centre of the net.
    \end{example*}

    \begin{remark*}
        Implementing firable petri nets is straightforward using an $n$-dimensional boolean array of places visited and a collection of $n$-tuples representing tokens.
        Within the context of coalescence proof search, this can be optimised in many ways, in this case a upper-triangular n-dimensional boolean matrix of places visited and a red-black tree of n-dimensional boolean array tokens can be used.
        This exploits most of the available symmetry in the structure of the nets as, in implementation, memory quickly becomes the limiting factor.
        Furthermore, this structure allows for $\mathcal{O}(\log{}n)$ insertion and removal, $\mathcal{O}(n)$ iteration and minimal memory.
    \end{remark*}


    \begin{proposition*}[Coalescence Proof Search]
        The coalescence algorithm on $P$ is exactly a proof search on $P$.
    \end{proposition*}

    \begin{proof}
        In particular, consider the additively stratified $\vdash P$ with $n - 1$ contractions (should such a proof exist).
        The first configuration of tokens is precisely a proof all possible combinations of the \textit{axiom} rule with $n - 3$ other terms through the \textit{weakening}.
        Each flow transition followed when fired is an application of either the $\vee$ or $\wedge$ rule.
        Finally, a token at the root of the formula is $\vdash P \ldots P$, with the \textit{contraction} rule applied implicitly.

        If there exists an additively stratified proof of $P$, the coalescence algorithm will find it.
        Since for every proof there exists an additively stratified proof, coalescence is precisely proof search.
    \end{proof}



    \section{Dimensionality}
    
    \begin{definition*}[Dimensionality]
        Given a formula $P$, the coalescence proof search produces a proof in an $n$-dimensional petri net.
        Equivalently, an additively stratified sequent proof $\vdash P$ has $n$ terms at the bottom before any contractions are applied.
        The dimensionality of $P$ is then defined $\dim(P) \defeq n$.
    \end{definition*}

    \begin{example}
        % TODO
    \end{example}


    \begin{definition}[Classes of Formulae]\label{sec-ctr:formulae-classes}
        Let $D^i$ be the subclasses of formulae defined as:
        \begin{align*}
            D^1 \quad &\defeq \quad \top \,|\, D^1 \wedge D^1 \,|\, D^* \vee D^1 \\
            D^2 \quad &\defeq \quad D^{i \geq 2} \vee D^2 \,|\, P \vee \neg P \,|\, \ldots
        \end{align*}
        where $D^* \defeq \bigcup_{i \in \mathbb{N}} D_i$, such that $D^n$ is the class of all formulae provable in n dimensions.

        \textit{This definition will be revised later.}
    \end{definition}


    \begin{remark*}[Satisfiability vs Provability]
        For any formula $P$, there exist four distinct classes: \textit{true, false, satisfiably true, satisfiably false}, where satisfiable differs by finding a particular assignment of values to each variable.
        Coalescence searches for a proof of $P$ in \textit{true}, whereas the SAT problem addresses a proof of $P$ in \textit{true}.
    \end{remark*}
    
    
    \begin{proposition*}[Bounds on Dimensionality]
        Given a formula $P$, the following provides a bound its dimensionality:
        \begin{equation*}
            P \in D^n \implies n+1 \leq vars(P)
        \end{equation*}
        Justification is left unaddressed here as the current construction does not emit a `nice' proof.
    \end{proposition*}

    \begin{remark*}
        These methods do not lend themselves to `nice' properties.
        It is possible that in the general case, \textit{finding an exact value for the dimensionality of a given formula $P$ may be equivalent to finding a proof of $P$}, but it remains to be proven.
        Instead, a natural optimisation is applied to the algorithm as follows, giving rise to a different set of classes of dimensionality and more pleasing properties.
    \end{remark*} 

    \begin{definition*}[$\top$-substitution]
         Given a sequent expressing a proof of $Q$, i.e. $\vdash Q, Q, \ldots Q$, the formula $P$ may be equivalently expressed $P[Q \seteq \top]$.
    \end{definition*}

    \begin{example}
        Consider the formula $P \defeq (a \vee \neg a) \wedge (b \vee \neg b)$.
        The sequent proof $\vdash P$ is large and has dimensionality 3 as follows:
        \begin{center}
            \begin{tabular}{@{}l@{}}
                \begin{varwidth}{\linewidth}
                    \begin{scprooftree}{0.47}
                        \AxiomC{$ \vdash a , a , \neg a $}
                        \UnaryInfC{$ \vdash a , ( a \vee \neg a ) , \neg a $}
                        \AxiomC{$ \vdash a , \neg a , b $}
                        \UnaryInfC{$ \vdash a , \neg a , ( b \vee \neg b ) $}
                        \BinaryInfC{$ \vdash a , \neg a , ( ( a \vee \neg a ) \wedge ( b \vee \neg b ) ) $}
                        \UnaryInfC{$ \vdash a , ( a \vee \neg a ) , ( ( a \vee \neg a ) \wedge ( b \vee \neg b ) ) $}
                        \AxiomC{$ \vdash a , \neg a , b $}
                        \UnaryInfC{$ \vdash a , \neg a , ( b \vee \neg b ) $}
                        \UnaryInfC{$ \vdash a , ( a \vee \neg a ) , ( b \vee \neg b ) $}
                        \AxiomC{$ \vdash a , b , \neg b $}
                        \UnaryInfC{$ \vdash a , b , ( b \vee \neg b ) $}
                        \UnaryInfC{$ \vdash a , ( b \vee \neg b ) , ( b \vee \neg b ) $}
                        \BinaryInfC{$ \vdash a , ( ( a \vee \neg a ) \wedge ( b \vee \neg b ) ) , ( b \vee \neg b ) $}
                        \BinaryInfC{$ \vdash a , ( ( a \vee \neg a ) \wedge ( b \vee \neg b ) ) , ( ( a \vee \neg a ) \wedge ( b \vee \neg b ) ) $}
                        \UnaryInfC{$ \vdash ( a \vee \neg a ) , ( ( a \vee \neg a ) \wedge ( b \vee \neg b ) ) , ( ( a \vee \neg a ) \wedge ( b \vee \neg b ) ) $}
                        \AxiomC{$ \vdash a , \neg a , b $}
                        \UnaryInfC{$ \vdash a , \neg a , ( b \vee \neg b ) $}
                        \UnaryInfC{$ \vdash a , ( a \vee \neg a ) , ( b \vee \neg b ) $}
                        \AxiomC{$ \vdash a , b , \neg b $}
                        \UnaryInfC{$ \vdash a , b , ( b \vee \neg b ) $}
                        \UnaryInfC{$ \vdash a , ( b \vee \neg b ) , ( b \vee \neg b ) $}
                        \BinaryInfC{$ \vdash a , ( ( a \vee \neg a ) \wedge ( b \vee \neg b ) ) , ( b \vee \neg b ) $}
                        \UnaryInfC{$ \vdash ( a \vee \neg a ) , ( ( a \vee \neg a ) \wedge ( b \vee \neg b ) ) , ( b \vee \neg b ) $}
                        \AxiomC{$ \vdash ( ( a \vee \neg a ) \wedge ( b \vee \neg b ) ) , b , \neg b $}
                        \UnaryInfC{$ \vdash ( ( a \vee \neg a ) \wedge ( b \vee \neg b ) ) , b , ( b \vee \neg b ) $}
                        \UnaryInfC{$ \vdash ( ( a \vee \neg a ) \wedge ( b \vee \neg b ) ) , ( b \vee \neg b ) , ( b \vee \neg b ) $}
                        \BinaryInfC{$ \vdash ( ( a \vee \neg a ) \wedge ( b \vee \neg b ) ) , ( ( a \vee \neg a ) \wedge ( b \vee \neg b ) ) , ( b \vee \neg b ) $}
                        \BinaryInfC{$ \vdash ( ( a \vee \neg a ) \wedge ( b \vee \neg b ) ) , ( ( a \vee \neg a ) \wedge ( b \vee \neg b ) ) , ( ( a \vee \neg a ) \wedge ( b \vee \neg b ) ) $}
                    \end{scprooftree}
                \end{varwidth}
            \end{tabular}
        \end{center}
        
        Using substitutions of $\top$ for provable subformulae, the proof $\vdash P$ becomes much more manageable:

        \begin{center}
            \begin{tabular}{@{}l@{}}
                $ A := ( a \vee \neg a ) $
                \begin{varwidth}{\linewidth}
                    \begin{prooftree}
                        \AxiomC{$ \vdash a , \neg a $}
                        \UnaryInfC{$ \vdash a , ( a \vee \neg a ) $}
                        \UnaryInfC{$ \vdash ( a \vee \neg a ) , ( a \vee \neg a ) $}
                    \end{prooftree}
                \end{varwidth}\\ \\

                $ B := ( b \vee \neg b ) $
                \begin{varwidth}{\linewidth}
                    \begin{prooftree}
                        \AxiomC{$ \vdash b , \neg b $}
                        \UnaryInfC{$ \vdash b , ( b \vee \neg b ) $}
                        \UnaryInfC{$ \vdash ( b \vee \neg b ) , ( b \vee \neg b ) $}
                    \end{prooftree}
                \end{varwidth}\\ \\

                \begin{varwidth}{\linewidth}
                    \begin{prooftree}
                        \AxiomC{$ \vdash A , A $}
                        \AxiomC{$ \vdash A , B $}
                        \BinaryInfC{$ \vdash A , ( A \wedge B ) $}
                        \AxiomC{$ \vdash A , B $}
                        \AxiomC{$ \vdash B , B $}
                        \BinaryInfC{$ \vdash B , ( A \wedge B ) $}
                        \BinaryInfC{$ \vdash ( A \wedge B ) , ( A \wedge B ) $}
                    \end{prooftree}
                \end{varwidth}\\
            \end{tabular}
        \end{center}

        While the final step need only be performed with dimensionality 1, the prior steps require dimensionality 2, giving the full formula a dimensionality of 2.
    \end{example}


    \begin{remark*}
        It is now important to reexamine the coalescence algorithm.

        Storing a token as a \texttt{Tuple} gives an inefficient algorithm due to the implicit ordering.
        By commutativity, given a token $(a_1 \ldots a_n)$, there exist up to $n!$ permutations of $a_i$'s and thus $n!$ equivalent tokens.
        
        Instead, order is abstracted over and a token is stored as a \texttt{Multiset}.
        This gives the storage optimisation exploiting the symmetry of petri nets that expresses commutativity of terms in a sequent.

        This may be taken a step further.
        Consider a token stored as a \texttt{Set} which expresses idempotency of terms in a sequent $\vdash P, P \equiv \, \vdash P$.
    \end{remark*}

    \begin{examples}
        Consider again the formula $(a \vee \neg a) \wedge (b \vee \neg b)$, with a proofs as follow, but instead tracing the values of tokens.
        
        Where tokens are member of the category \texttt{Tuple}, so as to demonstrate the exponential blow-up, both the `minimal' (black) and `saturated' (grey) proofs are as follows:
        \begin{figure}[H]
    \centering
    \begin{tikzpicture}[every node/.style={scale=0.7}]
        \def\x{1cm}
        \def\y{1cm}
        
        \node[anchor=center](root){};
        \node[on grid, left=       4*\x of root](lroot){};
        \node[on grid, left=       2*\x of lroot  ](13){$(1, 3)$};
        \node[on grid, right=      2*\x of lroot  ](31){$(3, 1)$};
        \node[on grid, below left= \y and \x of 13](23){$(2, 3)$};
        \node[on grid, below right=\y and \x of 13](12){$(1, 2)$};
        \node[on grid, below left= \y and \x of 31](21){$(2, 1)$};
        \node[on grid, below right=\y and \x/2 of 31](32){$(3, 2)$};
        \node[on grid, below right=\y and \x of 12](22){$(2, 2)$};
                        
        \node[on grid, right=      2*\x of root](rroot){};
        \node[on grid, left=       2*\x of rroot  ](57){$(5, 7)$};
        \node[on grid, right=      2*\x of rroot  ](75){$(7, 5)$};
        \node[on grid, below left= \y and \x/2 of 57](67){$(6, 7)$};
        \node[on grid, below right=\y and \x of 57](56){$(5, 6)$};
        \node[on grid, below left= \y and \x of 75](65){$(6, 5)$};
        \node[on grid, below right=\y and \x of 75](76){$(7, 6)$};
        \node[on grid, below right=\y and \x of 56](66){$(6, 6)$};
        
        \node[on grid, below left= \y and 2*\x of 22](224){$(2, 2, 4)$};
        \node[on grid, below left= \y and 0*\x of 22](242){$(2, 4, 2)$};
        \node[on grid, below right=\y and 2*\x of 22](422){$(4, 2, 2)$};
        \node[on grid, below left= \y and 2*\x of 66](466){$(4, 6, 6)$};
        \node[on grid, below left= \y and 0*\x of 66](646){$(6, 4, 6)$};
        \node[on grid, below right=\y and 2*\x of 66](664){$(6, 6, 4)$};
       
        \node[on grid, below left= 2*\y and 3*\x of 22](226){$(2, 2, 6)$};
        \node[on grid, below left= 2*\y and \x of 22  ](262){$(2, 6, 2)$};
        \node[on grid, below right=2*\y and \x of 22  ](622){$(6, 2, 2)$};
        \node[on grid, below left= 2*\y and 3*\x of 66](266){$(2, 6, 6)$};
        \node[on grid, below left= 2*\y and \x of 66  ](626){$(6, 2, 6)$};
        \node[on grid, below right=2*\y and \x of 66  ](662){$(6, 6, 2)$};

        \coordinate[on grid, below=\y of 226   ](246c){};
        \coordinate[on grid, right=2*\x of 246c](264c){};
        \coordinate[on grid, right=2*\x of 264c](426c){};
        \coordinate[on grid, right=2*\x of 426c](462c){};
        \coordinate[on grid, right=2*\x of 462c](624c){};
        \coordinate[on grid, right=2*\x of 624c](642c){};

        \node[on grid, below=\y/2 of 246c](246){$(2, 4, 6)$};
        \node[on grid, below=\y/2 of 264c](264){$(2, 6, 4)$};
        \node[on grid, below=\y/2 of 426c](426){$(4, 2, 6)$};
        \node[on grid, below=\y/2 of 462c](462){$(4, 6, 2)$};
        \node[on grid, below=\y/2 of 624c](624){$(6, 2, 4)$};
        \node[on grid, below=\y/2 of 642c](642){$(6, 4, 2)$};

        \coordinate[on grid, below=2.5*\y of 224](224c){};
        \coordinate[on grid, below=2.5*\y of 242](242c){};
        \coordinate[on grid, below=2.5*\y of 422](422c){};
        \coordinate[on grid, below=2.5*\y of 466](466c){};
        \coordinate[on grid, below=2.5*\y of 646](646c){};
        \coordinate[on grid, below=2.5*\y of 664](664c){};
    
        \coordinate[on grid, below left= 1.5*\y and \x/2 of 224c](244lc){};
        \coordinate[on grid, below right=1.5*\y and \x/2 of 224c](244rc){};
        \coordinate[on grid, below left= 1.5*\y and \x/2 of 242c](424lc){};
        \coordinate[on grid, below right=1.5*\y and \x/2 of 242c](424rc){};
        \coordinate[on grid, below left= 1.5*\y and \x/2 of 422c](442lc){};
        \coordinate[on grid, below right=1.5*\y and \x/2 of 422c](442rc){};
        
        \coordinate[on grid, below left= 1.5*\y and \x/2 of 664c](644lc){};
        \coordinate[on grid, below right=1.5*\y and \x/2 of 664c](644rc){};
        \coordinate[on grid, below left= 1.5*\y and \x/2 of 646c](464lc){};
        \coordinate[on grid, below right=1.5*\y and \x/2 of 646c](464rc){};
        \coordinate[on grid, below left= 1.5*\y and \x/2 of 466c](446lc){};
        \coordinate[on grid, below right=1.5*\y and \x/2 of 466c](446rc){};

        \node[on grid, below=2*\y of 224c](244){$(2, 4, 4)$};
        \node[on grid, below=2*\y of 242c](424){$(4, 2, 4)$};
        \node[on grid, below=2*\y of 422c](442){$(4, 4, 2)$};
        \node[on grid, below=2*\y of 466c](446){$(4, 4, 6)$};
        \node[on grid, below=2*\y of 646c](464){$(4, 6, 4)$};
        \node[on grid, below=2*\y of 664c](644){$(6, 4, 4)$};

        \draw[-arr] (13) to (23);
        \draw[-arr] (13) to (12);
        \draw[-arr] (31) to (21);
        \draw[-arr] (31) to (32);
        \draw[-arr] (23) to (22);
        \draw[-arr] (12) to (22);
        \draw[-arr] (21) to (22);
        \draw[-arr] (32) to (22);
        
        \draw[-arr] (57) to (67);
        \draw[-arr] (57) to (56);
        \draw[-arr] (75) to (65);
        \draw[-arr] (75) to (76);
        \draw[-arr] (67) to (66);
        \draw[-arr] (56) to (66);
        \draw[-arr] (65) to (66);
        \draw[-arr] (76) to (66);        
        
        \drawsquig (22) to (224);
        \drawsquig (22) to (242);
        \drawsquig (22) to (422);
        \drawsquig (66) to (466);
        \drawsquig (66) to (646);
        \drawsquig (66) to (664);

        \drawsquig (22) to (226);
        \drawsquig (22) to (262);
        \drawsquig (22) to (622);
        \drawsquig (66) to (266);
        \drawsquig (66) to (626);
        \drawsquig (66) to (662);

        \draw[-] (226) to (246c) to (266);
        \draw[-] (226) to (426c) to (626);
        \draw[-] (262) to (264c) to (266);
        \draw[-] (262) to (462c) to (662);
        \draw[-] (622) to (624c) to (626);
        \draw[-] (622) to (642c) to (662);

        \draw[-arr] (246c) to (246);
        \draw[-arr] (264c) to (264);
        \draw[-arr] (426c) to (426);
        \draw[-arr] (462c) to (462);
        \draw[-arr] (624c) to (624);
        \draw[-arr] (642c) to (642);

        \draw[-] (224) to (224c);
        \draw[-] (242) to (242c);
        \draw[-] (422) to (422c);
        \draw[-] (466) to (466c);
        \draw[-] (646) to (646c);
        \draw[-] (664) to (664c);

        \draw[-] (246) to (244lc) to (242c);
        \draw[-] (246) to (446lc) to (646c);
        \draw[-] (264) to (244rc) to (224c);
        \draw[-] (264) to (464lc) to (664c);
        \draw[-] (426) to (424lc) to (422c);
        \draw[-] (426) to (446rc) to (466c);
        \draw[-] (462) to (464rc) to (466c);
        \draw[-] (462) to (442lc) to (422c);
        \draw[-] (624) to (644lc) to (664c);
        \draw[-] (624) to (424rc) to (224c);
        \draw[-] (642) to (644rc) to (646c);
        \draw[-] (642) to (442rc) to (242c);

        \draw[-arr] (244lc) to (244);
        \draw[-arr] (424lc) to (424);
        \draw[-arr] (442lc) to (442);
        \draw[-arr] (644lc) to (644);
        \draw[-arr] (464lc) to (464);
        \draw[-arr] (446lc) to (446);
        \draw[-arr] (244rc) to (244);
        \draw[-arr] (424rc) to (424);
        \draw[-arr] (442rc) to (442);
        \draw[-arr] (644rc) to (644);
        \draw[-arr] (464rc) to (464);
        \draw[-arr] (446rc) to (446);

    \end{tikzpicture}
\end{figure}
 where $\rightsquigarrow$ is an application of weakening.
        This is notable as this will be the point coalescence will reach for $n \seteq 2$ and a suitably intelligent method for extending dimensions will maintain the proof search so far.
        Both `minimal' and `saturated' proofs are included as each represent the deterministic versus non-deterministic searches --- proof search versus validation.

        Where tokens are now members of \texttt{Multiset}, the proof of the full formula follows:
        \begin{figure}[H]
    \def\x{1cm}
    \def\y{1cm}
    \centering
    \begin{tikzpicture}[every node/.style={scale=0.7}]
        \node[anchor=center](root){};
        \node[on grid, left=       2*\x of root   ](13){$<1, 3>$};
        \node[on grid, right=      2*\x of root   ](57){$<5, 7>$};
        \node[on grid, below left= \y and \x of 13](23){$<2, 3>$};
        \node[on grid, below right=\y and \x of 13](12){$<1, 2>$};
        \node[on grid, below left= \y and \x of 57](67){$<6, 7>$};
        \node[on grid, below right=\y and \x of 57](56){$<5, 6>$};
        \node[on grid, below right=\y and \x of 23](22){$<2, 2>$};
        \node[on grid, below right=\y and \x of 67](66){$<6, 6>$};

        \node[on grid, below left= \y and \x of 22](224){$<2, 2, 4>$};
        \node[on grid, below right=\y and \x of 22](226){$<2, 2, 6>$};
        \node[on grid, below left= \y and \x of 66](266){$<2, 6, 6>$};
        \node[on grid, below right=\y and \x of 66](466){$<4, 6, 6>$};

        \coordinate[on grid, right=\x of 226](246c){};
        \node[on grid, below right=\y and \x of 226](246){$<2, 4, 6>$};
        \coordinate[on grid, left =2*\x of 246](244c){};
        \node[on grid, below left =\y and \x of 246](244){$<2, 4, 4>$};
        \coordinate[on grid, right=2*\x of 246](446c){};
        \node[on grid, below right=\y and \x of 246](446){$<4, 4, 6>$};
        \coordinate[on grid, below=\y of 246](444c){};
        \node[on grid, below right=\y and \x of 244](444){$<4, 4, 4>$};

        \draw[-arr] (13) to (23);
        \draw[-arr] (13) to (12);
        \draw[-arr] (57) to (67);
        \draw[-arr] (57) to (56);
        \draw[-arr] (23) to (22);
        \draw[-arr] (12) to (22);
        \draw[-arr] (67) to (66);
        \draw[-arr] (56) to (66);

        \drawsquig (22) to (224);
        \drawsquig (22) to (226);
        \drawsquig (66) to (266);
        \drawsquig (66) to (466);

        \draw[-] (226) to (246c) to (266);
        \draw[-] (224) to (244c) to (246);
        \draw[-] (466) to (446c) to (246);
        \draw[-] (244) to (444c) to (446);
        
        \draw[-arr] (246c) to (246);
        \draw[-arr] (244c) to (244);
        \draw[-arr] (446c) to (446);
        \draw[-arr] (444c) to (444);
    \end{tikzpicture}
\end{figure}

        The above is still tedious and unnatural.
        Proofs of each of $a \vee \neg a$ and $b \vee \neg b$ are constructed in two steps, while the remaining steps are to prove the conjunction of these (proven) subformulae.
        Instead, consider the tokens as members of \texttt{Set} as follows:

        \begin{figure}[H]
    \centering
    \begin{tikzpicture}
        \def\x{1cm}
        \def\y{1cm}
        
        \node[anchor=center](root){};
        \node[on grid, left=       2*\x of root   ](13){$\{1, 3\}$};
        \node[on grid, right=      2*\x of root   ](57){$\{5, 7\}$};
        \node[on grid, below left= \y and \x of 13](23){$\{2, 3\}$};
        \node[on grid, below right=\y and \x of 13](12){$\{1, 2\}$};
        \node[on grid, below left= \y and \x of 57](67){$\{6, 7\}$};
        \node[on grid, below right=\y and \x of 57](56){$\{5, 6\}$};
        \node[on grid, below right=\y and \x of 23](2){$\{2\}$};
        \node[on grid, below right=\y and \x of 67](6){$\{6\}$};
        \coordinate[on grid, right=2*\x of 2](4c){};
        \node[on grid, below right=\y and 2*\x of 2](4){$\{4\}$};

        \draw[-arr] (13) to (23);
        \draw[-arr] (13) to (12);
        \draw[-arr] (57) to (67);
        \draw[-arr] (57) to (56);
        \draw[-arr] (23) to (2);
        \draw[-arr] (12) to (2);
        \draw[-arr] (67) to (6);
        \draw[-arr] (56) to (6);
        \draw[-] (2) to (4c) to (6);
        \draw[-arr] (4c) to (4);
    \end{tikzpicture}
\end{figure}

        This is finally deemed a `satisfying' proof --- dimensionality is extracted as the largest set required in the proof, a nicer algebra of dimensionality may be constructed and there is some essence of justification for the correctness of this $\top$-substitution --- all through changing the datatype of tokens.
    \end{examples}

    \begin{definition*}[Classes of Formulae]
        With this new concept of coalescence, dimensionality and proof search, the definitions in~\ref{sec-ctr:formulae-classes} are revisited and revised.
        Let $\mathcal{D}^i$ be defined as the subclasses of formulae:
        \begin{align*}
            \mathcal{D}^1 \quad &\defeq \quad \top \,|\, \mathcal{D}^1 \wedge \mathcal{D}^1 \,|\, \mathcal{D}^* \vee \mathcal{D}^1 \\
            \mathcal{D}^2 \quad &\defeq \quad P \vee \neg P \,|\, \mathcal{D}^2 \vee \mathcal{D}^{i \geq 2} \,|\, \mathcal{D}^2 \wedge \mathcal{D}^{i \leq 2} \\
            \mathcal{D}^3 \quad &\defeq \quad (P \wedge Q) \vee (P \wedge \neg Q) \vee (\neg P \wedge Q) \vee (\neg P \wedge \neg Q) \,|\, \mathcal{D}^3 \vee \mathcal{D}^{i \geq 3} \,|\, \mathcal{D}^3 \wedge \mathcal{D}^{i \leq 3}
        \end{align*}
        where $\mathcal{D}^* \defeq \bigcup_{i \in \mathbb{N}} \mathcal{D}^i$, $P \in \mathcal{D}^*$, such that $\mathcal{D}^n$ is the class of all formulae provable in exactly n dimensions.
    \end{definition*}
    
    \begin{corollary*}
        For $P \in \mathcal{D}^i, Q \in \mathcal{D}^j$, the following statements hold:
        \begin{equation*}
            P \vee Q \in \mathcal{D}^{\min{i, j}}
            \qquad
            P \wedge Q \in \mathcal{D}^{\max{i, j}}
        \end{equation*}
        In fact, for $P \in \mathcal{D}^n$ in $v$ variables, $n \leq v + 1$.
    \end{corollary*}


    \begin{proposition}
        Let $\textit{DNF}^n$ denote the set of all formulae in $n$ variables in disjunctive normal form.
        That is, $P \in \textit{DNF}^n \implies P = \bigvee_{j = 1 .. 2^n} (\bigwedge_{i = 1 .. n} a_i)$ the disjunction of $2^n$ conjuctions, each representing an `assignment' of boolean values to $n$ variables.
        Then the following statement holds:
        \begin{equation*}
            P \in \textit{DNF}^n \implies P \in \mathcal{D}^{n+1}
        \end{equation*}
    \end{proposition}

    \begin{proof}
        By induction on n.

        Suppose:
        \begin{equation*}
            \forall \,\, i < n \,:\, P \in \textit{DNF}^i \implies P \in \mathcal{D}^{i+1}
        \end{equation*}
        We seek to show:
        \begin{equation*}
            \exists \, Q \in \textit{DNF}^n \text{ such that } Q \in \mathcal{D}^{n+1}
        \end{equation*}
        
        Let $a_n$ be fresh in $P$ and construct $Q$ as follows:
        \begin{equation*}
            Q \seteq (P \wedge a_n) \vee (P \wedge \neg a_n)
        \end{equation*}
        where $Q \in \textit{DNF}^n$ by distributivity of conjunction.
    \end{proof}

    \begin{proposition*}
        
        \begin{equation*}
            P \in \mathcal{D}^{n+1} \iff \exists \, Q \in P \text{ such that } Q \in DNF^{n}
        \end{equation*}
    \end{proposition*}

    
    \begin{remarks}
        \citet{tableaux-for-logic-of-proofs}
        \citet{connection-based-proof-method}
    \end{remarks}


    \chapter{References}
    \bibliography{dissertation}


\end{document}
