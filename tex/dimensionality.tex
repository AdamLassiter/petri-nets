\chapter{Dimensionality}

    \begin{definition*}[Dimensionality]
        Given a formula $P$, the coalescence proof search produces a proof in an $n$-dimensional petri net.
        Equivalently, an additively stratified sequent proof $\vdash P$ has $n$ terms at the bottom before any contractions are applied.
        The dimensionality of $P$ is then defined $\dim(P) \defeq n$.
    \end{definition*}

    \begin{example*}
        Borrowing the proof from Example~\ref{example:add-strat-proof}, it has dimensionality 2 since the tree concludes with $\vdash P, P$ before any contractions.
        \begin{prooftree}
            \AxiomC{}
            \RightLabel{$ax$}\UnaryInfC{$\vdash a, \neg a$}
            \RightLabel{$\vee$}\UnaryInfC{$\vdash a \vee \neg a, \neg a$}
            \RightLabel{$\vee$}\UnaryInfC{$\vdash a \vee \neg a, a \vee \neg a$}
            \AxiomC{}
            \RightLabel{$\top$}\UnaryInfC{$\vdash \top$}
            \RightLabel{$w$}\UnaryInfC{$\vdash \top, a \vee \neg a$}
            \RightLabel{$\wedge$}\BinaryInfC{$\vdash (a \vee \neg a) \wedge \top, a \vee \neg a$}
            \AxiomC{}
            \RightLabel{$\top$}\UnaryInfC{$\vdash \top$}
            \RightLabel{$w$}\UnaryInfC{$\vdash \top, (a \vee \neg a) \wedge \top$}
            \RightLabel{$\wedge$}\BinaryInfC{$\vdash (a \vee \neg a) \wedge \top, (a \vee \neg a) \wedge \top$}
            \RightLabel{$c$}\UnaryInfC{$\vdash (a \vee \neg a) \wedge \top$}
        \end{prooftree}
    \end{example*}


    \begin{definition}[Classes of Formulae]\label{sec-ctr:formulae-classes}
        Let $D^i$ be the subclasses of formulae defined as:
        \begin{align*}
            D^1 \quad &\defeq \quad \top \,|\, D^1 \wedge D^1 \,|\, D^* \vee D^1 \\
            D^2 \quad &\defeq \quad D^{i \geq 2} \vee D^2 \,|\, P \vee \neg P \,|\, \ldots
        \end{align*}
        where $D^* \defeq \bigcup_{i \in \mathbb{N}} D_i$, such that $D^n$ is the class of all formulae provable in n dimensions.

        \textit{This definition will be revised later.}
    \end{definition}


    \begin{remark*}[Satisfiability vs Provability]
        For any formula $P$, there exist four distinct classes: \textit{true, false, satisfiably true, satisfiably false}, where satisfiable differs by finding a particular assignment of values to each variable.
        Coalescence searches for a proof of $P$ in \textit{true}, whereas the SAT problem addresses a proof of $P$ in \textit{true}.
    \end{remark*}
    
    
    \begin{proposition*}[Bounds on Dimensionality]
        Given a formula $P$, the following provides a bound its dimensionality:
        \begin{equation*}
            P \in D^n \implies n+1 \leq vars(P)
        \end{equation*}
        Justification is left unaddressed here as the current construction does not emit a `nice' proof.
    \end{proposition*}

    \begin{remark*}
        These methods do not lend themselves to `nice' properties.
        It is possible that in the general case, \textit{finding an exact value for the dimensionality of a given formula $P$ may be equivalent to finding a proof of $P$}, but it remains to be proven.
        Instead, a natural optimisation is applied to the algorithm as follows, giving rise to a different set of classes of dimensionality and more pleasing properties.
    \end{remark*} 

    \begin{definition*}[$\top$-substitution]
         Given a sequent expressing a proof of $Q$, i.e. $\vdash Q, Q, \ldots Q$, the formula $P$ may be equivalently expressed $P[Q \seteq \top]$.
    \end{definition*}

    \begin{example*}
        Consider the formula $P \defeq (a \vee \neg a) \wedge (b \vee \neg b)$.
        The sequent proof $\vdash P$ is large and has dimensionality 3 as follows:
        \begin{center}
            \begin{tabular}{@{}l@{}}
                \begin{varwidth}{\linewidth}
                    \begin{scprooftree}{0.40}
                        \AxiomC{$ \vdash a , a , \neg a $}
                        \UnaryInfC{$ \vdash a , ( a \vee \neg a ) , \neg a $}
                        \AxiomC{$ \vdash a , \neg a , b $}
                        \UnaryInfC{$ \vdash a , \neg a , ( b \vee \neg b ) $}
                        \BinaryInfC{$ \vdash a , \neg a , ( ( a \vee \neg a ) \wedge ( b \vee \neg b ) ) $}
                        \UnaryInfC{$ \vdash a , ( a \vee \neg a ) , ( ( a \vee \neg a ) \wedge ( b \vee \neg b ) ) $}
                        \AxiomC{$ \vdash a , \neg a , b $}
                        \UnaryInfC{$ \vdash a , \neg a , ( b \vee \neg b ) $}
                        \UnaryInfC{$ \vdash a , ( a \vee \neg a ) , ( b \vee \neg b ) $}
                        \AxiomC{$ \vdash a , b , \neg b $}
                        \UnaryInfC{$ \vdash a , b , ( b \vee \neg b ) $}
                        \UnaryInfC{$ \vdash a , ( b \vee \neg b ) , ( b \vee \neg b ) $}
                        \BinaryInfC{$ \vdash a , ( ( a \vee \neg a ) \wedge ( b \vee \neg b ) ) , ( b \vee \neg b ) $}
                        \BinaryInfC{$ \vdash a , ( ( a \vee \neg a ) \wedge ( b \vee \neg b ) ) , ( ( a \vee \neg a ) \wedge ( b \vee \neg b ) ) $}
                        \UnaryInfC{$ \vdash ( a \vee \neg a ) , ( ( a \vee \neg a ) \wedge ( b \vee \neg b ) ) , ( ( a \vee \neg a ) \wedge ( b \vee \neg b ) ) $}
                        \AxiomC{$ \vdash a , \neg a , b $}
                        \UnaryInfC{$ \vdash a , \neg a , ( b \vee \neg b ) $}
                        \UnaryInfC{$ \vdash a , ( a \vee \neg a ) , ( b \vee \neg b ) $}
                        \AxiomC{$ \vdash a , b , \neg b $}
                        \UnaryInfC{$ \vdash a , b , ( b \vee \neg b ) $}
                        \UnaryInfC{$ \vdash a , ( b \vee \neg b ) , ( b \vee \neg b ) $}
                        \BinaryInfC{$ \vdash a , ( ( a \vee \neg a ) \wedge ( b \vee \neg b ) ) , ( b \vee \neg b ) $}
                        \UnaryInfC{$ \vdash ( a \vee \neg a ) , ( ( a \vee \neg a ) \wedge ( b \vee \neg b ) ) , ( b \vee \neg b ) $}
                        \AxiomC{$ \vdash a , b , \neg b $}
                        \UnaryInfC{$ \vdash a \vee \neg a , b, \neg b $}
                        \AxiomC{$ \vdash b , b , \neg b $}
                        \UnaryInfC{$ \vdash b \vee \neg b , b , \neg b $}
                        \BinaryInfC{$ \vdash ( ( a \vee \neg a ) \wedge ( b \vee \neg b ) ) , b , \neg b $}
                        \UnaryInfC{$ \vdash ( ( a \vee \neg a ) \wedge ( b \vee \neg b ) ) , b , ( b \vee \neg b ) $}
                        \UnaryInfC{$ \vdash ( ( a \vee \neg a ) \wedge ( b \vee \neg b ) ) , ( b \vee \neg b ) , ( b \vee \neg b ) $}
                        \BinaryInfC{$ \vdash ( ( a \vee \neg a ) \wedge ( b \vee \neg b ) ) , ( ( a \vee \neg a ) \wedge ( b \vee \neg b ) ) , ( b \vee \neg b ) $}
                        \BinaryInfC{$ \vdash ( ( a \vee \neg a ) \wedge ( b \vee \neg b ) ) , ( ( a \vee \neg a ) \wedge ( b \vee \neg b ) ) , ( ( a \vee \neg a ) \wedge ( b \vee \neg b ) ) $}
                    \end{scprooftree}
                \end{varwidth}
            \end{tabular}
        \end{center}
        
        Using substitutions of $\top$ for provable subformulae, the proof $\vdash P$ becomes much more manageable:

        \begin{center}
            \begin{tabular}{@{}l@{}}
                $ A := ( a \vee \neg a ) $
                \begin{varwidth}{\linewidth}
                    \begin{prooftree}
                        \AxiomC{$ \vdash a , \neg a $}
                        \UnaryInfC{$ \vdash a , ( a \vee \neg a ) $}
                        \UnaryInfC{$ \vdash ( a \vee \neg a ) , ( a \vee \neg a ) $}
                    \end{prooftree}
                \end{varwidth}\\ \\

                $ B := ( b \vee \neg b ) $
                \begin{varwidth}{\linewidth}
                    \begin{prooftree}
                        \AxiomC{$ \vdash b , \neg b $}
                        \UnaryInfC{$ \vdash b , ( b \vee \neg b ) $}
                        \UnaryInfC{$ \vdash ( b \vee \neg b ) , ( b \vee \neg b ) $}
                    \end{prooftree}
                \end{varwidth}\\ \\

                \begin{varwidth}{\linewidth}
                    \begin{prooftree}
                        \AxiomC{$ \vdash A , A $}
                        \AxiomC{$ \vdash A , B $}
                        \BinaryInfC{$ \vdash A , ( A \wedge B ) $}
                        \AxiomC{$ \vdash A , B $}
                        \AxiomC{$ \vdash B , B $}
                        \BinaryInfC{$ \vdash B , ( A \wedge B ) $}
                        \BinaryInfC{$ \vdash ( A \wedge B ) , ( A \wedge B ) $}
                    \end{prooftree}
                \end{varwidth}\\
            \end{tabular}
        \end{center}

        While the final step need only be performed with dimensionality 1, the prior steps require dimensionality 2, giving the full formula a dimensionality of 2.
    \end{example*}


    \begin{remark*}
        It is now important to reexamine the coalescence algorithm.

        Storing a token as a \texttt{Tuple} gives an inefficient algorithm due to the implicit ordering.
        By commutativity, given a token $(a_1 \ldots a_n)$, there exist up to $n!$ permutations of $a_i$'s and thus $n!$ equivalences per token.
        
        Instead, order is abstracted over and a token is stored as a \texttt{Multiset}.
        This gives the storage optimisation exploiting the symmetry of petri nets that expresses commutativity of terms in a sequent.
        There are now only $n$ equivalences per token.

        This may be taken a step further.
        Consider a token stored as a \texttt{Set} which expresses idempotency of terms in a sequent $\vdash P, P \equiv \, \vdash P$.
        This forms a canonical representation of tokens --- the only token equivalent to $T \in \texttt{Set}$ is T itself.
    \end{remark*}

    \begin{examples}
        Consider again the formula $(a \vee \neg a) \wedge (b \vee \neg b)$, with a proofs as follow, but instead tracing the values of tokens.
        
        Where tokens are member of the category \texttt{Tuple}, so as to demonstrate the exponential blow-up, both the `minimal' (black) and `saturated' (grey) proofs are as follows:
        \begin{figure}[H]
    \centering
    \begin{tikzpicture}[every node/.style={scale=0.7}]
        \def\x{1cm}
        \def\y{1cm}
        
        \node[anchor=center](root){};
        \node[on grid, left=       4*\x of root](lroot){};
        \node[on grid, left=       2*\x of lroot  ](13){$(1, 3)$};
        \node[on grid, right=      2*\x of lroot  ](31){$(3, 1)$};
        \node[on grid, below left= \y and \x of 13](23){$(2, 3)$};
        \node[on grid, below right=\y and \x of 13](12){$(1, 2)$};
        \node[on grid, below left= \y and \x of 31](21){$(2, 1)$};
        \node[on grid, below right=\y and \x/2 of 31](32){$(3, 2)$};
        \node[on grid, below right=\y and \x of 12](22){$(2, 2)$};
                        
        \node[on grid, right=      2*\x of root](rroot){};
        \node[on grid, left=       2*\x of rroot  ](57){$(5, 7)$};
        \node[on grid, right=      2*\x of rroot  ](75){$(7, 5)$};
        \node[on grid, below left= \y and \x/2 of 57](67){$(6, 7)$};
        \node[on grid, below right=\y and \x of 57](56){$(5, 6)$};
        \node[on grid, below left= \y and \x of 75](65){$(6, 5)$};
        \node[on grid, below right=\y and \x of 75](76){$(7, 6)$};
        \node[on grid, below right=\y and \x of 56](66){$(6, 6)$};
        
        \node[on grid, below left= \y and 2*\x of 22](224){$(2, 2, 4)$};
        \node[on grid, below left= \y and 0*\x of 22](242){$(2, 4, 2)$};
        \node[on grid, below right=\y and 2*\x of 22](422){$(4, 2, 2)$};
        \node[on grid, below left= \y and 2*\x of 66](466){$(4, 6, 6)$};
        \node[on grid, below left= \y and 0*\x of 66](646){$(6, 4, 6)$};
        \node[on grid, below right=\y and 2*\x of 66](664){$(6, 6, 4)$};
       
        \node[on grid, below left= 2*\y and 3*\x of 22](226){$(2, 2, 6)$};
        \node[on grid, below left= 2*\y and \x of 22  ](262){$(2, 6, 2)$};
        \node[on grid, below right=2*\y and \x of 22  ](622){$(6, 2, 2)$};
        \node[on grid, below left= 2*\y and 3*\x of 66](266){$(2, 6, 6)$};
        \node[on grid, below left= 2*\y and \x of 66  ](626){$(6, 2, 6)$};
        \node[on grid, below right=2*\y and \x of 66  ](662){$(6, 6, 2)$};

        \coordinate[on grid, below=\y of 226   ](246c){};
        \coordinate[on grid, right=2*\x of 246c](264c){};
        \coordinate[on grid, right=2*\x of 264c](426c){};
        \coordinate[on grid, right=2*\x of 426c](462c){};
        \coordinate[on grid, right=2*\x of 462c](624c){};
        \coordinate[on grid, right=2*\x of 624c](642c){};

        \node[on grid, below=\y/2 of 246c](246){$(2, 4, 6)$};
        \node[on grid, below=\y/2 of 264c](264){$(2, 6, 4)$};
        \node[on grid, below=\y/2 of 426c](426){$(4, 2, 6)$};
        \node[on grid, below=\y/2 of 462c](462){$(4, 6, 2)$};
        \node[on grid, below=\y/2 of 624c](624){$(6, 2, 4)$};
        \node[on grid, below=\y/2 of 642c](642){$(6, 4, 2)$};

        \coordinate[on grid, below=2.5*\y of 224](224c){};
        \coordinate[on grid, below=2.5*\y of 242](242c){};
        \coordinate[on grid, below=2.5*\y of 422](422c){};
        \coordinate[on grid, below=2.5*\y of 466](466c){};
        \coordinate[on grid, below=2.5*\y of 646](646c){};
        \coordinate[on grid, below=2.5*\y of 664](664c){};
    
        \coordinate[on grid, below left= 1.5*\y and \x/2 of 224c](244lc){};
        \coordinate[on grid, below right=1.5*\y and \x/2 of 224c](244rc){};
        \coordinate[on grid, below left= 1.5*\y and \x/2 of 242c](424lc){};
        \coordinate[on grid, below right=1.5*\y and \x/2 of 242c](424rc){};
        \coordinate[on grid, below left= 1.5*\y and \x/2 of 422c](442lc){};
        \coordinate[on grid, below right=1.5*\y and \x/2 of 422c](442rc){};
        
        \coordinate[on grid, below left= 1.5*\y and \x/2 of 664c](644lc){};
        \coordinate[on grid, below right=1.5*\y and \x/2 of 664c](644rc){};
        \coordinate[on grid, below left= 1.5*\y and \x/2 of 646c](464lc){};
        \coordinate[on grid, below right=1.5*\y and \x/2 of 646c](464rc){};
        \coordinate[on grid, below left= 1.5*\y and \x/2 of 466c](446lc){};
        \coordinate[on grid, below right=1.5*\y and \x/2 of 466c](446rc){};

        \node[on grid, below=2*\y of 224c](244){$(2, 4, 4)$};
        \node[on grid, below=2*\y of 242c](424){$(4, 2, 4)$};
        \node[on grid, below=2*\y of 422c](442){$(4, 4, 2)$};
        \node[on grid, below=2*\y of 466c](446){$(4, 4, 6)$};
        \node[on grid, below=2*\y of 646c](464){$(4, 6, 4)$};
        \node[on grid, below=2*\y of 664c](644){$(6, 4, 4)$};

        \draw[-arr] (13) to (23);
        \draw[-arr] (13) to (12);
        \draw[-arr] (31) to (21);
        \draw[-arr] (31) to (32);
        \draw[-arr] (23) to (22);
        \draw[-arr] (12) to (22);
        \draw[-arr] (21) to (22);
        \draw[-arr] (32) to (22);
        
        \draw[-arr] (57) to (67);
        \draw[-arr] (57) to (56);
        \draw[-arr] (75) to (65);
        \draw[-arr] (75) to (76);
        \draw[-arr] (67) to (66);
        \draw[-arr] (56) to (66);
        \draw[-arr] (65) to (66);
        \draw[-arr] (76) to (66);        
        
        \drawsquig (22) to (224);
        \drawsquig (22) to (242);
        \drawsquig (22) to (422);
        \drawsquig (66) to (466);
        \drawsquig (66) to (646);
        \drawsquig (66) to (664);

        \drawsquig (22) to (226);
        \drawsquig (22) to (262);
        \drawsquig (22) to (622);
        \drawsquig (66) to (266);
        \drawsquig (66) to (626);
        \drawsquig (66) to (662);

        \draw[-] (226) to (246c) to (266);
        \draw[-] (226) to (426c) to (626);
        \draw[-] (262) to (264c) to (266);
        \draw[-] (262) to (462c) to (662);
        \draw[-] (622) to (624c) to (626);
        \draw[-] (622) to (642c) to (662);

        \draw[-arr] (246c) to (246);
        \draw[-arr] (264c) to (264);
        \draw[-arr] (426c) to (426);
        \draw[-arr] (462c) to (462);
        \draw[-arr] (624c) to (624);
        \draw[-arr] (642c) to (642);

        \draw[-] (224) to (224c);
        \draw[-] (242) to (242c);
        \draw[-] (422) to (422c);
        \draw[-] (466) to (466c);
        \draw[-] (646) to (646c);
        \draw[-] (664) to (664c);

        \draw[-] (246) to (244lc) to (242c);
        \draw[-] (246) to (446lc) to (646c);
        \draw[-] (264) to (244rc) to (224c);
        \draw[-] (264) to (464lc) to (664c);
        \draw[-] (426) to (424lc) to (422c);
        \draw[-] (426) to (446rc) to (466c);
        \draw[-] (462) to (464rc) to (466c);
        \draw[-] (462) to (442lc) to (422c);
        \draw[-] (624) to (644lc) to (664c);
        \draw[-] (624) to (424rc) to (224c);
        \draw[-] (642) to (644rc) to (646c);
        \draw[-] (642) to (442rc) to (242c);

        \draw[-arr] (244lc) to (244);
        \draw[-arr] (424lc) to (424);
        \draw[-arr] (442lc) to (442);
        \draw[-arr] (644lc) to (644);
        \draw[-arr] (464lc) to (464);
        \draw[-arr] (446lc) to (446);
        \draw[-arr] (244rc) to (244);
        \draw[-arr] (424rc) to (424);
        \draw[-arr] (442rc) to (442);
        \draw[-arr] (644rc) to (644);
        \draw[-arr] (464rc) to (464);
        \draw[-arr] (446rc) to (446);

    \end{tikzpicture}
\end{figure}
 where $\rightsquigarrow$ is an application of weakening.
        This is notable as this will be the point coalescence will reach for $n \seteq 2$ and a suitably intelligent method for extending dimensions will maintain the proof search so far.
        Both `minimal' and `saturated' proofs are included as each represent the deterministic versus non-deterministic searches --- proof search versus validation.

        Where tokens are now members of \texttt{Multiset}, the proof of the full formula follows:
        \begin{figure}[H]
    \def\x{1cm}
    \def\y{1cm}
    \centering
    \begin{tikzpicture}[every node/.style={scale=0.7}]
        \node[anchor=center](root){};
        \node[on grid, left=       2*\x of root   ](13){$<1, 3>$};
        \node[on grid, right=      2*\x of root   ](57){$<5, 7>$};
        \node[on grid, below left= \y and \x of 13](23){$<2, 3>$};
        \node[on grid, below right=\y and \x of 13](12){$<1, 2>$};
        \node[on grid, below left= \y and \x of 57](67){$<6, 7>$};
        \node[on grid, below right=\y and \x of 57](56){$<5, 6>$};
        \node[on grid, below right=\y and \x of 23](22){$<2, 2>$};
        \node[on grid, below right=\y and \x of 67](66){$<6, 6>$};

        \node[on grid, below left= \y and \x of 22](224){$<2, 2, 4>$};
        \node[on grid, below right=\y and \x of 22](226){$<2, 2, 6>$};
        \node[on grid, below left= \y and \x of 66](266){$<2, 6, 6>$};
        \node[on grid, below right=\y and \x of 66](466){$<4, 6, 6>$};

        \coordinate[on grid, right=\x of 226](246c){};
        \node[on grid, below right=\y and \x of 226](246){$<2, 4, 6>$};
        \coordinate[on grid, left =2*\x of 246](244c){};
        \node[on grid, below left =\y and \x of 246](244){$<2, 4, 4>$};
        \coordinate[on grid, right=2*\x of 246](446c){};
        \node[on grid, below right=\y and \x of 246](446){$<4, 4, 6>$};
        \coordinate[on grid, below=\y of 246](444c){};
        \node[on grid, below right=\y and \x of 244](444){$<4, 4, 4>$};

        \draw[-arr] (13) to (23);
        \draw[-arr] (13) to (12);
        \draw[-arr] (57) to (67);
        \draw[-arr] (57) to (56);
        \draw[-arr] (23) to (22);
        \draw[-arr] (12) to (22);
        \draw[-arr] (67) to (66);
        \draw[-arr] (56) to (66);

        \drawsquig (22) to (224);
        \drawsquig (22) to (226);
        \drawsquig (66) to (266);
        \drawsquig (66) to (466);

        \draw[-] (226) to (246c) to (266);
        \draw[-] (224) to (244c) to (246);
        \draw[-] (466) to (446c) to (246);
        \draw[-] (244) to (444c) to (446);
        
        \draw[-arr] (246c) to (246);
        \draw[-arr] (244c) to (244);
        \draw[-arr] (446c) to (446);
        \draw[-arr] (444c) to (444);
    \end{tikzpicture}
\end{figure}

        The above remains tedious and unnatural.
        Proofs of each of $a \vee \neg a$ and $b \vee \neg b$ are constructed in two steps, while the remaining steps are to prove the conjunction of these (proven) subformulae.
        Instead, consider the tokens as members of \texttt{Set} as follows:

        \begin{figure}[H]
    \centering
    \begin{tikzpicture}
        \def\x{1cm}
        \def\y{1cm}
        
        \node[anchor=center](root){};
        \node[on grid, left=       2*\x of root   ](13){$\{1, 3\}$};
        \node[on grid, right=      2*\x of root   ](57){$\{5, 7\}$};
        \node[on grid, below left= \y and \x of 13](23){$\{2, 3\}$};
        \node[on grid, below right=\y and \x of 13](12){$\{1, 2\}$};
        \node[on grid, below left= \y and \x of 57](67){$\{6, 7\}$};
        \node[on grid, below right=\y and \x of 57](56){$\{5, 6\}$};
        \node[on grid, below right=\y and \x of 23](2){$\{2\}$};
        \node[on grid, below right=\y and \x of 67](6){$\{6\}$};
        \coordinate[on grid, right=2*\x of 2](4c){};
        \node[on grid, below right=\y and 2*\x of 2](4){$\{4\}$};

        \draw[-arr] (13) to (23);
        \draw[-arr] (13) to (12);
        \draw[-arr] (57) to (67);
        \draw[-arr] (57) to (56);
        \draw[-arr] (23) to (2);
        \draw[-arr] (12) to (2);
        \draw[-arr] (67) to (6);
        \draw[-arr] (56) to (6);
        \draw[-] (2) to (4c) to (6);
        \draw[-arr] (4c) to (4);
    \end{tikzpicture}
\end{figure}

        
        Equivalently with symbols:
        \begin{figure}[H]
    \def\x{1cm}
    \def\y{0.7cm}
    \centering
    \begin{tikzpicture}[every node/.style={scale=0.7}]
        \node[anchor=center](root){};
        \node[on grid, left=       2*\x of root   ](13){$\{a, \neg a\}$};
        \node[on grid, right=      2*\x of root   ](57){$\{b, \neg b\}$};
        \node[on grid, below left= \y and \x of 13](23){$\{a \vee \neg a, \neg a\}$};
        \node[on grid, below right=\y and \x of 13](12){$\{a, a \vee \neg a\}$};
        \node[on grid, below left= \y and \x of 57](67){$\{b \vee \neg b, \neg b\}$};
        \node[on grid, below right=\y and \x of 57](56){$\{b, b \vee \neg b\}$};
        \node[on grid, below right=\y and \x of 23](2){$\{a \vee \neg a\}$};
        \node[on grid, below right=\y and \x of 67](6){$\{b \vee \neg b\}$};
        \coordinate[on grid, below right=0.5*\y and 2*\x of 2](4c){};
        \node[on grid, below=0.5*\y of 4c](4){$\{(a \vee \neg a) \wedge (b \vee \neg b)\}$};

        \draw[-arr] (13) to (23);
        \draw[-arr] (13) to (12);
        \draw[-arr] (57) to (67);
        \draw[-arr] (57) to (56);
        \draw[-arr] (23) to (2);
        \draw[-arr] (12) to (2);
        \draw[-arr] (67) to (6);
        \draw[-arr] (56) to (6);
        \draw[-] (2) to (4c) to (6);
        \draw[-arr] (4c) to (4);
    \end{tikzpicture}
\end{figure}

        This is finally deemed a `satisfying' proof --- dimensionality is extracted as the largest set required in the proof, a nicer algebra of dimensionality may be constructed and there is some essence of justification for the correctness of this $\top$-substitution --- all through changing the datatype of tokens.
    \end{examples}
    
    \begin{remark}
        In essence, allowing for efficient substitution of a $\top$ term for proven subformulae allows the coalescence algorithm to run without restricting the output to additively stratified proofs.
        A proof may be reconstructed including substitutions, where each subproof substituted is itself additively stratified, but now there are contractions for the conclusion of a subproof on top of weakenings for the axioms of a `superproof'.
    \end{remark}


    \begin{definition*}[Classes of Formulae]
        With this new concept of coalescence, dimensionality and proof search, the definitions in~\ref{sec-ctr:formulae-classes} are revisited and revised.
        Let $\mathcal{D}^i$ be defined as the subclasses of formulae:
        \begin{align*}
            \mathcal{D}^1 \quad \defeq \quad &\top \quad | \quad \mathcal{D}^1 \wedge \mathcal{D}^1 \quad | \quad \mathcal{D}^* \vee \mathcal{D}^1 \\
            \mathcal{D}^2 \quad \defeq \quad &P \vee \neg P \quad | \quad \mathcal{D}^2 \vee \mathcal{D}^{i \geq 2} \quad | \quad \mathcal{D}^2 \wedge \mathcal{D}^{i \leq 2} \\
            \mathcal{D}^3 \quad \defeq \quad &(P \wedge Q) \vee (P \wedge \neg Q) \vee (\neg P \wedge Q) \vee (\neg P \wedge \neg Q) \\
                                       \quad &| \quad \mathcal{D}^3 \vee \mathcal{D}^{i \geq 3} \quad | \quad \mathcal{D}^3 \wedge \mathcal{D}^{i \leq 3}
        \end{align*}
        where $\mathcal{D}^* \defeq \bigcup_{i \in \mathbb{N}} \mathcal{D}^i$, $P \in \mathcal{D}^*$, such that $\mathcal{D}^n$ is the class of all formulae provable in exactly n dimensions.
    \end{definition*}
    
    \begin{corollary}\label{cory:dimensionality-algebra}
        For $P \in \mathcal{D}^i, Q \in \mathcal{D}^j$, the following statements hold:
        \begin{equation*}
            P \vee Q \in \mathcal{D}^{\min{i, j}}
            \qquad
            P \wedge Q \in \mathcal{D}^{\max{i, j}}
        \end{equation*}
        In fact, for $P \in \mathcal{D}^n$ in $v$ variables, $n \leq v + 1$.
    \end{corollary}


    \begin{definition*}[Assignments]
        Given $n$ variables $a_1 \ldots a_n$, then the conjunction of each combination of atoms (with or without negation) is called an \textit{assignment} of boolean values to these variables.
        For $n$ variables, there exist $2^n$ possible \textit{assignments}, each analogous to one of:
        \begin{align*}
            (a_1 \seteq \textit{True} \quad &\ldots \quad a_n \seteq \textit{True}) \\
            (a_1 \seteq \textit{False}, \, a_2 \seteq \textit{True} \quad &\ldots \quad a_n \seteq \textit{True}) \\
            &\ldots \\
            (a_1 \seteq \textit{False} \quad &\ldots \quad a_{n-1} \seteq \textit{False}, \, a_n \seteq \textit{True}) \\
            (a_1 \seteq \textit{False} \quad &\ldots \quad a_n \seteq \textit{False})
        \end{align*}

        An assignment $\Delta$ is found in \textit{DNF} terms as $\delta_j \defeq a_1 \wedge \ldots \wedge a_n$ and can therefore be seen as satisfiably true for a unique assignment of values to each $a_i$.
    \end{definition*}
    
    \begin{proposition}[Dimensionality Bound of Disjunctive Normal Form]\label{propn:dimensionality-bounds}
        Let $\textit{DNF}^n$ denote the set of all formulae in $n$ variables in disjunctive normal form.
        That is, $P \in \textit{DNF}^n \implies P = \bigvee_{j = 1 .. 2^n} (\bigwedge_{i = 1 .. n} a_i)$ the disjunction of $2^n$ conjunctions, each a unique assignment of $n$ variables.
        Then the following statement holds:
        \begin{equation*}
            P \in \textit{DNF}^n \implies P \in \mathcal{D}^{n+1}
        \end{equation*}
    \end{proposition}

    Two proofs of this statement are presented.
    The former may be most intuitive to follow and hold the fewest edge-cases, but breaks away from how proof search and calculation of dimensionality is performed in the coalescence algorithm.
    To this end, the latter is also included which stays truer to the methods used.

    \begin{proof}[Sequent Tree Proof]
        Working from the bottom of the proof tree upwards, suppose there exists a proof of $Q \seteq \bigvee_j \delta_j \in \textit{DNF}^n$.
        Let this be written as the sequent $\vdash Q \ldots_m Q$ for some $m$ --- the proof seeks to show $m = n+1$.
        Suppose $\Delta_i \seteq a_1 \ldots a_n$ the `all true' assignment with associated conjunction $\delta_i$, then any one of these $Q$ must have been derived as follows:
        \begin{prooftree}
            \AxiomC{}
            \doubleLine\UnaryInfC{$\vdash Q \ldots_{m-1} Q, a_1$}
            \AxiomC{$\ldots_n$}
            \AxiomC{}
            \doubleLine\UnaryInfC{$\vdash Q \ldots_{m-1} Q, a_n$}
            \doubleLine\TrinaryInfC{$\vdash Q \ldots_{m-1} Q, \delta_i$}
            \doubleLine\UnaryInfC{$\vdash Q \ldots_m Q$}
        \end{prooftree}
        
        Without loss of generality, focus on the rightmost subtree and recurse, then second-rightmost subtree and recurse:
        \begin{scprooftree}{0.7}
            \AxiomC{}
            \doubleLine\UnaryInfC{$\vdash Q \ldots_{m-3} Q, a_1, a_{n-1}, a_n$}
            \AxiomC{$\ldots_n$}
            \AxiomC{}
            \doubleLine\UnaryInfC{$\vdash Q \ldots_{m-3} Q, a_n, a_{n-1}, a_n$}
            \doubleLine\TrinaryInfC{$\vdash Q \ldots_{m-3} Q, \delta_i, a_{n-1}, a_n$}
            \doubleLine\UnaryInfC{$\vdash Q \ldots_{m-2} Q, a_1, a_n$}
            \AxiomC{$\ldots_n$}
            \AxiomC{}
            \doubleLine\UnaryInfC{$\vdash Q \ldots_{m-2} Q, a_n, a_n$}
            \doubleLine\TrinaryInfC{$\vdash Q \ldots_{m-2} Q, \delta_i, a_n$}
            \doubleLine\UnaryInfC{$\vdash Q \ldots_{m-1} Q, a_n$}
        \end{scprooftree}

        Then, by induction on $n$, the process may be repeated until there exists a sequent containing atoms for each $a_i \in \delta_i$ as seen in the upper left here:
        \begin{prooftree}
            \AxiomC{}
            \doubleLine\UnaryInfC{$\vdash Q \ldots_{m-n} Q, a_1, a_2 \ldots_{n-1} a_n$}
            \AxiomC{$\ldots_n$}
            \AxiomC{}
            \doubleLine\UnaryInfC{$\vdash Q \ldots_{m-n} Q, a_n, a_2 \ldots_{n-1} a_n$}
            \doubleLine\TrinaryInfC{$\vdash Q \ldots_{m-n} Q, \delta_i, a_n$}
            \doubleLine\UnaryInfC{$\vdash Q \ldots_{m+1-n} Q, a_2 \ldots_{n-1} a_n$}
        \end{prooftree}
        
        Now, every other branch in as many steps will have \textit{at most} as many unique atoms as the above, if not less.
        For the sequent $\vdash Q \ldots_{m-n} Q, a_1 \ldots_n a_n$ to reach the top $\vdash \top$, all that is required is a single $\neg a_i$.
        This similarly applies for all other leaves.

        This means the sequent $\vdash a_i, a_1 \ldots a_n$ (accompanied by $n^n$ other sequents, most simpler than this `worst-case') is sufficient to provide a proof of $Q$.
        Therefore, $m = n + 1$ as required and the proof is complete.
    \end{proof}

    \begin{remark*}
        This bounds the exponential blowup of \textit{DNF} formulae as $\mathcal{O}(n^n$) --- an exceptionally poor worst-case,
    \end{remark*}

    \begin{proof}[ALL Proof]
        By induction on the number of variables $n$.

        Let the base case be $n \seteq 1$ and the proof is trivial --- the sequent proof $\vdash a \vee \neg a$ requires 2 dimensions and the hypothesis holds.

        Suppose the inductive hypothesis holds up to $n - 1$ variables --- in $i$ dimensions where $i \leq n$ and all \textit{DNF}-terms in $v$ variables where $i - 1 \equiv v < n$ such that:
        \begin{equation*}
            \forall \, i \leq n \,:\, P \in \textit{DNF}^{i} \implies P \in \mathcal{D}^{i}
        \end{equation*}
        It is now sufficient to show the hypothesis continues to hold for up to $n$ variables:
        \begin{equation*}
            \forall \, Q \in \textit{DNF}^n :  Q \in \mathcal{D}^{n+1}
        \end{equation*}

        Let $a_n$ be fresh in $P$ and $a_1 \ldots a_{n-1}$ be used.
        Write $P \defeq \bigvee_j \delta_j \quad \forall \, j \in 1 \ldots 2^{n-1}, \, \forall j \in 1 \ldots n-1$ where each $\delta_j$ is the $j$'th assignment of variables $a_1 \ldots a_{n-1}$.
        Now construct $Q$ as follows:
        \begin{equation*}
            Q \seteq Q_{l} \vee Q_{r} \seteq (\bigvee_j \delta_j \wedge a_n) \vee (\bigvee_j \delta_j \wedge \neg a_n) \in \textit{DNF}^n
        \end{equation*}
        where $Q$ is the simplest $\textit{DNF}^n$ extension of $P$.
        Each $\delta_j$ is of course not provable by itself.
        Since the $P$ term has been deconstructed, $P$ provable only in $n$ dimensions or greater and therefore the conditions for the inductive hypothesis have been broken (by contrapositive as the dimensionality is too high), $Q$ is therefore not provable in $n$ dimensions.

        Choose an assignment in $Q$ --- without loss of generality, choose $\Delta_Q \seteq a_1 \ldots a_n$ the `all true' assignment and the associated $d_Q$ conjunction.
        Since $Q$ not provable in $n$ dimensions, extend the proof to $n+1$ dimensions --- on the left, the leaf sequents of $Q_l$ may be weakened from $\vdash \Gamma$ all atomic terms to $\vdash \Gamma, \neg a_n$ giving at least $n + 1$ dimensionality as required (if the formula can now be proven of course).

        Let $a_i$ be as in the decided assignment $\Delta_Q$.
        There exist a sufficient number of sequents containing assignments $\Delta_i$ for $P$ of $n-1$ variables (and thus $n-1$ sequent terms) $\vdash a_j, \Delta_i, a_i$ such that $a_j \in \Delta_Q$ and $\neg a_i \in \Delta_i$.
        By applying the steps made in $\vdash P$, each $\vdash a_j, \Delta_i, a_i$ can produce a sequent proof down to $\vdash a_j, \delta_i \ldots_n \delta_i$ for any given $a_j \in \Delta$, so construct for all of them.
        By applying the $\wedge$ rule across all these subtrees, this produces $\vdash \delta_j \wedge a_n, \delta_i \ldots_n \delta_i \,\, \equiv \,\, \vdash Q_l, \delta_i \ldots_n \delta_i \,\, \equiv \,\, \vdash Q, \delta_i \ldots_n \delta_i$.

        Tediously, from here the proof can be grown up to the following sequents:
        \begin{equation*}\label{eqn:proof-omission}
            \vdash Q \ldots_n Q, \delta_i \qquad \vdash Q \ldots_n Q, a_n \text{\, for \,} a_n \in \Delta_Q \tag{+}
        \end{equation*}
        
        Reducing across subtrees by the $\wedge$ rule yields $\vdash Q \ldots_n Q, \delta_i \wedge a_n$.
        Finally, the $\vee$ rule may be applied up the formula syntax tree to yield $\vdash Q \ldots_{n} Q, Q \equiv \, \vdash Q \ldots_{n+1} Q \implies Q \in \mathcal{D}^{n+1}$, where $Q \in \textit{DNF}^n$ as required.
    \end{proof}

    \begin{remark}\label{remark:proof-subtleties}
        The latter proof omits the subtleties and specifics of deriving the results in~\eqref{eqn:proof-omission}.
        This is considered tedious and does not aid to a clear representation of the intermediate steps but remains to be formally proven, should such a thing be desirable.
    \end{remark}


    \begin{proposition*}
        Building on the above, the dimensionality of any given formula can be predicted by examining the most variables in a disjunctive normal form subformula:
        \begin{equation*}
            P \in \mathcal{D}^{n+1} \iff \exists \, Q \in P \,:\, Q \in DNF^{n}
        \end{equation*}
    \end{proposition*}

    \begin{proof}
        It can be trivially seen that by combining the results in Corollary~\ref{cory:dimensionality-algebra} and Proposition~\ref{propn:dimensionality-bounds} that the statement is true.
        
        $(\impliedby)$
        The reverse implication is given exactly by Proposition~\ref{propn:dimensionality-bounds}.

        $(\implies)$
        Suppose $P \in \mathcal{D}^{n+1}$.
        Then, one of the following is true:

        \begin{enumerate}
            \item
            $P \equiv (L \wedge R)$  where there exist proofs for each of $L, R$ and by Corollary~\ref{cory:dimensionality-algebra}, without loss of generality, $L \in \mathcal{D}^{n+1}$.
            Then, the proof recurses on $L$.

            \item
            $P \equiv (L \vee R)$ where there exist proofs for each of $L, R$ and by Corollary~\ref{cory:dimensionality-algebra} $L, R \in \mathcal{D}^{n+1}$.
            Then, the proof recurses on both $L$ and $R$.

            \item
            $P \equiv (L \vee R)$ where there do not exist proofs for both $L, R$.
            Therefore, if there is a proof of $P$, then $P \in \textit{DNF}$ and $P \in \mathcal{D}^{n+1} \implies P \in \textit{DNF}^n$ as required.
        \end{enumerate}
    \end{proof}
