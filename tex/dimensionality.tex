\section{Dimensionality}
    
    \begin{definition*}[Dimensionality]
        Given a formula $P$, the coalescence proof search produces a proof in an $n$-dimensional petri net.
        Equivalently, an additively stratified sequent proof $\vdash P$ has $n$ terms at the bottom before any contractions are applied.
        The dimensionality of $P$ is then defined $\dim(P) \defeq n$.
    \end{definition*}

    \begin{example}
    \end{example}


    \begin{definition*}[Classes of Formulae]
        Let $A^i$ be the subclasses of formulae defined as:
        \begin{align*}
            A^1 \quad &\defeq \quad \top \,|\, \bot \,|\, A^1 \wedge A^1 \,|\, A^* \vee A^1 \\
            A^2 \quad &\defeq \quad A^1 \,|\, \textit{Additive Linear Logic} \,|\, A^* \vee A^2 \,|\, P \vee \neg P
        \end{align*}
        where $A^* \defeq \bigcup_{i \in \mathbb{N}} A_i$, such that $A^n$ is the class of all formulae provable in n dimensions.
    \end{definition*}


    \begin{remark*}[Satisfiability vs Provability]
        For any formula $P$, there exist four distinct classes: \textit{true, false, satisfiably true, satisfiably false}, where satisfiable differs by finding a particular assignment of values to each variable.
        Coalescence searches for a proof of $P$ in \textit{true}, whereas the SAT problem addresses a proof of $P$ in \textit{true}.
    \end{remark*}
    
    
    \begin{proposition*}[$ax$-Bound on Dimensionality]
        \begin{align*}
            \dim(P) &\leq 1 + \#\{ax\textit{-Rule} \in \,\, \vdash P\} \\
            \dim(P) &\leq 1 + \#\{vars \in P\}
        \end{align*}
    \end{proposition*}

    \begin{proof}
        The first case is trivial.
        Given a formula $P$ and a additively-stratified sequent proof $\vdash P$ (through coalescence or otherwise), additive stratification ensures that $\dim(P) = n \implies \#leaves(P) \geq n$.
        As each of the $n$ branches of the tree moves `upwards', they must either terminate at an $ax$-rule or branch, leaving $n + 1$ branches total.
        Subsequently, for each $ax$-rule forming a leaf of the tree, $\dim(P)$ increases by no more than one.
        Finally, consider a base case such as $P \seteq \vdash a \vee \neg a$ where $\dim(P) = 2 \leq 1 + \#\{ax-Rule \in \vdash P\}$ (in this case, it is equal).
        
        Second case omitted. %% TODO
    \end{proof}


    \begin{remark*}
        These methods do not lend themselves to `nice' properties.
        It is possible that in the general case, \textit{finding an exact value for the dimensionality of a given formula $P$ may be equivalent to finding a proof of $P$}, but it remains to be proven.
        Instead, a natural optimisation is applied to the algorithm as follows, giving rise to a different set of classes of dimensionality and more pleasing properties.
    \end{remark*} 

    \begin{definition}[$\top$-substitution]
         Given a sequent expressing a proof of $Q$, i.e. $\vdash Q, Q, \ldots Q$, the formula $P$ may be equivalently expressed $P[Q \seteq \top]$.
    \end{definition}

    \begin{example}
        Consider the formula $P \defeq (a \vee \neg a) \wedge (b \vee \neg b)$.
        The sequent proof $\vdash P$ is large and has dimensionality 3 as follows:

        \begin{center}
            \begin{tabular}{@{}l@{}}
                \begin{varwidth}{\linewidth}
                    \begin{scprooftree}{0.5}
                        \AxiomC{$ \vdash a , a , \neg a $}
                        \UnaryInfC{$ \vdash a , ( a \vee \neg a ) , \neg a $}
                        \AxiomC{$ \vdash a , \neg a , b $}
                        \UnaryInfC{$ \vdash a , \neg a , ( b \vee \neg b ) $}
                        \BinaryInfC{$ \vdash a , \neg a , ( ( a \vee \neg a ) \wedge ( b \vee \neg b ) ) $}
                        \UnaryInfC{$ \vdash a , ( a \vee \neg a ) , ( ( a \vee \neg a ) \wedge ( b \vee \neg b ) ) $}
                        \AxiomC{$ \vdash a , \neg a , b $}
                        \UnaryInfC{$ \vdash a , \neg a , ( b \vee \neg b ) $}
                        \UnaryInfC{$ \vdash a , ( a \vee \neg a ) , ( b \vee \neg b ) $}
                        \AxiomC{$ \vdash a , b , \neg b $}
                        \UnaryInfC{$ \vdash a , b , ( b \vee \neg b ) $}
                        \UnaryInfC{$ \vdash a , ( b \vee \neg b ) , ( b \vee \neg b ) $}
                        \BinaryInfC{$ \vdash a , ( ( a \vee \neg a ) \wedge ( b \vee \neg b ) ) , ( b \vee \neg b ) $}
                        \BinaryInfC{$ \vdash a , ( ( a \vee \neg a ) \wedge ( b \vee \neg b ) ) , ( ( a \vee \neg a ) \wedge ( b \vee \neg b ) ) $}
                        \UnaryInfC{$ \vdash ( a \vee \neg a ) , ( ( a \vee \neg a ) \wedge ( b \vee \neg b ) ) , ( ( a \vee \neg a ) \wedge ( b \vee \neg b ) ) $}
                        \AxiomC{$ \vdash a , \neg a , b $}
                        \UnaryInfC{$ \vdash a , \neg a , ( b \vee \neg b ) $}
                        \UnaryInfC{$ \vdash a , ( a \vee \neg a ) , ( b \vee \neg b ) $}
                        \AxiomC{$ \vdash a , b , \neg b $}
                        \UnaryInfC{$ \vdash a , b , ( b \vee \neg b ) $}
                        \UnaryInfC{$ \vdash a , ( b \vee \neg b ) , ( b \vee \neg b ) $}
                        \BinaryInfC{$ \vdash a , ( ( a \vee \neg a ) \wedge ( b \vee \neg b ) ) , ( b \vee \neg b ) $}
                        \UnaryInfC{$ \vdash ( a \vee \neg a ) , ( ( a \vee \neg a ) \wedge ( b \vee \neg b ) ) , ( b \vee \neg b ) $}
                        \AxiomC{$ \vdash ( ( a \vee \neg a ) \wedge ( b \vee \neg b ) ) , b , \neg b $}
                        \UnaryInfC{$ \vdash ( ( a \vee \neg a ) \wedge ( b \vee \neg b ) ) , b , ( b \vee \neg b ) $}
                        \UnaryInfC{$ \vdash ( ( a \vee \neg a ) \wedge ( b \vee \neg b ) ) , ( b \vee \neg b ) , ( b \vee \neg b ) $}
                        \BinaryInfC{$ \vdash ( ( a \vee \neg a ) \wedge ( b \vee \neg b ) ) , ( ( a \vee \neg a ) \wedge ( b \vee \neg b ) ) , ( b \vee \neg b ) $}
                        \BinaryInfC{$ \vdash ( ( a \vee \neg a ) \wedge ( b \vee \neg b ) ) , ( ( a \vee \neg a ) \wedge ( b \vee \neg b ) ) , ( ( a \vee \neg a ) \wedge ( b \vee \neg b ) ) $}
                    \end{scprooftree}
                \end{varwidth}
            \end{tabular}
        \end{center}
        
        Using substitutions of $\top$ for provable subformulae, the proof $\vdash P$ becomes much more manageable:

        \begin{center}
            \begin{tabular}{@{}l@{}}
                $ A := ( a \vee \neg a ) $
                \begin{varwidth}{\linewidth}
                    \begin{prooftree}
                        \AxiomC{$ \vdash a , \neg a $}
                        \UnaryInfC{$ \vdash a , ( a \vee \neg a ) $}
                        \UnaryInfC{$ \vdash ( a \vee \neg a ) , ( a \vee \neg a ) $}
                    \end{prooftree}
                \end{varwidth}\\ \\

                $ B := ( b \vee \neg b ) $
                \begin{varwidth}{\linewidth}
                    \begin{prooftree}
                        \AxiomC{$ \vdash b , \neg b $}
                        \UnaryInfC{$ \vdash b , ( b \vee \neg b ) $}
                        \UnaryInfC{$ \vdash ( b \vee \neg b ) , ( b \vee \neg b ) $}
                    \end{prooftree}
                \end{varwidth}\\ \\

                \begin{varwidth}{\linewidth}
                    \begin{prooftree}
                        \AxiomC{$ \vdash A , A $}
                        \AxiomC{$ \vdash A , B $}
                        \BinaryInfC{$ \vdash A , ( A \wedge B ) $}
                        \AxiomC{$ \vdash A , B $}
                        \AxiomC{$ \vdash B , B $}
                        \BinaryInfC{$ \vdash B , ( A \wedge B ) $}
                        \BinaryInfC{$ \vdash ( A \wedge B ) , ( A \wedge B ) $}
                    \end{prooftree}
                \end{varwidth}\\
            \end{tabular}
        \end{center}

        While the final step need only be performed with dimensionality 1, the prior steps require dimensionality 2, giving the full formula a dimensionality of 2.
    \end{example}
