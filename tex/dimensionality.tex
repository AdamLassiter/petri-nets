\section{Dimensionality}
    
    \begin{definition}[Dimensionality]
        Given a formula $P$, the coalescence proof search produces a proof in an $n$ dimensional petri net.
        The dimensionality of $P$ is then defined $\dim(P) \defeq n$.
        Equivalently, an additively stratified sequent proof requires $n - 1$ contractions at the bottom of the proof.
        Given $\vdash P$, its dimensionality is defined $\dim(P) \defeq n$.
    \end{definition}

    \begin{examples}
    \end{examples}


    \begin{definition}[Classes of Formulae]
        Let $A^i$ be the subclasses of formulae defined as:
        \begin{align*}
            A^1 \quad &\defeq \quad \top \,|\, \bot \,|\, A^1 \wedge A^1 \,|\, A^* \vee A^1 \\
            A^2 \quad &\defeq \quad A^1 \,|\, \textit{Additive Linear Logic} \,|\, A^* \vee A^2 \,|\, P \vee \neg P
        \end{align*}
        where $P \in A^*$, such that $A^n$ is the class of all formulae provable in n dimensions.
    \end{definition}


    \begin{remark}[Satisfiability vs Provability]
        For any formula $P$, there exist four distinct classes: \textit{true, false, satisfiably true, satisfiably false}, where satisfiable differs by finding a particular assignment of values to each variable.
        Coalescence searches for a proof of $P$ in \textit{true}, whereas the SAT problem addresses a proof of $P$ in \textit{true}.
        In particular, \textit{true} is the complement class to \textit{satisfiably false} and \textit{false} complement to \textit{satisfiably true}. % TODO: Is this true?
    \end{remark}
    
    
    \begin{definition}[Dimensionality when not Provable]
        Given a formula $P$ such that there does not exist $\vdash P$, for all formulae $Q$ such that there does exist $\vdash Q$, the dimensionality of $P$ is defined as the least $n$ such that:
        \begin{equation*}
            \dim(P) \defeq n \implies 
            \begin{cases}
                \exists \vdash (P \vee Q) \implies \dim(P \vee Q) = \min(n, \dim(Q)) \\
                \exists \vdash (P \wedge Q) \implies \dim(P \wedge Q) = \max(n, \dim(Q))
            \end{cases}
        \end{equation*}
    \end{definition}


    \begin{remark}
        This definition of a `partial dimensionality' amounts to finding the highest dimension in coalescence proof search that some useful deduction was made, outside of trivial cases.
    \end{remark}


    \begin{definition}[Paths in a Tree]
        Path from root to any leaf (effectively iterating leaves, but tracing parents)
    \end{definition}


    \begin{proposition}[$\vee$-Bound on Dimensionality]
        \begin{align*}
            \dim(P) &\leq 1 + \#\{\vee \in P\} \\
            \dim(P) &\leq 1 + \max\{\#\{\vee \in path\} \, \forall \, path \in tree(P)\}
        \end{align*}
    \end{proposition}

    \begin{proof}
        Omitted
    \end{proof}


    \begin{proposition}[$ax$-Bound on Dimensionality]
        \begin{align*}
            \dim(P) &\leq 1 + \#\{ax\textit{-Rule} \in \,\, \vdash P\} \\
            \dim(P) &\leq 1 + \#\{vars \in P\}
        \end{align*}
    \end{proposition}

    \begin{proof}
        The first case is trivial.
        Given a formula $P$ and a constructed sequent proof $\vdash P$ through coalescence, additive stratification ensures that $\dim(P) = n \implies leaves(P) \geq n$.
        As each of the $n$ branches of the tree moves `upwards', they must either terminate at an $ax$-rule or branch, leaving $n + 1$ branches total.
        Subsequently, for each $ax$-rule forming a leaf of the tree, $\dim(P)$ increases by no more than one.
        Finally, consider a base case such as $P \seteq \vdash a \vee \neg a$ where $\dim(P) = 2 \leq 1 + \#\{ax-Rule \in \vdash P\}$ (in this case, it is equal).
        
        Second case omitted
    \end{proof}

