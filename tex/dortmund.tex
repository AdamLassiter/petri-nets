\documentclass{article}

% ~~~~~~~~~~~~~~~~~~~~~~~ Preamble ~~~~~~~~~~~~~~~~~~~~~~~

\usepackage{amsmath}
\usepackage{amssymb}
\usepackage{amsthm}
\usepackage{bpextra}
\usepackage{enumitem}
\usepackage{float}
\usepackage[margin=1.5in]{geometry}
\usepackage[hidelinks]{hyperref}
\usepackage{latexsym}
\usepackage{natbib}
\usepackage{subcaption}
\usepackage{tikz}
\usepackage{tkz-berge}
\usepackage{titlesec}
\usepackage{varwidth}

% anti-tikz things
\usetikzlibrary{external}
% \tikzexternalize

% tikz things
\usetikzlibrary{petri, topaths, positioning, decorations.pathmorphing}

% natbib source
\bibliographystyle{custom}
\renewcommand{\bibsection}{}

\pgfarrowsdeclare{arr}{arr}{
  \setlength{\arrowsize}{0.6pt}
  \addtolength{\arrowsize}{.5\pgflinewidth}
  \pgfarrowsrightextend{1\arrowsize}
  \pgfarrowsleftextend{1\arrowsize}
}{
  \setlength{\arrowsize}{0.6pt}
  \addtolength{\arrowsize}{.5\pgflinewidth}
  \pgfpathmoveto{\pgfpoint{-5\arrowsize}{4\arrowsize}}
  \pgfpathlineto{\pgfpointorigin}
  \pgfpathlineto{\pgfpoint{-5\arrowsize}{-4\arrowsize}}
  \pgfusepathqstroke
}

\newcommand{\drawsquig}{\draw[-arr,
line join=round,
decorate, decoration={
    zigzag,
    segment length=4,
    amplitude=.9,post=lineto,
    post length=2pt
}]}


\titlespacing{\section}{0pt}{12pt plus 4pt minus 2pt}{0pt plus 2pt minus 2pt}
%\titleformat{\section}
%  {\normalfont\scshape}{\thesection}{1em}{}

% no paragraph indent
\setlength{\parindent}{0em}
\setlength{\parskip}{1em}
%\setlength{\columnsep}{2em}

% delta-equals (unused)
\def\deltaeq{\mathrel{\ensurestackMath{\stackon[1pt]{=}{\scriptstyle\Delta}}}}
% define-equals
\def\defeq{::=}
% set-to-equals
\def\seteq{:=}

% scalable proof tree
\newenvironment{scprooftree}[1]%
  {\gdef\scalefactor{#1}\begin{center}\proofSkipAmount \leavevmode}%
  {\scalebox{\scalefactor}{\DisplayProof}\proofSkipAmount \end{center} }


% indented definitions, lemmas etc
\makeatletter
\newtheoremstyle{indented}
    {15pt}% space before
    {5pt}% space after
    {\addtolength{\@totalleftmargin}{0em}
     \addtolength{\linewidth}{-0em}
     \parshape 1 0em \linewidth}% body font
    {-0em}% indent
    {\bfseries}% header font
    {.}% punctuation
    {\newline}% after theorem header
    {}% header specification (empty for default)
\makeatother


% theorems with global counter
\theoremstyle{indented}
\newtheorem{sec-ctr}{???}[section]
\newtheorem{definition}[sec-ctr]{Definition}
\newtheorem*{definition*}{Definition}
\newtheorem{proposition}[sec-ctr]{Proposition}
\newtheorem*{proposition*}{Proposition}
\newtheorem{lemma}[sec-ctr]{Lemma}
\newtheorem*{lemma*}{Lemma}
\newtheorem*{example}{Example}
\newtheorem*{examples}{Examples}
\newtheorem{corollary}[sec-ctr]{Corollary}
\newtheorem*{corollary*}{Corollary}
\newtheorem{remark}[sec-ctr]{Remark}
\newtheorem*{remark*}{Remark}
\newtheorem*{remarks}{Remarks}


% ~~~~~~~~~~~~~~~~~~~~~~~~~~~~~~~~~~~~~~~~~~~~~~~~~~~~~~~~


\title{Natural Proof Search for Additive Linear Logic}
\author{Adam Lassiter\\Department of Computer Science\\University of Bath \and Willem Heijltjes\\Department of Computer Science\\University of Bath}
\date{\today}

\begin{document}

    \maketitle
    \begin{abstract}
        \textbf{
            We investigate a natural algorithm for proof search within classical logic and bounds on the complexity class of such a search.
            We further examine natural optimisations to this algorithm and how they affect complexity bounds.
        }
    \end{abstract}

    \section*{Introduction}
        Building on the work done by~\cite{petri-nets}, we investigate proof search in classical logic through additive linear logic (ALL).
        The process we investigate, called \textit{coalescence}, is a top-down proof search from axiom links down to the conclusion.
        This method is promising as it boasts great efficiency for ALL proof search and has a natural transformation to sequent calculus proofs.

    \section*{Sequent Calculus and ALL}
        A classical formula can be proved by an n-dimensional additive proof, for some n depending upon the formula.
        We propose some simple classes for formulae --- boolean constants \textit{only} are 1-dimensional, normal additive proofs are 2-dimensional amongst others etc.
        We prove that this idea is consistent through \textit{additive stratification} of the sequent calculus --- that is, any sequent proof may be `rearranged' up to order of rules applied, in particular with \textit{weakening} at the top and \textit{contraction} at the bottom.
        Then coalescence is exactly (additively stratified) proof search.

    \section*{Coalescence}
        We construct our proof search through use of `natural' deductions.
        Using a system analogous to \textit{petri nets}, we construct a net through the n-dimensional cross product of places in a formula and associated transitions across the net.
        Each token in our net begins at an axiom and transitions are exhaustively applied up the formula's syntax tree --- or down an equivalent sequent proof tree.
        A place is a coordinate in the n-dimensional grid representing a context of n places in the formula that is provable.
        The process either halts when the root of the formula syntax tree is reached and a proof is constructed, or restarts in a higher dimension when applicable transitions are exhausted and the dimensionality must be increased.
        The dimensionality of a proof is then the dimensionality of our grid when the root is reached.

    \section*{Motivation}
        Clearly complexity scales with dimensionality and our motivation is then: \textit{`What dimension is sufficient for a given formula?'}.
        In essence, this gives an upper bound for proof search.

    \section*{Some Examples}
        It turns out that this does not yield results as expected --- for example, in $(a \vee \neg a) \wedge (b \vee \neg b)$ we would expect a dimension of 2 since each component may be proved in 2 dimensions and the conjunction should be trivial.
        We instead find, unsatisfyingly, that this increases dimension quickly and unreasonably, with the aforementioned case yielding dimensionality 3 and linear growth for subsequent additional variables.

    \section*{Solution}
        To solve this issue, we then investigate liberating the search algorithm and generalising over the properties of sequents --- namely, idempotency and commutativity.
        This includes: some notion of applying conjunctions `diagonally' and switching from tuple or multiset links to set links.
        The latter takes us into more familiar/obvious proof search territory.
        
        Beyond these two points, there exist various other implementation trade-offs: dense/sparse representation, high-performance data structures and assorted other ad-hoc optimisations.

    \section*{Further Examples}
        Our optimisation through generalising over sequents yields favourable results for conjunctive normal form (CNF) but maintains poor performance for disjunctive normal form (DNF) formulae.
        We construct a simple algebra of classes, with conjunction and disjunction of proven formulae equivalent to maximum and minimum functions of left and right subformulae dimensionality.
        Finally, we hypothesise a generalised bound for any formula and provide the `essence' of the associated proof --- we expect dimensionality to be equivalent to the most number of variables in any DNF subformula.

    \section*{Ongoing Work}
        We continue to study how coalescence proof search relates to traditional proof search methods, but we expect it to be similar to either the `connections' method, see~\cite{connection-based-proof-method}, or the `matrix' method, see~\cite{tableaux-for-logic-of-proofs}.
        Progress is still to be made as to DNF formulae --- we expect generalisation over associativity of ALL terms and automatic construction of subformulae may hold the key to a more natural computation.

    \section*{References}
        \bibliography{dissertation}

\end{document}
