\section{Coalescence}
    
    \begin{definition*}[Petri Nets]
        For the purposes required here, a \textit{petri net} $\mathcal{N}$ is $(\mathcal{P, F})$ where $f \in \mathcal{F} : \mathcal{P}^m \times \mathcal{P}$.
        In particular, $\mathcal{P}$ is a set of places and $\mathcal{F}$ a set of flows or transitions.
        A \textit{configuration} is a set $\mathcal{C} \subset \mathcal{P}$ of tokens in places.

        Given a formula in classical logic, conjunction or disjunction is encoded as:
        \begin{align*}
            A_1 \vee_2 B_3    &\quad\mapsto\quad \{ (A_1) \times \vee_2, (B_3) \times \vee_2 \} \\
            A_1 \wedge_2 B_3 &\quad\mapsto\quad \{ (A_1, B_3) \times \wedge_2 \}
        \end{align*}
        where $A, B$ are (not necessarily unique) subformulae in unique places iterated over by $\{1, 2, \ldots\}$.
    \end{definition*}

    \begin{example}
        Consider the 2-d petri net representing the cross-product $a \vee \neg a \otimes \neg a \wedge a$ as follows:
        \begin{figure}[H]
    \centering
    \begin{tikzpicture}[transform shape, label distance=5mm, every node/.style={circle, fill=black!100, inner sep=0.05cm}]
        % nodes
        % top
        \node[on grid, anchor=center, label=left:\rotatebox{-90}{$\neg a$}, label=above:{$a$}](aa){};
        \node[on grid, right=1cm of aa, label=above:{$\vee$}](av){};
        \node[on grid, right=1cm of av, label=above:{$\neg a$}](ana){};
        % middle
        \node[on grid, below=1cm of aa, label=left:\rotatebox{-90}{$\wedge$}](va){};
        \node[on grid, right=1cm of va](vv){};
        \node[on grid, right=1cm of vv](vna){};
        % bottom
        \node[on grid, below=1cm of va, label=left:\rotatebox{-90}{$a$}](naa){};
        \node[on grid, right=1cm of naa](nav){};
        \node[on grid, right=1cm of nav](nana){};
        % disjunction helpers
        \coordinate[on grid, right=0.28cm of va](da);
        \coordinate[on grid, right=0.28cm of vv](dv);
        \coordinate[on grid, right=0.28cm of vna](dna);

        % edges
        % horizontal
        \draw[-arr] (aa) to [bend left] (av);
        \draw[-arr] (ana) to [bend right] (av);
        \draw[-arr] (va) to [bend left] (vv);
        \draw[-arr] (vna) to [bend right] (vv);
        \draw[-arr] (naa) to [bend left] (nav);
        \draw[-arr] (nana) to [bend right] (nav);
        % vertical
        \draw[-] (aa) to [bend left] (naa);
        \draw[-arr] (da) to (va);
        \draw[-] (av) to [bend left] (nav);
        \draw[-arr] (dv) to (vv);
        \draw[-] (ana) to [bend left] (nana);
        \draw[-arr] (dna) to (vna);
    \end{tikzpicture}
\end{figure}

    \end{example}


    \begin{definition*}[Firing Petri Nets]
        Given a petri net $\mathcal{N}$ and configuration $\mathcal{C}$, a \textit{firing} of the net $\mathcal{N}$ is a new configuration generated by application of a transition $f \in \mathcal{F}$ on $m$ tokens $c_1 \ldots c_n \in \mathcal{C}$.
        In particular:
        \begin{equation*}
            (\mathcal{N = (P, F), C}) \mapsto (\mathcal{N}, (\mathcal{C} \cup f_{right}) \setminus f_{left})
        \end{equation*}
        for some $f = (f_{left}, f_{right}) \in \mathcal{F}$
        
        For the uses required here, a variant of firing is used instead called \textit{spawning}.
        This generates new configurations in the same manner as firing, with one key difference:
        \begin{equation*}
            (\mathcal{N = (P, F), C}) \mapsto (\mathcal{N}, \mathcal{C} \cup f_{right})
        \end{equation*}
        for some $f = (f_{left}, f_{right}) \in \mathcal{F}$.
        That is, when a transition $f$ is performed on tokens $x_1 \ldots _n$, these tokens remain present in the configuration in addition to the new token $f(x_1 \ldots x_n)$.
        
        A petri net is said to be \textit{exhaustively fired} if it is fired until there does not exist any such $f \in \mathcal{F}$ to fire.
    \end{definition*}

    \begin{example}
        Consider the 2d petri net representing $a \vee \neg a \otimes \neg a \wedge a$ with tokens at $\{(a, \neg a), (\neg a, a)\}$.
        \begin{figure}[H]
    \centering
    \begin{subfigure}{0.3\linewidth}
        \begin{tikzpicture}[transform shape, label distance=5mm, every path/.style={color=black!25}, every node/.style={circle, fill=black!25, inner sep=0.05cm}, every label/.append style={color=black!100}]
            % nodes
            % top
            \node[anchor=center, label=left:\rotatebox{-90}{$\neg a$}, label=above:{$a$}, style={color=black!100}](aa){};
            \node[right=1cm of aa, label=above:{$\vee$}](av){};
            \node[right=1cm of av, label=above:{$\neg a$}](ana){};
            % middle
            \node[below=1cm of aa, label=left:\rotatebox{-90}{$\wedge$}](va){};
            \node[right=1cm of va](vv){};
            \node[right=1cm of vv](vna){};
            % bottom
            \node[below=1cm of va, label=left:\rotatebox{-90}{$a$}](naa){};
            \node[right=1cm of naa](nav){};
            \node[right=1cm of nav, style={color=black!100}](nana){};
            % disjunction helpers
            \coordinate[right=0.28cm of va](da);
            \coordinate[right=0.28cm of vv](dv);
            \coordinate[right=0.28cm of vna](dna);

            % edges
            % horizontal
            \draw[-arr] (aa) to [bend left] (av);
            \draw[-arr] (ana) to [bend right] (av);
            \draw[-arr] (va) to [bend left] (vv);
            \draw[-arr] (vna) to [bend right] (vv);
            \draw[-arr] (naa) to [bend left] (nav);
            \draw[-arr] (nana) to [bend right] (nav);
            % vertical
            \draw[-] (aa) to [bend left] (naa);
            \draw[-arr] (da) to (va);
            \draw[-] (av) to [bend left] (nav);
            \draw[-arr] (dv) to (vv);
            \draw[-] (ana) to [bend left] (nana);
            \draw[-arr] (dna) to (vna);
        \end{tikzpicture}
    \end{subfigure}
    $\leadsto$
    \begin{subfigure}{0.3\linewidth}
        \begin{tikzpicture}[transform shape, label distance=5mm, every path/.style={color=black!25}, every node/.style={circle, fill=black!25, inner sep=0.05cm}, every label/.append style={color=black!100}]
            % nodes
            % top
            \node[anchor=center, label=left:\rotatebox{-90}{$\neg a$}, label=above:{$a$}, style={color=black!100}](aa){};
            \node[right=1cm of aa, label=above:{$\vee$}, style={color=black!100}](av){};
            \node[right=1cm of av, label=above:{$\neg a$}](ana){};
            % middle
            \node[below=1cm of aa, label=left:\rotatebox{-90}{$\wedge$}](va){};
            \node[right=1cm of va](vv){};
            \node[right=1cm of vv](vna){};
            % bottom
            \node[below=1cm of va, label=left:\rotatebox{-90}{$a$}](naa){};
            \node[right=1cm of naa, style={color=black!100}](nav){};
            \node[right=1cm of nav, style={color=black!100}](nana){};
            % disjunction helpers
            \coordinate[right=0.28cm of va](da);
            \coordinate[right=0.28cm of vv](dv);
            \coordinate[right=0.28cm of vna](dna);

            % edges
            % horizontal
            \draw[-arr, style={color=black!100}] (aa) to [bend left] (av);
            \draw[-arr] (ana) to [bend right] (av);
            \draw[-arr] (va) to [bend left] (vv);
            \draw[-arr] (vna) to [bend right] (vv);
            \draw[-arr] (naa) to [bend left] (nav);
            \draw[-arr, style={color=black!100}] (nana) to [bend right] (nav);
            % vertical
            \draw[-] (aa) to [bend left] (naa);
            \draw[-arr] (da) to (va);
            \draw[-] (av) to [bend left] (nav);
            \draw[-arr] (dv) to (vv);
            \draw[-] (ana) to [bend left] (nana);
            \draw[-arr] (dna) to (vna);
        \end{tikzpicture}
    \end{subfigure}
    $\leadsto$
    \begin{subfigure}{0.3\linewidth}
        \begin{tikzpicture}[transform shape, label distance=5mm, every path/.style={color=black!25}, every node/.style={circle, fill=black!25, inner sep=0.05cm}, every label/.append style={color=black!100}]
            % nodes
            % top
            \node[anchor=center, label=left:\rotatebox{-90}{$\neg a$}, label=above:{$a$}, style={color=black!100}](aa){};
            \node[right=1cm of aa, label=above:{$\vee$}, style={color=black!100}](av){};
            \node[right=1cm of av, label=above:{$\neg a$}](ana){};
            % middle
            \node[below=1cm of aa, label=left:\rotatebox{-90}{$\wedge$}](va){};
            \node[right=1cm of va, style={color=black!100}](vv){};
            \node[right=1cm of vv](vna){};
            % bottom
            \node[below=1cm of va, label=left:\rotatebox{-90}{$a$}](naa){};
            \node[right=1cm of naa, style={color=black!100}](nav){};
            \node[right=1cm of nav, style={color=black!100}](nana){};
            % disjunction helpers
            \coordinate[right=0.28cm of va](da);
            \coordinate[right=0.28cm of vv](dv);
            \coordinate[right=0.28cm of vna](dna);

            % edges
            % horizontal
            \draw[-arr, style={color=black!100}] (aa) to [bend left] (av);
            \draw[-arr] (ana) to [bend right] (av);
            \draw[-arr] (va) to [bend left] (vv);
            \draw[-arr] (vna) to [bend right] (vv);
            \draw[-arr] (naa) to [bend left] (nav);
            \draw[-arr, style={color=black!100}] (nana) to [bend right] (nav);
            % vertical
            \draw[-] (aa) to [bend left] (naa);
            \draw[-arr] (da) to (va);
            \draw[-, style={color=black!100}] (av) to [bend left] (nav);
            \draw[-arr, style={color=black!100}] (dv) to (vv);
            \draw[-] (ana) to [bend left] (nana);
            \draw[-arr] (dna) to (vna);
        \end{tikzpicture}
    \end{subfigure}
\end{figure}


        Note in the final step the pair of tokens required to perform the conjunction transition.
    \end{example}


    \begin{definition*}[Coalescence]
        Given a formula $P$, the coalescence algorithm is as follows:
        \begin{enumerate}[nosep]
            \item Set  $n \seteq 1$
            \item Construct a $n$-dimensional petri net $\mathcal{N}$ as $\bigotimes_n P$ where each subformula represents a place and each conjunction and disjunction a flow
            \item Construct a configuration $C$ with a token at each place $p = (\ldots, a, \ldots, \neg a, \ldots)$ the intersection of a pair of tautological atoms or $(\ldots, \top, \ldots)$ an instance of \textit{top}
            \item Exhaustively fire the petri net $\mathcal{N}$ using the \textit{spawning} method
            \item If there exists a token in the configuration $\mathcal{C^*}$ at the root of the formula $P$, halt and return $n$
            \item Otherwise, increment $n \seteq n + 1$ and go to step 2
        \end{enumerate}
    \end{definition*}
    
    \begin{example}
        Consider the petri net proof for the formula $(a \vee b) \vee \neg a$, with a solution for $n = 2$.
        The net is initialised with tokens in all places satisfying either the $ax$-rule or $\top$-rule and fired exhaustively:
        \begin{figure}[H]
    \def\x{0.8cm}
    \centering
    \begin{subfigure}{0.4\linewidth}
        \begin{tikzpicture}[transform shape, label distance=5mm, every path/.style={color=black!25}, every node/.style={circle, fill=black!25, inner sep=0.05cm}, every label/.append style={color=black!100}]
            % nodes
            % a
            \node[on grid, anchor=center, label=left:\rotatebox{-90}{$(a$}, label=above:{$(a$}](aa){};
            \node[on grid, right=\x of aa, label=above:{$\vee$}](av){};
            \node[on grid, right=\x of av, label=above:{$b)$}](ab){};
            \node[on grid, right=\x of ab, label=above:{$\vee$}](aw){};
            \node[on grid, right=\x of aw, style={color=black!100}, label=above:{$\neg a$}](ae){};
            % v
            \node[on grid, below=\x of aa, label=left:\rotatebox{-90}{$\vee$}](va){};
            \node[on grid, right=\x of va](vv){};
            \node[on grid, right=\x of vv](vb){};
            \node[on grid, right=\x of vb](vw){};
            \node[on grid, right=\x of vw](ve){};
            % b
            \node[on grid, below=\x of va, label=left:\rotatebox{-90}{$b)$}](ba){};
            \node[on grid, right=\x of ba](bv){};
            \node[on grid, right=\x of bv](bb){};
            \node[on grid, right=\x of bb](bw){};
            \node[on grid, right=\x of bw](be){};
            % v
            \node[on grid, below=\x of ba, label=left:\rotatebox{-90}{$\vee$}](wa){};
            \node[on grid, right=\x of wa](wv){};
            \node[on grid, right=\x of wv](wb){};
            \node[on grid, right=\x of wb](ww){};
            \node[on grid, right=\x of ww](we){};
            % ~a
            \node[on grid, below=\x of wa, style={color=black!100}, label=left:\rotatebox{-90}{$\neg a$}](ea){};
            \node[on grid, right=\x of ea](ev){};
            \node[on grid, right=\x of ev](eb){};
            \node[on grid, right=\x of eb](ew){};
            \node[on grid, right=\x of ew](ee){};

            % edges
            % horizontal
            % a
            \draw[-arr] (aa) to [bend left]  (av);
            \draw[-arr] (ab) to [bend right] (av);
            \draw[-arr] (av) to [bend right] (aw);
            \draw[-arr] (ae) to [bend right] (aw);
            % v
            \draw[-arr] (va) to [bend left]  (vv);
            \draw[-arr] (vb) to [bend right] (vv);
            \draw[-arr] (vv) to [bend right] (vw);
            \draw[-arr] (ve) to [bend right] (vw);
            % b
            \draw[-arr] (ba) to [bend left]  (bv);
            \draw[-arr] (bb) to [bend right] (bv);
            \draw[-arr] (bv) to [bend right] (bw);
            \draw[-arr] (be) to [bend right] (bw);
            % v
            \draw[-arr] (wa) to [bend left]  (wv);
            \draw[-arr] (wb) to [bend right] (wv);
            \draw[-arr] (wv) to [bend right] (ww);
            \draw[-arr] (we) to [bend right] (ww);
            % ~a
            \draw[-arr] (ea) to [bend left]  (ev);
            \draw[-arr] (eb) to [bend right] (ev);
            \draw[-arr] (ev) to [bend right] (ew);
            \draw[-arr] (ee) to [bend right] (ew);
            
            %vertical
            % a
            \draw[-arr] (aa) to [bend left]  (va);
            \draw[-arr] (ba) to [bend right] (va);
            \draw[-arr] (va) to [bend right] (wa);
            \draw[-arr] (ea) to [bend right] (wa);
            % v
            \draw[-arr] (av) to [bend left]  (vv);
            \draw[-arr] (bv) to [bend right] (vv);
            \draw[-arr] (vv) to [bend right] (wv);
            \draw[-arr] (ev) to [bend right] (wv);
            % b
            \draw[-arr] (ab) to [bend left]  (vb);
            \draw[-arr] (bb) to [bend right] (vb);
            \draw[-arr] (vb) to [bend right] (wb);
            \draw[-arr] (eb) to [bend right] (wb);
            % v
            \draw[-arr] (aw) to [bend left]  (vw);
            \draw[-arr] (bw) to [bend right] (vw);
            \draw[-arr] (vw) to [bend right] (ww);
            \draw[-arr] (ew) to [bend right] (ww);
            % ~a
            \draw[-arr] (ae) to [bend left]  (ve);
            \draw[-arr] (be) to [bend right] (ve);
            \draw[-arr] (ve) to [bend right] (we);
            \draw[-arr] (ee) to [bend right] (we);
        \end{tikzpicture}
    \end{subfigure}
    \begin{subfigure}{0.1\linewidth}
        $\leadsto$
    \end{subfigure}
    \begin{subfigure}{0.4\linewidth}
        \begin{tikzpicture}[transform shape, label distance=5mm, every path/.style={color=black!25}, every node/.style={circle, fill=black!25, inner sep=0.05cm}, every label/.append style={color=black!100}]
            % nodes
            % a
            \node[on grid, anchor=center, label=left:\rotatebox{-90}{$(a$}, label=above:{$(a$}](aa){};
            \node[on grid, right=\x of aa, label=above:{$\vee$}](av){};
            \node[on grid, right=\x of av, label=above:{$b)$}](ab){};
            \node[on grid, right=\x of ab, style={color=black!100}, label=above:{$\vee$}](aw){};
            \node[on grid, right=\x of aw, style={color=black!100}, label=above:{$\neg a$}](ae){};
            % v
            \node[on grid, below=\x of aa, label=left:\rotatebox{-90}{$\vee$}](va){};
            \node[on grid, right=\x of va](vv){};
            \node[on grid, right=\x of vv](vb){};
            \node[on grid, right=\x of vb, style={color=black!100}](vw){};
            \node[on grid, right=\x of vw](ve){};
            % b
            \node[on grid, below=\x of va, label=left:\rotatebox{-90}{$b)$}](ba){};
            \node[on grid, right=\x of ba](bv){};
            \node[on grid, right=\x of bv](bb){};
            \node[on grid, right=\x of bb](bw){};
            \node[on grid, right=\x of bw](be){};
            % v
            \node[on grid, below=\x of ba, style={color=black!100}, label=left:\rotatebox{-90}{$\vee$}](wa){};
            \node[on grid, right=\x of wa, style={color=black!100}](wv){};
            \node[on grid, right=\x of wv](wb){};
            \node[on grid, right=\x of wb, style={color=black!100}, style={color=black!100}](ww){};
            \node[on grid, right=\x of ww](we){};
            % ~a
            \node[on grid, below=\x of wa, style={color=black!100}, label=left:\rotatebox{-90}{$\neg a$}](ea){};
            \node[on grid, right=\x of ea](ev){};
            \node[on grid, right=\x of ev](eb){};
            \node[on grid, right=\x of eb](ew){};
            \node[on grid, right=\x of ew](ee){};

            % edges
            % horizontal
            % a
            \draw[-arr] (aa) to [bend left]  (av);
            \draw[-arr] (ab) to [bend right] (av);
            \draw[-arr] (av) to [bend right] (aw);
            \draw[-arr, style={color=black!100}] (ae) to [bend right] (aw);
            % v
            \draw[-arr] (va) to [bend left]  (vv);
            \draw[-arr] (vb) to [bend right] (vv);
            \draw[-arr] (vv) to [bend right] (vw);
            \draw[-arr] (ve) to [bend right] (vw);
            % b
            \draw[-arr] (ba) to [bend left]  (bv);
            \draw[-arr] (bb) to [bend right] (bv);
            \draw[-arr] (bv) to [bend right] (bw);
            \draw[-arr] (be) to [bend right] (bw);
            % v
            \draw[-arr, style={color=black!100}] (wa) to [bend left]  (wv);
            \draw[-arr] (wb) to [bend right] (wv);
            \draw[-arr, style={color=black!100}] (wv) to [bend right] (ww);
            \draw[-arr] (we) to [bend right] (ww);
            % ~a
            \draw[-arr] (ea) to [bend left]  (ev);
            \draw[-arr] (eb) to [bend right] (ev);
            \draw[-arr] (ev) to [bend right] (ew);
            \draw[-arr] (ee) to [bend right] (ew);
            
            %vertical
            % a
            \draw[-arr] (aa) to [bend left]  (va);
            \draw[-arr] (ba) to [bend right] (va);
            \draw[-arr] (va) to [bend right] (wa);
            \draw[-arr, style={color=black!100}] (ea) to [bend right] (wa);
            % v
            \draw[-arr] (av) to [bend left]  (vv);
            \draw[-arr] (bv) to [bend right] (vv);
            \draw[-arr] (vv) to [bend right] (wv);
            \draw[-arr] (ev) to [bend right] (wv);
            % b
            \draw[-arr] (ab) to [bend left]  (vb);
            \draw[-arr] (bb) to [bend right] (vb);
            \draw[-arr] (vb) to [bend right] (wb);
            \draw[-arr] (eb) to [bend right] (wb);
            % v
            \draw[-arr, style={color=black!100}] (aw) to [bend left]  (vw);
            \draw[-arr] (bw) to [bend right] (vw);
            \draw[-arr, style={color=black!100}] (vw) to [bend right] (ww);
            \draw[-arr] (ew) to [bend right] (ww);
            % ~a
            \draw[-arr] (ae) to [bend left]  (ve);
            \draw[-arr] (be) to [bend right] (ve);
            \draw[-arr] (ve) to [bend right] (we);
            \draw[-arr] (ee) to [bend right] (we);
        \end{tikzpicture}
    \end{subfigure}
\end{figure}


    \end{example}

    \begin{remark*}
        Note the symmetry along the upper-left lower-right diagonal.
        In fact, this symmetry can be exploited in the general case, ignoring all tokens below (w.l.o.g.) the diagonal as they simply express the commutativity property $\vdash A, B \iff \vdash B, A$.
        If this is the case, the coalescence algorithm may halt early if the root of the formula is reached.
    \end{remark*}

    \begin{example}
        Consider the petri net proof for the formula $(a \vee \neg a) \wedge (b \vee \neg b)$, with a solution for $n = 3$.
        In an exhaustively fired net for $n = 2$, the proof looks as follows:
        \begin{figure}[H]
    \centering
    \begin{subfigure}{0.4\linewidth}
        \begin{tikzpicture}
            \begin{scope}[transform shape, every path/.style={color=black!25}, every node/.style={circle, fill=black!25, inner sep=0.05cm}, every label/.append style={color=black!100}]
                % nodes
                % a
                \node[anchor=center    ](aa){};
                \node[right=0.5cm of aa](av){};
                \node[right=0.5cm of av, style={color=black!100}](ae){};
                \node[right=0.5cm of ae](an){};
                \node[right=0.5cm of an](ab){};
                \node[right=0.5cm of ab](aw){};
                \node[right=0.5cm of aw](ap){};
                % v
                \node[below=0.5cm of aa](va){};
                \node[right=0.5cm of va](vv){};
                \node[right=0.5cm of vv](ve){};
                \node[right=0.5cm of ve](vn){};
                \node[right=0.5cm of vn](vb){};
                \node[right=0.5cm of vb](vw){};
                \node[right=0.5cm of vw](vp){};
                % ~a
                \node[below=0.5cm of va, style={color=black!100}](ea){};
                \node[right=0.5cm of ea](ev){};
                \node[right=0.5cm of ev](ee){};
                \node[right=0.5cm of ee](en){};
                \node[right=0.5cm of en](eb){};
                \node[right=0.5cm of eb](ew){};
                \node[right=0.5cm of ew](ep){};
                % ^
                \node[below=0.5cm of ea](na){};
                \node[right=0.5cm of na](nv){};
                \node[right=0.5cm of nv](ne){};
                \node[right=0.5cm of ne](nn){};
                \node[right=0.5cm of nn](nb){};
                \node[right=0.5cm of nb](nw){};
                \node[right=0.5cm of nw](np){};
                % b
                \node[below=0.5cm of na](ba){};
                \node[right=0.5cm of ba](bv){};
                \node[right=0.5cm of bv](be){};
                \node[right=0.5cm of be](bn){};
                \node[right=0.5cm of bn](bb){};
                \node[right=0.5cm of bb](bw){};
                \node[right=0.5cm of bw, style={color=black!100}](bp){};
                % v
                \node[below=0.5cm of ba](wa){};
                \node[right=0.5cm of wa](wv){};
                \node[right=0.5cm of wv](we){};
                \node[right=0.5cm of we](wn){};
                \node[right=0.5cm of wn](wb){};
                \node[right=0.5cm of wb](ww){};
                \node[right=0.5cm of ww](wp){};
                % ~b
                \node[below=0.5cm of wa](pa){};
                \node[right=0.5cm of pa](pv){};
                \node[right=0.5cm of pv](pe){};
                \node[right=0.5cm of pe](pn){};
                \node[right=0.5cm of pn, style={color=black!100}](pb){};
                \node[right=0.5cm of pb](pw){};
                \node[right=0.5cm of pw](pp){};

                % disjunction helpers
                % horizontal
                \coordinate[below=0.2cm of an](ad);
                \coordinate[below=0.2cm of vn](vd);
                \coordinate[below=0.2cm of en](ed);
                \coordinate[below=0.2cm of nn](nd);
                \coordinate[below=0.2cm of bn](bd);
                \coordinate[below=0.2cm of wn](wd);
                \coordinate[below=0.2cm of pn](pd);
                % vertical
                \coordinate[left=0.2cm of na](da);
                \coordinate[left=0.2cm of nv](dv);
                \coordinate[left=0.2cm of ne](de);
                \coordinate[left=0.2cm of nn](dn);
                \coordinate[left=0.2cm of nb](db);
                \coordinate[left=0.2cm of nw](dw);
                \coordinate[left=0.2cm of np](dp);


                % edges
                % horizontal
                % a
                \draw[-arr] (aa) to [bend left]     (av);
                \draw[-arr] (ae) to [bend right]    (av);
                \draw[-arr] (ab) to [bend left]     (aw);
                \draw[-arr] (ap) to [bend right]    (aw);
                \draw[-]    (av) to [bend right=20] (aw);
                \draw[-arr] (ad) to                 (an);
                % v
                \draw[-arr] (va) to [bend left]     (vv);
                \draw[-arr] (ve) to [bend right]    (vv);
                \draw[-arr] (vb) to [bend left]     (vw);
                \draw[-arr] (vp) to [bend right]    (vw);
                \draw[-]    (vv) to [bend right=20] (vw);
                \draw[-arr] (vd) to                 (vn);
                % ~a
                \draw[-arr] (ea) to [bend left]     (ev);
                \draw[-arr] (ee) to [bend right]    (ev);
                \draw[-arr] (eb) to [bend left]     (ew);
                \draw[-arr] (ep) to [bend right]    (ew);
                \draw[-]    (ev) to [bend right=20] (ew);
                \draw[-arr] (ed) to                 (en);
                % ^
                \draw[-arr] (na) to [bend left]     (nv);
                \draw[-arr] (ne) to [bend right]    (nv);
                \draw[-arr] (nb) to [bend left]     (nw);
                \draw[-arr] (np) to [bend right]    (nw);
                \draw[-]    (nv) to [bend right=20] (nw);
                \draw[-arr] (nd) to                 (nn);
                % b
                \draw[-arr] (ba) to [bend left]     (bv);
                \draw[-arr] (be) to [bend right]    (bv);
                \draw[-arr] (bb) to [bend left]     (bw);
                \draw[-arr] (bp) to [bend right]    (bw);
                \draw[-]    (bv) to [bend right=20] (bw);
                \draw[-arr] (bd) to                 (bn);
                % v
                \draw[-arr] (wa) to [bend left]     (wv);
                \draw[-arr] (we) to [bend right]    (wv);
                \draw[-arr] (wb) to [bend left]     (ww);
                \draw[-arr] (wp) to [bend right]    (ww);
                \draw[-]    (wv) to [bend right=20] (ww);
                \draw[-arr] (wd) to                 (wn);
                % ~b
                \draw[-arr] (pa) to [bend left]     (pv);
                \draw[-arr] (pe) to [bend right]    (pv);
                \draw[-arr] (pb) to [bend left]     (pw);
                \draw[-arr] (pp) to [bend right]    (pw);
                \draw[-]    (pv) to [bend right=20] (pw);
                \draw[-arr] (pd) to                 (pn);
                % vertical
                % a
                \draw[-arr] (aa) to [bend left]     (va);
                \draw[-arr] (ea) to [bend right]    (va);
                \draw[-arr] (ba) to [bend left]     (wa);
                \draw[-arr] (pa) to [bend right]    (wa);
                \draw[-]    (va) to [bend right=20] (wa);
                \draw[-arr] (da) to                 (na);
                % v                                    
                \draw[-arr] (av) to [bend left]     (vv);
                \draw[-arr] (ev) to [bend right]    (vv);
                \draw[-arr] (bv) to [bend left]     (wv);
                \draw[-arr] (pv) to [bend right]    (wv);
                \draw[-]    (vv) to [bend right=20] (wv);
                \draw[-arr] (dv) to                 (nv);
                % ~a                                   
                \draw[-arr] (ae) to [bend left]     (ve);
                \draw[-arr] (ee) to [bend right]    (ve);
                \draw[-arr] (be) to [bend left]     (we);
                \draw[-arr] (pe) to [bend right]    (we);
                \draw[-]    (ve) to [bend right=20] (we);
                \draw[-arr] (de) to                 (ne);
                % ^                                    
                \draw[-arr] (an) to [bend left]     (vn);
                \draw[-arr] (en) to [bend right]    (vn);
                \draw[-arr] (bn) to [bend left]     (wn);
                \draw[-arr] (pn) to [bend right]    (wn);
                \draw[-]    (vn) to [bend right=20] (wn);
                \draw[-arr] (dn) to                 (nn);
                % b                                    
                \draw[-arr] (ab) to [bend left]     (vb);
                \draw[-arr] (eb) to [bend right]    (vb);
                \draw[-arr] (bb) to [bend left]     (wb);
                \draw[-arr] (pb) to [bend right]    (wb);
                \draw[-]    (vb) to [bend right=20] (wb);
                \draw[-arr] (db) to                 (nb);
                % v                                    
                \draw[-arr] (aw) to [bend left]     (vw);
                \draw[-arr] (ew) to [bend right]    (vw);
                \draw[-arr] (bw) to [bend left]     (ww);
                \draw[-arr] (pw) to [bend right]    (ww);
                \draw[-]    (vw) to [bend right=20] (ww);
                \draw[-arr] (dw) to                 (nw);
                % ~b                                   
                \draw[-arr] (ap) to [bend left]     (vp);
                \draw[-arr] (ep) to [bend right]    (vp);
                \draw[-arr] (bp) to [bend left]     (wp);
                \draw[-arr] (pp) to [bend right]    (wp);
                \draw[-]    (vp) to [bend right=20] (wp);
                \draw[-arr] (dp) to                 (np);
            \end{scope}

            % labels
            % horizontal
            \node[above=0.5cm of aa](la){\small $(a$};
            \node[above=0.5cm of av](lv){\small $\vee$};
            \node[above=0.5cm of ae](le){\small $\neg a)$};
            \node[above=0.5cm of an](ln){\small $\wedge$};
            \node[above=0.5cm of ab](lb){\small $(b$};
            \node[above=0.5cm of aw](lw){\small $\vee$};
            \node[above=0.5cm of ap](lp){\small $\neg b)$};
            % vertical
            \node[left=0.5cm of aa](al){\rotatebox{-90}{\small $(a$}};
            \node[left=0.5cm of va](vl){\rotatebox{-90}{\small $\vee$}};
            \node[left=0.5cm of ea](el){\rotatebox{-90}{\small $\neg a)$}};
            \node[left=0.5cm of na](nl){\rotatebox{-90}{\small $\wedge$}};
            \node[left=0.5cm of ba](bl){\rotatebox{-90}{\small $(b$}};
            \node[left=0.5cm of wa](wl){\rotatebox{-90}{\small $\vee$}};
            \node[left=0.5cm of pa](pl){\rotatebox{-90}{\small $\neg b)$}};

        \end{tikzpicture}
    \end{subfigure}
    \begin{subfigure}{0.1\linewidth}
        $\leadsto$
    \end{subfigure}
    \begin{subfigure}{0.4\linewidth}
        \begin{tikzpicture}
            \begin{scope}[transform shape, label distance=5mm, every path/.style={color=black!25}, every node/.style={circle, fill=black!25, inner sep=0.05cm}, every label/.append style={color=black!100}]
                % nodes
                % a
                \node[anchor=center    ](aa){};
                \node[right=0.5cm of aa, style={color=black!100}](av){};
                \node[right=0.5cm of av, style={color=black!100}](ae){};
                \node[right=0.5cm of ae](an){};
                \node[right=0.5cm of an](ab){};
                \node[right=0.5cm of ab](aw){};
                \node[right=0.5cm of aw](ap){};
                % v
                \node[below=0.5cm of aa, style={color=black!100}](va){};
                \node[right=0.5cm of va, style={color=black!100}](vv){};
                \node[right=0.5cm of vv, style={color=black!100}](ve){};
                \node[right=0.5cm of ve](vn){};
                \node[right=0.5cm of vn](vb){};
                \node[right=0.5cm of vb](vw){};
                \node[right=0.5cm of vw](vp){};
                % ~a
                \node[below=0.5cm of va, style={color=black!100}](ea){};
                \node[right=0.5cm of ea, style={color=black!100}](ev){};
                \node[right=0.5cm of ev](ee){};
                \node[right=0.5cm of ee](en){};
                \node[right=0.5cm of en](eb){};
                \node[right=0.5cm of eb](ew){};
                \node[right=0.5cm of ew](ep){};
                % ^
                \node[below=0.5cm of ea](na){};
                \node[right=0.5cm of na](nv){};
                \node[right=0.5cm of nv](ne){};
                \node[right=0.5cm of ne](nn){};
                \node[right=0.5cm of nn](nb){};
                \node[right=0.5cm of nb](nw){};
                \node[right=0.5cm of nw](np){};
                % b
                \node[below=0.5cm of na](ba){};
                \node[right=0.5cm of ba](bv){};
                \node[right=0.5cm of bv](be){};
                \node[right=0.5cm of be](bn){};
                \node[right=0.5cm of bn](bb){};
                \node[right=0.5cm of bb, style={color=black!100}](bw){};
                \node[right=0.5cm of bw, style={color=black!100}](bp){};
                % v
                \node[below=0.5cm of ba](wa){};
                \node[right=0.5cm of wa](wv){};
                \node[right=0.5cm of wv](we){};
                \node[right=0.5cm of we](wn){};
                \node[right=0.5cm of wn, style={color=black!100}](wb){};
                \node[right=0.5cm of wb, style={color=black!100}](ww){};
                \node[right=0.5cm of ww, style={color=black!100}](wp){};
                % ~b
                \node[below=0.5cm of wa](pa){};
                \node[right=0.5cm of pa](pv){};
                \node[right=0.5cm of pv](pe){};
                \node[right=0.5cm of pe](pn){};
                \node[right=0.5cm of pn, style={color=black!100}](pb){};
                \node[right=0.5cm of pb, style={color=black!100}](pw){};
                \node[right=0.5cm of pw](pp){};

                % disjunction helpers
                % horizontal
                \coordinate[below=0.2cm of an](ad);
                \coordinate[below=0.2cm of vn](vd);
                \coordinate[below=0.2cm of en](ed);
                \coordinate[below=0.2cm of nn](nd);
                \coordinate[below=0.2cm of bn](bd);
                \coordinate[below=0.2cm of wn](wd);
                \coordinate[below=0.2cm of pn](pd);
                % vertical
                \coordinate[left=0.2cm of na](da);
                \coordinate[left=0.2cm of nv](dv);
                \coordinate[left=0.2cm of ne](de);
                \coordinate[left=0.2cm of nn](dn);
                \coordinate[left=0.2cm of nb](db);
                \coordinate[left=0.2cm of nw](dw);
                \coordinate[left=0.2cm of np](dp);

                % edges
                % horizontal
                % a
                \draw[-arr] (aa) to [bend left]     (av);
                \draw[-arr, style={color=black!100}] (ae) to [bend right]    (av);
                \draw[-arr] (ab) to [bend left]     (aw);
                \draw[-arr] (ap) to [bend right]    (aw);
                \draw[-]    (av) to [bend right=20] (aw);
                \draw[-arr] (ad) to                 (an);
                % v
                \draw[-arr, style={color=black!100}] (va) to [bend left]     (vv);
                \draw[-arr, style={color=black!100}] (ve) to [bend right]    (vv);
                \draw[-arr] (vb) to [bend left]     (vw);
                \draw[-arr] (vp) to [bend right]    (vw);
                \draw[-]    (vv) to [bend right=20] (vw);
                \draw[-arr] (vd) to                 (vn);
                % ~a
                \draw[-arr, style={color=black!100}] (ea) to [bend left]     (ev);
                \draw[-arr] (ee) to [bend right]    (ev);
                \draw[-arr] (eb) to [bend left]     (ew);
                \draw[-arr] (ep) to [bend right]    (ew);
                \draw[-]    (ev) to [bend right=20] (ew);
                \draw[-arr] (ed) to                 (en);
                % ^
                \draw[-arr] (na) to [bend left]     (nv);
                \draw[-arr] (ne) to [bend right]    (nv);
                \draw[-arr] (nb) to [bend left]     (nw);
                \draw[-arr] (np) to [bend right]    (nw);
                \draw[-]    (nv) to [bend right=20] (nw);
                \draw[-arr] (nd) to                 (nn);
                % b
                \draw[-arr] (ba) to [bend left]     (bv);
                \draw[-arr] (be) to [bend right]    (bv);
                \draw[-arr] (bb) to [bend left]     (bw);
                \draw[-arr, style={color=black!100}] (bp) to [bend right]    (bw);
                \draw[-]    (bv) to [bend right=20] (bw);
                \draw[-arr] (bd) to                 (bn);
                % v
                \draw[-arr] (wa) to [bend left]     (wv);
                \draw[-arr] (we) to [bend right]    (wv);
                \draw[-arr, style={color=black!100}] (wb) to [bend left]     (ww);
                \draw[-arr, style={color=black!100}] (wp) to [bend right]    (ww);
                \draw[-]    (wv) to [bend right=20] (ww);
                \draw[-arr] (wd) to                 (wn);
                % ~b
                \draw[-arr] (pa) to [bend left]     (pv);
                \draw[-arr] (pe) to [bend right]    (pv);
                \draw[-arr, style={color=black!100}] (pb) to [bend left]     (pw);
                \draw[-arr] (pp) to [bend right]    (pw);
                \draw[-]    (pv) to [bend right=20] (pw);
                \draw[-arr] (pd) to                 (pn);
                % vertical
                % a
                \draw[-arr] (aa) to [bend left]     (va);
                \draw[-arr, style={color=black!100}] (ea) to [bend right]    (va);
                \draw[-arr] (ba) to [bend left]     (wa);
                \draw[-arr] (pa) to [bend right]    (wa);
                \draw[-]    (va) to [bend right=20] (wa);
                \draw[-arr] (da) to                 (na);
                % v                                    
                \draw[-arr, style={color=black!100}] (av) to [bend left]     (vv);
                \draw[-arr, style={color=black!100}] (ev) to [bend right]    (vv);
                \draw[-arr] (bv) to [bend left]     (wv);
                \draw[-arr] (pv) to [bend right]    (wv);
                \draw[-]    (vv) to [bend right=20] (wv);
                \draw[-arr] (dv) to                 (nv);
                % ~a                                   
                \draw[-arr, style={color=black!100}] (ae) to [bend left]     (ve);
                \draw[-arr] (ee) to [bend right]    (ve);
                \draw[-arr] (be) to [bend left]     (we);
                \draw[-arr] (pe) to [bend right]    (we);
                \draw[-]    (ve) to [bend right=20] (we);
                \draw[-arr] (de) to                 (ne);
                % ^                                    
                \draw[-arr] (an) to [bend left]     (vn);
                \draw[-arr] (en) to [bend right]    (vn);
                \draw[-arr] (bn) to [bend left]     (wn);
                \draw[-arr] (pn) to [bend right]    (wn);
                \draw[-]    (vn) to [bend right=20] (wn);
                \draw[-arr] (dn) to                 (nn);
                % b                                    
                \draw[-arr] (ab) to [bend left]     (vb);
                \draw[-arr] (eb) to [bend right]    (vb);
                \draw[-arr] (bb) to [bend left]     (wb);
                \draw[-arr, style={color=black!100}] (pb) to [bend right]    (wb);
                \draw[-]    (vb) to [bend right=20] (wb);
                \draw[-arr] (db) to                 (nb);
                % v                                    
                \draw[-arr] (aw) to [bend left]     (vw);
                \draw[-arr] (ew) to [bend right]    (vw);
                \draw[-arr, style={color=black!100}] (bw) to [bend left]     (ww);
                \draw[-arr, style={color=black!100}] (pw) to [bend right]    (ww);
                \draw[-]    (vw) to [bend right=20] (ww);
                \draw[-arr] (dw) to                 (nw);
                % ~b                                   
                \draw[-arr] (ap) to [bend left]     (vp);
                \draw[-arr] (ep) to [bend right]    (vp);
                \draw[-arr, style={color=black!100}] (bp) to [bend left]     (wp);
                \draw[-arr] (pp) to [bend right]    (wp);
                \draw[-]    (vp) to [bend right=20] (wp);
                \draw[-arr] (dp) to                 (np);
            \end{scope}

            % labels
            % horizontal
            \node[above=0.5cm of aa](la){\small $(a$};
            \node[above=0.5cm of av](lv){\small $\vee$};
            \node[above=0.5cm of ae](le){\small $\neg a)$};
            \node[above=0.5cm of an](ln){\small $\wedge$};
            \node[above=0.5cm of ab](lb){\small $(b$};
            \node[above=0.5cm of aw](lw){\small $\vee$};
            \node[above=0.5cm of ap](lp){\small $\neg b)$};
            % vertical
            \node[left=0.5cm of aa](al){\rotatebox{-90}{\small $(a$}};
            \node[left=0.5cm of va](vl){\rotatebox{-90}{\small $\vee$}};
            \node[left=0.5cm of ea](el){\rotatebox{-90}{\small $\neg a)$}};
            \node[left=0.5cm of na](nl){\rotatebox{-90}{\small $\wedge$}};
            \node[left=0.5cm of ba](bl){\rotatebox{-90}{\small $(b$}};
            \node[left=0.5cm of wa](wl){\rotatebox{-90}{\small $\vee$}};
            \node[left=0.5cm of pa](pl){\rotatebox{-90}{\small $\neg b)$}};

        \end{tikzpicture}
    \end{subfigure}
\end{figure}


        The coalescence algorithm fails to find a solution in 2 dimensions due to the lack of tokens in upper-right or lower-left quadrants.
        If there were tokens in such a place, the tree would be able to continue to grow towards the formula root at the centre of the net.
    \end{example}

    \begin{remark*}
        Implementing firable petri nets is straightforward using an $n$-dimensional boolean array of places visited and a collection of $n$-tuples representing tokens.
        Within the context of coalescence proof search, this can be optimised in many ways, in this case a upper-triangular n-dimensional boolean matrix of places visited and a red-black tree of n-dimensional boolean array tokens can be used.
        This exploits most of the available symmetry in the structure of the nets as, in implementation, memory quickly becomes the limiting factor.
        Furthermore, this structure allows for $\mathcal{O}(\log{}n)$ insertion and removal, $\mathcal{O}(n)$ iteration and minimal memory.
    \end{remark*}


    \begin{proposition*}[Coalescence Proof Search]
        The coalescence algorithm on $P$ is exactly a proof search on $P$.
    \end{proposition*}

    \begin{proof}
        In particular, consider the additively stratified $\vdash P$ with $n - 1$ contractions (should such a proof exist).
        The first configuration of tokens is precisely a proof all possible combinations of the \textit{axiom} rule with $n - 3$ other terms through the \textit{weakening}.
        Each flow transition followed when fired is an application of either the $\vee$ or $\wedge$ rule.
        Finally, a token at the root of the formula is $\vdash P \ldots P$, with the \textit{contraction} rule applied implicitly.

        If there exists an additively stratified proof of $P$, the coalescence algorithm will find it.
        Since for every proof there exists an additively stratified proof, coalescence is precisely proof search.
    \end{proof}

