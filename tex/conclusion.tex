\chapter{Conclusion}
    
    % TODO past tense, consistency, person
        
    \section*{Introduction}
        Building on the work done by \citet{petri-nets}, proof search in classical logic through additive linear logic (ALL) was further investigated, with particular focus as to the complexity of search.
        The process investigated, called \textit{coalescence}, is a top-down proof search from axiom links down to the conclusion.
        This method is promising as it boasts great efficiency for ALL proof search and has a natural transformation to sequent calculus proofs.

    \section*{Sequent Calculus and ALL}
        A classical formula can be proved by an $n$-dimensional additive proof, for some $n$ dependant upon the formula.
        Some simple classes for formulae were proposed --- boolean constants \textit{only} are 1-dimensional, normal additive proofs are 2-dimensional amongst others etc.
        It can be proven that this idea is consistent through \textit{additive stratification} of the sequent calculus --- that is, any sequent proof may be `rearranged' up to the order of rules applied, in particular with \textit{weakening} at the top and \textit{contraction} at the bottom.
        Coalescence is then exactly (additively stratified) proof search.

    \section*{Coalescence}
        A proof search is then constructed through use of `natural' deductions.
        Using a system analogous to \textit{petri nets}, construct a net through the $n$-dimensional cross product of places in a formula and associated transitions across the net.
        Each token in the net begins at an axiom and transitions are exhaustively applied up the formula's syntax tree --- or down an equivalent sequent proof tree.
        A place is a coordinate in the $n$-dimensional grid representing a context of $n$ places in the formula that is provable.
        The process either halts when the root of the formula syntax tree is reached and a proof is constructed, or restarts in a higher dimension when applicable transitions are exhausted and the dimensionality must be increased.
        The dimensionality of a proof is then the dimensionality of our grid when the root is reached.

    \section*{Motivation}
        Clearly complexity scales with dimensionality and the motivating question is then: \textit{`What dimension is sufficient for a given formula?'}.
        In essence, this gives an upper bound for proof search.

    \section*{Some Examples}
        We find that this does not yield results as expected --- for example, in $(a \vee \neg a) \wedge (b \vee \neg b)$ we would expect a dimension of 2 since each component may be proved in 2 dimensions and the conjunction should be trivial.
        We instead find, unsatisfyingly, that this increases dimension quickly and unreasonably, with the aforementioned case yielding dimensionality 3 and linear growth for subsequent additional variables.

    \section*{Solution}
        To solve this issue, we then investigate liberating the search algorithm and generalising over the properties of sequents --- namely, idempotency and commutativity.
        This includes: some notion of applying conjunctions `diagonally' and switching from tuple or multiset links to set links.
        The latter takes us into more familiar/obvious proof search territory.
        
        Beyond these two points, there exist various other implementation trade-offs: dense/sparse representation, high-performance data structures and assorted other ad-hoc optimisations.

    \section*{Further Examples}
        Our optimisation through generalising over sequents yields favourable results for conjunctive normal form (CNF) but maintains poor performance for disjunctive normal form (DNF) formulae.
        We construct a simple algebra of classes, with conjunction and disjunction of proven formulae equivalent to maximum and minimum functions of left and right subformulae dimensionality.
        Finally, we hypothesise a generalised bound for any formula and provide the `essence' of the associated proof --- we expect dimensionality to be equivalent to the most number of variables in any DNF subformula.

    \section*{Ongoing Work}
        We continue to study how coalescence proof search relates to traditional proof search methods, but we expect it to be similar to either the `connections' or `matrix' method --- see: \citet{tableaux-for-logic-of-proofs}, \citet{matrices-with-connections}, \citet{connection-based-proof-method} and \citet{proving-by-matings}.
        Progress is still to be made as to DNF formulae --- we expect generalisation over associativity of ALL terms and automatic construction of subformulae may hold the key to a more natural computation.
   
    Amend~\ref{remark:proof-subtleties}.
