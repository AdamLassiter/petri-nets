\chapter{Introduction}
    
    This paper investigates a natural proof search introduced by \citet{petri-nets}.
    Given a formula composed of conjunctions and disjunctions of variables, it falls into one of three categories: provably true, satisfiably true/false, provably false.
    Satisfiably true or false terms are equivalent to those studied in the boolean satisfiability problem (SAT).
    Since the set of provably false formulae forms the complement to the set of satisfiably true formulae, with provably true similarly forming the complement to satisfiably false, the question of proof search and its complexity ties directly into problems of P versus NP\@.
    The motivating question was `What is the complexity of the chosen proof search'.
    
    The paper begins with a run-through of classical logic and sequent proofs before examining the proof search algorithm in question.
    The research entails studying the properties of this algorithm and implementation details that affect computational complexity, in particular a natural optimisation that gives great performance benefits for certain classes of formulae.
    Finally, the essence of the problem is dissected --- how can the complexity of the proof search be bounded by the structure of the formula?
    
    A C implementation of the described algorithm complements the research and may be found at \url{https://gitlab.com/adamlassiter/petri-nets}.
