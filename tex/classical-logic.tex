\section{Classical Logic}

    \begin{definition*}[Formulae]
        A \textit{formula} within classical logic is constructed as follows:
        \begin{align*}
            A, B, C                \quad &\defeq \quad \top \,|\, \bot \,|\, a \,|\, \neg\, a \,|\, A \vee B \,|\, A \wedge B \\
            \Gamma, \Delta, \Sigma \quad &\defeq \quad A \,|\, A,B \,|\, A,B,C \ldots
        \end{align*}
        where $\vee, \wedge$ are additive linear logic disjunction and conjunction respectively and $\Gamma, \Delta, \Sigma$ are contexts.
    \end{definition*}

    \begin{example}
        \begin{equation*}
            A \seteq a \vee b \quad B \seteq \neg b \vee c \quad C \seteq A \wedge B \equiv (a \vee b) \wedge (\neg b \vee c)
        \end{equation*}
    \end{example}


    \begin{definition*}[Sequent Proofs]
        Within \textit{classical logic}, a \textit{sequent proof} is constructed from the following rules:

        \begin{minipage}[H]{\linewidth}
            \centering
            \begin{minipage}[H]{.3\linewidth}
                \begin{prooftree}
                    \AxiomC{~}
                    \RightLabel{$\top$}
                    \UnaryInfC{$\vdash \top$}
                \end{prooftree}
                \begin{prooftree}
                    \AxiomC{~}
                    \RightLabel{$ax$}
                    \UnaryInfC{$\vdash a, \neg\, a$}
                \end{prooftree}
            \end{minipage}
            \begin{minipage}[H]{.3\linewidth}
                \begin{prooftree}
                    \AxiomC{$\vdash \Gamma, A$}
                    \RightLabel{$\vee$R}
                    \UnaryInfC{$\vdash \Gamma, A \vee B$}
                \end{prooftree}
                \begin{prooftree}
                    \AxiomC{$\vdash \Gamma, A$}
                    \AxiomC{$\vdash \Gamma, B$}
                    \RightLabel{$\wedge$R}
                    \BinaryInfC{$\vdash \Gamma, A \wedge B$}
                \end{prooftree}
            \end{minipage}
            \begin{minipage}[H]{.3\linewidth}
                \begin{prooftree}
                    \AxiomC{$\vdash \Gamma$}
                    \RightLabel{$w$}
                    \UnaryInfC{$\vdash \Gamma, A$}
                \end{prooftree}
                \begin{prooftree}
                    \AxiomC{$\vdash \Gamma, A, A$}
                    \RightLabel{$c$}
                    \UnaryInfC{$\vdash \Gamma, A$}
                \end{prooftree}
            \end{minipage}
        \end{minipage}~\par
        where $A, B, C$ are formulae and $\Gamma, \Delta, \Sigma$ are sequents.
        A sequent proof provides, without context, a proof of its conclusion and each line of the proof represents a tautology.
    \end{definition*}

    \begin{example}
    \end{example}


    \begin{remark*}
        Within the context of weakening and contraction, \textit{additive} and \textit{multiplicative} rules are inter-derivable.
    \end{remark*}


    \begin{definition*}[Derivations]
        Given \textit{tops} $\Gamma_1 \ldots \Gamma_n$ for the sequent proof $\vdash \Delta$, a \textit{derivation} is a tree providing a proof of $\Gamma_1 \ldots \Gamma_n \implies \Delta$.

        A derivation is written as:
        \begin{prooftree}
            \AxiomC{$\vdash \Gamma_1$}
            \AxiomC{$\ldots$}
            \AxiomC{$\vdash \Gamma_n$}
            \RightLabel{\textit{[label]}} \doubleLine\TrinaryInfC{$\vdash \Delta$}
        \end{prooftree}
        where the \textit{label} describes which rules may be used within the derivation.
    \end{definition*}

    \begin{corollary*}[Derivation Equivalence]
        A sequent proof is a derivation where all top derivations of the tree are $=\joinrel= \top, ax$.
        Equivalence of derivations may be weakly defined up to equivalence of leaves and conclusion.
    \end{corollary*}

    \begin{example}
    \end{example}


    \begin{definition*}[Additive Stratification]
        A proof tree is said to be \textit{additively stratified} if $\vdash P$ is structured as follows:
        \begin{prooftree}
            \AxiomC{}
            \RightLabel{$\top, ax$}\doubleLine\UnaryInfC{$\vdash A_1$}
            \RightLabel{$w$}\doubleLine\UnaryInfC{$\vdash \Gamma_1$}
            \AxiomC{\ldots}
            \AxiomC{}
            \RightLabel{$\top, ax$}\doubleLine\UnaryInfC{$\vdash A_n$}
            \RightLabel{$w$}\doubleLine\UnaryInfC{$\vdash \Gamma_n$}
            \RightLabel{$\wedge, \vee$}\doubleLine\TrinaryInfC{$\vdash P \ldots P$}
            \RightLabel{$c$}\doubleLine\UnaryInfC{$\vdash P$}
        \end{prooftree}
        That is, the inferences made in an additively stratified proof are strictly ordered by:
        \begin{enumerate}[nosep]
            \item Top/Axiomatic
            \item Weakening
            \item Conjunction/Disjunction
            \item Contraction
        \end{enumerate}
    \end{definition*}
    
    \begin{example}
    \end{example}


    \begin{proposition*}[Stratification Equivalence]
        Given $\vdash A$, there exists an additively stratified proof of $A$.
    \end{proposition*}
    \begin{proof}
        For each instance of a weakening below another inference, there exists an equivalent subproof that is additively stratified:

        \begin{minipage}[H]{\linewidth}
            \centering
            \begin{minipage}[H]{0.4\linewidth}
                \begin{prooftree}
                    \AxiomC{$\vdash \Gamma, A, B$}
                    \RightLabel{$\vee$}\UnaryInfC{$\vdash \Gamma, A \vee B$}
                    \RightLabel{$w$}\UnaryInfC{$\vdash \Gamma, A \vee B$, C}
                \end{prooftree}
            \end{minipage}
            $\leadsto$
            \begin{minipage}[H]{0.4\linewidth}
                \begin{prooftree}
                    \AxiomC{$\vdash \Gamma, A, B$}
                    \RightLabel{$w$}\UnaryInfC{$\vdash \Gamma, A, B, C$}
                    \RightLabel{$\vee$}\UnaryInfC{$\vdash \Gamma, A \vee B$, C}
                \end{prooftree}
            \end{minipage}
        \end{minipage}
        
        \begin{minipage}[H]{\linewidth}
            \centering
            \begin{minipage}[H]{0.4\linewidth}
                \begin{prooftree}
                    \AxiomC{$\vdash \Gamma, A$}
                    \AxiomC{$\vdash \Gamma, B$}
                    \RightLabel{$\wedge$}\BinaryInfC{$\vdash \Gamma, A \wedge B$}
                    \RightLabel{$w$}\UnaryInfC{$\vdash \Gamma, A \wedge B, C$}
                \end{prooftree}
            \end{minipage}
            $\leadsto\quad$
            \begin{minipage}[H]{0.4\linewidth}
                \begin{prooftree}
                    \AxiomC{$\vdash \Gamma, A$}
                    \RightLabel{$w$}\UnaryInfC{$\vdash \Gamma, A, C$}
                    \AxiomC{$\vdash \Gamma, B$}
                    \RightLabel{$w$}\UnaryInfC{$\vdash \Gamma, B, C$}
                    \RightLabel{$\wedge$}\BinaryInfC{$\vdash \Gamma, A \wedge B, C$}
                \end{prooftree}
            \end{minipage}
        \end{minipage}

        \begin{minipage}[H]{\linewidth}
            \centering
            \begin{minipage}[H]{0.4\linewidth}
                \begin{prooftree}
                    \AxiomC{$\vdash \Gamma, A, A$}
                    \RightLabel{$c$}\UnaryInfC{$\vdash \Gamma, A$}
                    \RightLabel{$w$}\UnaryInfC{$\vdash \Gamma, A, B$}
                \end{prooftree}
            \end{minipage}
            $\leadsto\quad$
            \begin{minipage}[H]{0.4\linewidth}
                \begin{prooftree}
                    \AxiomC{$\vdash \Gamma, A, A$}
                    \RightLabel{$w$}\UnaryInfC{$\vdash \Gamma, A, A, B$}
                    \RightLabel{$c$}\UnaryInfC{$\vdash \Gamma, A, B$}
                \end{prooftree}
            \end{minipage}
        \end{minipage}

        Similarly, for each instance of a contraction above another inference, there exists an equivalent subproof that is additively stratified:

        \begin{minipage}[H]{\linewidth}
            \centering
            \begin{minipage}[H]{0.4\linewidth}
                \begin{prooftree}
                    \AxiomC{$\vdash \Gamma, A, A, B$}
                    \RightLabel{$c$}\UnaryInfC{$\vdash \Gamma, A, B$}
                    \RightLabel{$\vee$}\UnaryInfC{$\vdash \Gamma, A \vee B$}
                \end{prooftree}
            \end{minipage}
            $\leadsto$
            \begin{minipage}[H]{0.4\linewidth}
                \begin{prooftree}
                    \AxiomC{$\vdash \Gamma, A, A, B$}
                    \RightLabel{$w$}\UnaryInfC{$\vdash \Gamma, A, A, B, B$}
                    \RightLabel{$\vee$}\UnaryInfC{$\vdash \Gamma, A \vee B, A, B$}
                    \RightLabel{$\vee$}\UnaryInfC{$\vdash \Gamma, A \vee B, A \vee B$}
                    \RightLabel{$c$}\UnaryInfC{$\vdash \Gamma, A \vee B$}
                \end{prooftree}
            \end{minipage}
        \end{minipage}

        \begin{minipage}[H]{\linewidth}
            \centering
            \begin{minipage}[H]{0.3\linewidth}
                \begin{prooftree}
                    \AxiomC{$\vdash \Gamma, A, A$}
                    \RightLabel{$c$}\UnaryInfC{$\vdash \Gamma, A$}
                    \AxiomC{$\vdash \Gamma, B$}
                    \RightLabel{$\wedge$}\BinaryInfC{$\vdash \Gamma, A \wedge B$}
                \end{prooftree}
            \end{minipage}
            $\leadsto\quad$
            \begin{minipage}[H]{0.6\linewidth}
                \begin{prooftree}
                    \AxiomC{$\vdash \Gamma, A, A$}
                    \AxiomC{$\vdash \Gamma, B$}
                    \RightLabel{$w$}\UnaryInfC{$\vdash \Gamma, A, B$}
                    \RightLabel{$\wedge$}\BinaryInfC{$\vdash \Gamma, A, A \wedge B$}
                    \AxiomC{$\vdash \Gamma, B$}
                    \RightLabel{$w$}\UnaryInfC{$\vdash \Gamma, B, A \wedge B$}
                    \RightLabel{$\wedge$}\BinaryInfC{$\vdash \Gamma, A \wedge B, A \wedge B$}
                    \RightLabel{$c$}\UnaryInfC{$\vdash \Gamma, A \wedge B$}
                \end{prooftree}
            \end{minipage}
        \end{minipage}
        
        By induction from the leaves downwards on a finite height tree, apply the associated rule to each pair of inferences of the form ($c$ above \textit{inf}).
        Any given $\vdash P$ may be rewritten:
        \begin{prooftree}
            \AxiomC{}
            \RightLabel{$\top, ax$}\doubleLine\UnaryInfC{$\vdash A_1$}
            \RightLabel{$\wedge, \vee, w$}\doubleLine\UnaryInfC{$\vdash \Gamma_1$}
            \AxiomC{\ldots}
            \AxiomC{}
            \RightLabel{$\top, ax$}\doubleLine\UnaryInfC{$\vdash A_n$}
            \RightLabel{$\wedge, \vee, w$}\doubleLine\UnaryInfC{$\vdash \Gamma_n$}
            \RightLabel{$c$}\doubleLine\TrinaryInfC{$\vdash P$}
        \end{prooftree}
        
        Again, by induction from the root upwards on this partially stratified tree, apply the associated rule to each pair of inferences of the form ($w$ below \textit{inf}).
        $\vdash P$ may then be further rewritten:
        \begin{prooftree}
            \AxiomC{}
            \RightLabel{$\top, ax$}\doubleLine\UnaryInfC{$\vdash A_1$}
            \RightLabel{$w$}\doubleLine\UnaryInfC{$\vdash \Gamma_1$}
            \AxiomC{\ldots}
            \AxiomC{}
            \RightLabel{$\top, ax$}\doubleLine\UnaryInfC{$\vdash A_n$}
            \RightLabel{$w$}\doubleLine\UnaryInfC{$\vdash \Gamma_n$}
            \RightLabel{$\wedge, \vee$}\doubleLine\TrinaryInfC{$\vdash P \ldots P$}
            \RightLabel{$c$}\doubleLine\UnaryInfC{$\vdash P$}
        \end{prooftree}

    \end{proof}

    \begin{example}
    \end{example}
