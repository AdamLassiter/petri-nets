\subsection{Solo Calculus}

    Developed by Cosimo Laneve and Bj{\"o}rn Victor in the early 2000s, the Solo Calculus aims to be an improvement of the Fusion Calculus.
    As such, there exists an encoding of the Fusion Calculus within the Solo Calculus (and hence an encoding of the $\pi$-calculus).
    The name comes from the strong distinction between the components of the calculus: \textit{solos} and \textit{agents}.
    These are roughly analogous to input/output actions and a calculus syntax similar to the $\pi$-calculus.
    Through some clever design choices, the Solo Calculus is found to have some interesting properties over other process calculi.

    \begin{definition}{Syntax\\}
        \label{solo-calculus-syntax}
        As defined by~\cite{solo-calculus}, the Solo Calculus is constructed from \textit{solos} ranged over by $\alpha, \beta \ldots$ and \textit{agents} ranged over by $P, Q \ldots$ as such:
        \begin{center}
            \begin{tabular}{ l l l }
                $\alpha \quad \defeq$   & $u \, \tilde{x}$          & (input) \\
                                        & $\bar{u} \, \tilde{x}$    & (output)~\footnotemark\\ \\
                $P \quad \defeq$        & $0$                       & (inaction) \\
                                        & $\alpha$                  & (solo) \\
                                        & $Q \; | \; R$             & (composition) \\
                                        & $(x) \, Q$                & (scope) \\
                                        & $!\,P$                    & (replication)
            \end{tabular}
        \end{center}\footnotetext{$\tilde{x}$ is used as shorthand for any tuple $(x_1 \ldots x_n)$.}
        where the scope operator $(x) \, P $ is a declaration of the named variable $x$ in $P$.
        This ensures that $x$ is local to $P$, even if it assigned outside $P$ (ex. $(x \, y | (x) \, P)$ will never have $x \defeq y$ unless explicitly assigned such in P).
    \end{definition}
    This is a much more minimal syntax when compared to CCS (and certainly Higher-Order CCS as described by~\cite{pi-calculus-in-ccs}).
    It will further be seen that the reduction rules retain this simplicity.
    It should be noted that the names $u, x, etc\ldots$ within a solo may be treated as both channel names and as values.


    \begin{remark*}{Match Operator\\}
        The full definition by~\cite{solo-calculus} also includes the match operator $[x \, = \, y] \, P$ which computes $P$ if $x$ and $y$ are the same name, otherwise computes \textbf{0}.
        It has been shown that the inclusion of the match operator is in fact extraneous, so it is excluded.
    \end{remark*}


    \begin{definition}{Structural Congruence\\}
        The structural congruence relation $\equiv$ in the Solo Calculus is exactly that defined in Definition~\ref{fusion-calculus-structural-congruence}.
    \end{definition}


    \begin{definition}{Reduction\\}
        Reduction semantics on solo expressions are defined as:
        \begin{align*}
            (x)(\bar{u} \, x \; | \; u \, y \; | \; P) \rightarrow & \, P\{y / x\} \\
            P \rightarrow P' \implies &
            \begin{cases}
                P \; | \; Q \rightarrow P' \; | \; Q \\
                (x) \, P \rightarrow (x) \, P' \\
                P \equiv Q \text{ and } P' \equiv Q' \implies Q \rightarrow Q'
            \end{cases}
        \end{align*}
        where $P\{y / x\}$ is $\alpha$-substitution of the name $x$ to the name $y$.
    \end{definition}
    The case of reducing $!P \rightarrow P \; | \; !P$ is rarely performed due to the potential explosion of terms.
    Instead, it is subject to lazy evaluation --- the expansion is only performed in a reduction step if it is known that there exists an agent in $P$ that may perform a reduction.
    Details of this will be seen later, or can be found discussed by~\cite{solo-diagrams}.
    It is interesting to note also the asynchronous behaviour of the Solo Calculus.
    Where in the $\pi$-calculus and CCS input/output actions where synchronised and preceded processes as guards, the Solo Calculus naturally treats all agents as unguarded and names may be substituted whenever is desired.
    That is, there is no sequential aspect to the calculus, allowing for useful structural congruences, and terms may be evaluated exceptionally lazily.


    \begin{remark*}
        There exists an encoding of the Fusion Calculus within the Solo Calculus.
        This can most easily be seen as an encoding of the choice-free Fusion Calculus as a combination of the above syntax and semantics of the Solo Calculus and also the prefix operator $\alpha \, . \, P$\footnotemark
        \footnotetext{For further details,~\cite{solo-calculus} discuss this implementation in Section 3}.
        Hence there exists an encoding of the $\pi$-calculus also, complete with the same style of guarded input/output communication.
    \end{remark*}
